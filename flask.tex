
% Default to the notebook output style

    


% Inherit from the specified cell style.




    
\documentclass{article}

    
    
    \usepackage{graphicx} % Used to insert images
    \usepackage{adjustbox} % Used to constrain images to a maximum size 
    \usepackage{color} % Allow colors to be defined
    \usepackage{enumerate} % Needed for markdown enumerations to work
    \usepackage{geometry} % Used to adjust the document margins
    \usepackage{amsmath} % Equations
    \usepackage{amssymb} % Equations
    \usepackage[mathletters]{ucs} % Extended unicode (utf-8) support
    \usepackage[utf8x]{inputenc} % Allow utf-8 characters in the tex document
    \usepackage{fancyvrb} % verbatim replacement that allows latex
    \usepackage{grffile} % extends the file name processing of package graphics 
                         % to support a larger range 
    % The hyperref package gives us a pdf with properly built
    % internal navigation ('pdf bookmarks' for the table of contents,
    % internal cross-reference links, web links for URLs, etc.)
    \usepackage{hyperref}
    \usepackage{longtable} % longtable support required by pandoc >1.10
    \usepackage{booktabs}  % table support for pandoc > 1.12.2
    

    
    
    \definecolor{orange}{cmyk}{0,0.4,0.8,0.2}
    \definecolor{darkorange}{rgb}{.71,0.21,0.01}
    \definecolor{darkgreen}{rgb}{.12,.54,.11}
    \definecolor{myteal}{rgb}{.26, .44, .56}
    \definecolor{gray}{gray}{0.45}
    \definecolor{lightgray}{gray}{.95}
    \definecolor{mediumgray}{gray}{.8}
    \definecolor{inputbackground}{rgb}{.95, .95, .85}
    \definecolor{outputbackground}{rgb}{.95, .95, .95}
    \definecolor{traceback}{rgb}{1, .95, .95}
    % ansi colors
    \definecolor{red}{rgb}{.6,0,0}
    \definecolor{green}{rgb}{0,.65,0}
    \definecolor{brown}{rgb}{0.6,0.6,0}
    \definecolor{blue}{rgb}{0,.145,.698}
    \definecolor{purple}{rgb}{.698,.145,.698}
    \definecolor{cyan}{rgb}{0,.698,.698}
    \definecolor{lightgray}{gray}{0.5}
    
    % bright ansi colors
    \definecolor{darkgray}{gray}{0.25}
    \definecolor{lightred}{rgb}{1.0,0.39,0.28}
    \definecolor{lightgreen}{rgb}{0.48,0.99,0.0}
    \definecolor{lightblue}{rgb}{0.53,0.81,0.92}
    \definecolor{lightpurple}{rgb}{0.87,0.63,0.87}
    \definecolor{lightcyan}{rgb}{0.5,1.0,0.83}
    
    % commands and environments needed by pandoc snippets
    % extracted from the output of `pandoc -s`
    \DefineVerbatimEnvironment{Highlighting}{Verbatim}{commandchars=\\\{\}}
    % Add ',fontsize=\small' for more characters per line
    \newenvironment{Shaded}{}{}
    \newcommand{\KeywordTok}[1]{\textcolor[rgb]{0.00,0.44,0.13}{\textbf{{#1}}}}
    \newcommand{\DataTypeTok}[1]{\textcolor[rgb]{0.56,0.13,0.00}{{#1}}}
    \newcommand{\DecValTok}[1]{\textcolor[rgb]{0.25,0.63,0.44}{{#1}}}
    \newcommand{\BaseNTok}[1]{\textcolor[rgb]{0.25,0.63,0.44}{{#1}}}
    \newcommand{\FloatTok}[1]{\textcolor[rgb]{0.25,0.63,0.44}{{#1}}}
    \newcommand{\CharTok}[1]{\textcolor[rgb]{0.25,0.44,0.63}{{#1}}}
    \newcommand{\StringTok}[1]{\textcolor[rgb]{0.25,0.44,0.63}{{#1}}}
    \newcommand{\CommentTok}[1]{\textcolor[rgb]{0.38,0.63,0.69}{\textit{{#1}}}}
    \newcommand{\OtherTok}[1]{\textcolor[rgb]{0.00,0.44,0.13}{{#1}}}
    \newcommand{\AlertTok}[1]{\textcolor[rgb]{1.00,0.00,0.00}{\textbf{{#1}}}}
    \newcommand{\FunctionTok}[1]{\textcolor[rgb]{0.02,0.16,0.49}{{#1}}}
    \newcommand{\RegionMarkerTok}[1]{{#1}}
    \newcommand{\ErrorTok}[1]{\textcolor[rgb]{1.00,0.00,0.00}{\textbf{{#1}}}}
    \newcommand{\NormalTok}[1]{{#1}}
    
    % Define a nice break command that doesn't care if a line doesn't already
    % exist.
    \def\br{\hspace*{\fill} \\* }
    % Math Jax compatability definitions
    \def\gt{>}
    \def\lt{<}
    % Document parameters
    \title{flask}
    
    
    

    % Pygments definitions
    
\makeatletter
\def\PY@reset{\let\PY@it=\relax \let\PY@bf=\relax%
    \let\PY@ul=\relax \let\PY@tc=\relax%
    \let\PY@bc=\relax \let\PY@ff=\relax}
\def\PY@tok#1{\csname PY@tok@#1\endcsname}
\def\PY@toks#1+{\ifx\relax#1\empty\else%
    \PY@tok{#1}\expandafter\PY@toks\fi}
\def\PY@do#1{\PY@bc{\PY@tc{\PY@ul{%
    \PY@it{\PY@bf{\PY@ff{#1}}}}}}}
\def\PY#1#2{\PY@reset\PY@toks#1+\relax+\PY@do{#2}}

\expandafter\def\csname PY@tok@gd\endcsname{\def\PY@tc##1{\textcolor[rgb]{0.63,0.00,0.00}{##1}}}
\expandafter\def\csname PY@tok@gu\endcsname{\let\PY@bf=\textbf\def\PY@tc##1{\textcolor[rgb]{0.50,0.00,0.50}{##1}}}
\expandafter\def\csname PY@tok@gt\endcsname{\def\PY@tc##1{\textcolor[rgb]{0.00,0.27,0.87}{##1}}}
\expandafter\def\csname PY@tok@gs\endcsname{\let\PY@bf=\textbf}
\expandafter\def\csname PY@tok@gr\endcsname{\def\PY@tc##1{\textcolor[rgb]{1.00,0.00,0.00}{##1}}}
\expandafter\def\csname PY@tok@cm\endcsname{\let\PY@it=\textit\def\PY@tc##1{\textcolor[rgb]{0.25,0.50,0.50}{##1}}}
\expandafter\def\csname PY@tok@vg\endcsname{\def\PY@tc##1{\textcolor[rgb]{0.10,0.09,0.49}{##1}}}
\expandafter\def\csname PY@tok@m\endcsname{\def\PY@tc##1{\textcolor[rgb]{0.40,0.40,0.40}{##1}}}
\expandafter\def\csname PY@tok@mh\endcsname{\def\PY@tc##1{\textcolor[rgb]{0.40,0.40,0.40}{##1}}}
\expandafter\def\csname PY@tok@go\endcsname{\def\PY@tc##1{\textcolor[rgb]{0.53,0.53,0.53}{##1}}}
\expandafter\def\csname PY@tok@ge\endcsname{\let\PY@it=\textit}
\expandafter\def\csname PY@tok@vc\endcsname{\def\PY@tc##1{\textcolor[rgb]{0.10,0.09,0.49}{##1}}}
\expandafter\def\csname PY@tok@il\endcsname{\def\PY@tc##1{\textcolor[rgb]{0.40,0.40,0.40}{##1}}}
\expandafter\def\csname PY@tok@cs\endcsname{\let\PY@it=\textit\def\PY@tc##1{\textcolor[rgb]{0.25,0.50,0.50}{##1}}}
\expandafter\def\csname PY@tok@cp\endcsname{\def\PY@tc##1{\textcolor[rgb]{0.74,0.48,0.00}{##1}}}
\expandafter\def\csname PY@tok@gi\endcsname{\def\PY@tc##1{\textcolor[rgb]{0.00,0.63,0.00}{##1}}}
\expandafter\def\csname PY@tok@gh\endcsname{\let\PY@bf=\textbf\def\PY@tc##1{\textcolor[rgb]{0.00,0.00,0.50}{##1}}}
\expandafter\def\csname PY@tok@ni\endcsname{\let\PY@bf=\textbf\def\PY@tc##1{\textcolor[rgb]{0.60,0.60,0.60}{##1}}}
\expandafter\def\csname PY@tok@nl\endcsname{\def\PY@tc##1{\textcolor[rgb]{0.63,0.63,0.00}{##1}}}
\expandafter\def\csname PY@tok@nn\endcsname{\let\PY@bf=\textbf\def\PY@tc##1{\textcolor[rgb]{0.00,0.00,1.00}{##1}}}
\expandafter\def\csname PY@tok@no\endcsname{\def\PY@tc##1{\textcolor[rgb]{0.53,0.00,0.00}{##1}}}
\expandafter\def\csname PY@tok@na\endcsname{\def\PY@tc##1{\textcolor[rgb]{0.49,0.56,0.16}{##1}}}
\expandafter\def\csname PY@tok@nb\endcsname{\def\PY@tc##1{\textcolor[rgb]{0.00,0.50,0.00}{##1}}}
\expandafter\def\csname PY@tok@nc\endcsname{\let\PY@bf=\textbf\def\PY@tc##1{\textcolor[rgb]{0.00,0.00,1.00}{##1}}}
\expandafter\def\csname PY@tok@nd\endcsname{\def\PY@tc##1{\textcolor[rgb]{0.67,0.13,1.00}{##1}}}
\expandafter\def\csname PY@tok@ne\endcsname{\let\PY@bf=\textbf\def\PY@tc##1{\textcolor[rgb]{0.82,0.25,0.23}{##1}}}
\expandafter\def\csname PY@tok@nf\endcsname{\def\PY@tc##1{\textcolor[rgb]{0.00,0.00,1.00}{##1}}}
\expandafter\def\csname PY@tok@si\endcsname{\let\PY@bf=\textbf\def\PY@tc##1{\textcolor[rgb]{0.73,0.40,0.53}{##1}}}
\expandafter\def\csname PY@tok@s2\endcsname{\def\PY@tc##1{\textcolor[rgb]{0.73,0.13,0.13}{##1}}}
\expandafter\def\csname PY@tok@vi\endcsname{\def\PY@tc##1{\textcolor[rgb]{0.10,0.09,0.49}{##1}}}
\expandafter\def\csname PY@tok@nt\endcsname{\let\PY@bf=\textbf\def\PY@tc##1{\textcolor[rgb]{0.00,0.50,0.00}{##1}}}
\expandafter\def\csname PY@tok@nv\endcsname{\def\PY@tc##1{\textcolor[rgb]{0.10,0.09,0.49}{##1}}}
\expandafter\def\csname PY@tok@s1\endcsname{\def\PY@tc##1{\textcolor[rgb]{0.73,0.13,0.13}{##1}}}
\expandafter\def\csname PY@tok@kd\endcsname{\let\PY@bf=\textbf\def\PY@tc##1{\textcolor[rgb]{0.00,0.50,0.00}{##1}}}
\expandafter\def\csname PY@tok@sh\endcsname{\def\PY@tc##1{\textcolor[rgb]{0.73,0.13,0.13}{##1}}}
\expandafter\def\csname PY@tok@sc\endcsname{\def\PY@tc##1{\textcolor[rgb]{0.73,0.13,0.13}{##1}}}
\expandafter\def\csname PY@tok@sx\endcsname{\def\PY@tc##1{\textcolor[rgb]{0.00,0.50,0.00}{##1}}}
\expandafter\def\csname PY@tok@bp\endcsname{\def\PY@tc##1{\textcolor[rgb]{0.00,0.50,0.00}{##1}}}
\expandafter\def\csname PY@tok@c1\endcsname{\let\PY@it=\textit\def\PY@tc##1{\textcolor[rgb]{0.25,0.50,0.50}{##1}}}
\expandafter\def\csname PY@tok@kc\endcsname{\let\PY@bf=\textbf\def\PY@tc##1{\textcolor[rgb]{0.00,0.50,0.00}{##1}}}
\expandafter\def\csname PY@tok@c\endcsname{\let\PY@it=\textit\def\PY@tc##1{\textcolor[rgb]{0.25,0.50,0.50}{##1}}}
\expandafter\def\csname PY@tok@mf\endcsname{\def\PY@tc##1{\textcolor[rgb]{0.40,0.40,0.40}{##1}}}
\expandafter\def\csname PY@tok@err\endcsname{\def\PY@bc##1{\setlength{\fboxsep}{0pt}\fcolorbox[rgb]{1.00,0.00,0.00}{1,1,1}{\strut ##1}}}
\expandafter\def\csname PY@tok@mb\endcsname{\def\PY@tc##1{\textcolor[rgb]{0.40,0.40,0.40}{##1}}}
\expandafter\def\csname PY@tok@ss\endcsname{\def\PY@tc##1{\textcolor[rgb]{0.10,0.09,0.49}{##1}}}
\expandafter\def\csname PY@tok@sr\endcsname{\def\PY@tc##1{\textcolor[rgb]{0.73,0.40,0.53}{##1}}}
\expandafter\def\csname PY@tok@mo\endcsname{\def\PY@tc##1{\textcolor[rgb]{0.40,0.40,0.40}{##1}}}
\expandafter\def\csname PY@tok@kn\endcsname{\let\PY@bf=\textbf\def\PY@tc##1{\textcolor[rgb]{0.00,0.50,0.00}{##1}}}
\expandafter\def\csname PY@tok@mi\endcsname{\def\PY@tc##1{\textcolor[rgb]{0.40,0.40,0.40}{##1}}}
\expandafter\def\csname PY@tok@gp\endcsname{\let\PY@bf=\textbf\def\PY@tc##1{\textcolor[rgb]{0.00,0.00,0.50}{##1}}}
\expandafter\def\csname PY@tok@o\endcsname{\def\PY@tc##1{\textcolor[rgb]{0.40,0.40,0.40}{##1}}}
\expandafter\def\csname PY@tok@kr\endcsname{\let\PY@bf=\textbf\def\PY@tc##1{\textcolor[rgb]{0.00,0.50,0.00}{##1}}}
\expandafter\def\csname PY@tok@s\endcsname{\def\PY@tc##1{\textcolor[rgb]{0.73,0.13,0.13}{##1}}}
\expandafter\def\csname PY@tok@kp\endcsname{\def\PY@tc##1{\textcolor[rgb]{0.00,0.50,0.00}{##1}}}
\expandafter\def\csname PY@tok@w\endcsname{\def\PY@tc##1{\textcolor[rgb]{0.73,0.73,0.73}{##1}}}
\expandafter\def\csname PY@tok@kt\endcsname{\def\PY@tc##1{\textcolor[rgb]{0.69,0.00,0.25}{##1}}}
\expandafter\def\csname PY@tok@ow\endcsname{\let\PY@bf=\textbf\def\PY@tc##1{\textcolor[rgb]{0.67,0.13,1.00}{##1}}}
\expandafter\def\csname PY@tok@sb\endcsname{\def\PY@tc##1{\textcolor[rgb]{0.73,0.13,0.13}{##1}}}
\expandafter\def\csname PY@tok@k\endcsname{\let\PY@bf=\textbf\def\PY@tc##1{\textcolor[rgb]{0.00,0.50,0.00}{##1}}}
\expandafter\def\csname PY@tok@se\endcsname{\let\PY@bf=\textbf\def\PY@tc##1{\textcolor[rgb]{0.73,0.40,0.13}{##1}}}
\expandafter\def\csname PY@tok@sd\endcsname{\let\PY@it=\textit\def\PY@tc##1{\textcolor[rgb]{0.73,0.13,0.13}{##1}}}

\def\PYZbs{\char`\\}
\def\PYZus{\char`\_}
\def\PYZob{\char`\{}
\def\PYZcb{\char`\}}
\def\PYZca{\char`\^}
\def\PYZam{\char`\&}
\def\PYZlt{\char`\<}
\def\PYZgt{\char`\>}
\def\PYZsh{\char`\#}
\def\PYZpc{\char`\%}
\def\PYZdl{\char`\$}
\def\PYZhy{\char`\-}
\def\PYZsq{\char`\'}
\def\PYZdq{\char`\"}
\def\PYZti{\char`\~}
% for compatibility with earlier versions
\def\PYZat{@}
\def\PYZlb{[}
\def\PYZrb{]}
\makeatother


    % Exact colors from NB
    \definecolor{incolor}{rgb}{0.0, 0.0, 0.5}
    \definecolor{outcolor}{rgb}{0.545, 0.0, 0.0}



    
    % Prevent overflowing lines due to hard-to-break entities
    \sloppy 
    % Setup hyperref package
    \hypersetup{
      breaklinks=true,  % so long urls are correctly broken across lines
      colorlinks=true,
      urlcolor=blue,
      linkcolor=darkorange,
      citecolor=darkgreen,
      }
    % Slightly bigger margins than the latex defaults
    
    \geometry{verbose,tmargin=1in,bmargin=1in,lmargin=1in,rmargin=1in}
    
    

    \begin{document}
    
    
    \maketitle
    
    

    

    \section{5 Applications Web et python}


    Classiquement sur internet on utilise un serveur \textbf{\emph{lamp}}
:\\\textbf{l}inux (os), \textbf{a}pache (serveur), \textbf{m}ysql (base
de données), \textbf{p}hp (language de programmation pour avoir des
pages dynamiques).

Il existe de nombreux outils pour le web écrit en python: \emph{serveur
Web} (Zope, gunicorn) ; \emph{script cgi}; \emph{frameworks Web} (Flask,
Django, cherrypy etc \ldots{},

nous utiliserons le framework \textbf{\emph{flask}}. Pour se documenter
je vous conseille deux sites.\\- \url{http://flask.pocoo.org/docs} -
\url{http://openclassrooms.com/courses/creez-vos-applications-web-avec-flask}

\subsection{5.1 Une page web avec flask}\label{une-page-web-avec-flask}

\begin{itemize}
\itemsep1pt\parskip0pt\parsep0pt
\item
  Télécharger le fichier
  \url{https://github.com/debimax/cours-debimax/raw/master/documents/isn-flask.tar.gz}.
\item
  Décompressez le (en console : tar zxvf isn-flask.tar.gz).
\end{itemize}

\subsubsection{5.1.1 Flask et les
templates}\label{flask-et-les-templates}

\paragraph{a) Le principe}\label{a-le-principe}

Avec \emph{Flask} on pourrait mettre le code dans un seul
\emph{fichier.py} mais on utilise de préférence des \emph{templates}
(avec \emph{jinja2}).

L'arborescence d'un projet Flask sera :

\begin{verbatim}
   projet/script.py
   projet/static/
   projet/templates
   projet/templates/les_templates.html
\end{verbatim}

On met tous les templates dans le dossier \textbf{\emph{templates/}}. Le
dossier \textbf{\emph{static/}} contiendra lui toutes les images,
fichiers css, js \ldots{}

Ouvrez avec \emph{geany} le fichier \textbf{\emph{exemple.py}} :

\begin{Shaded}
\begin{Highlighting}[]
 \DecValTok{1} \CommentTok{#!/usr/bin/python3}
 \DecValTok{2} \CommentTok{# -*- coding: utf-8 -*-}
 \DecValTok{3} \CharTok{from} \NormalTok{flask }\CharTok{import} \NormalTok{Flask, render_template, url_for}
 \DecValTok{4} \NormalTok{app = Flask(}\OtherTok{__name__}\NormalTok{)   }\CommentTok{# Initialise l'application Flask}
 \DecValTok{5}
 \DecValTok{6} \NormalTok{@app.route(}\StringTok{'/'}\NormalTok{)}
 \DecValTok{7} \KeywordTok{def} \NormalTok{accueil():}
 \DecValTok{8}     \NormalTok{lignes=[}\StringTok{'ligne \{\}'}\NormalTok{.}\DataTypeTok{format}\NormalTok{(i) }\KeywordTok{for} \NormalTok{i in }\DataTypeTok{range}\NormalTok{(}\DecValTok{1}\NormalTok{,}\DecValTok{10}\NormalTok{)]}
 \DecValTok{9}     \KeywordTok{return} \NormalTok{render_template(}\StringTok{"accueil.html"}\NormalTok{, titre=}\StringTok{"Bienvenue !"}\NormalTok{,lignes=lignes)}
\DecValTok{10}
\DecValTok{11} \KeywordTok{if} \OtherTok{__name__} \NormalTok{== __main__ :}
\DecValTok{12}     \NormalTok{app.run(debug=}\OtherTok{True}\NormalTok{)}
\end{Highlighting}
\end{Shaded}

Dans le code ci-dessous, quel est le contenu de la variable
\textbf{\emph{lignes}}?

\begin{Shaded}
\begin{Highlighting}[]
 \NormalTok{liges=[}\StringTok{'ligne\{\}'}\NormalTok{.}\DataTypeTok{format}\NormalTok{(i) }\KeywordTok{for} \NormalTok{i in }\DataTypeTok{range}\NormalTok{(}\DecValTok{1}\NormalTok{,}\DecValTok{10}\NormalTok{)]}
\end{Highlighting}
\end{Shaded}

\ldots{}\ldots{}\ldots{}\ldots{}\ldots{}\ldots{}\ldots{}\ldots{}\ldots{}\ldots{}\ldots{}\ldots{}\ldots{}\ldots{}\ldots{}\ldots{}\ldots{}\ldots{}\ldots{}\ldots{}\ldots{}\ldots{}\ldots{}\ldots{}\ldots{}\ldots{}\ldots{}\ldots{}\ldots{}\ldots{}\ldots{}\ldots{}\ldots{}\ldots{}\ldots{}\ldots{}\ldots{}

Ouvrez avec \emph{geany} maintenant le fichier
\textbf{\emph{templates/accueil.html}}.

\begin{Shaded}
\begin{Highlighting}[]
 \NormalTok{1 }\DataTypeTok{<!DOCTYPE }\NormalTok{html}\DataTypeTok{>}
 \NormalTok{2 }\KeywordTok{<html>}
 \NormalTok{3 }\KeywordTok{<head>}
 \NormalTok{4 }\KeywordTok{<meta}\OtherTok{ charset=}\StringTok{"utf-8"} \KeywordTok{/>}
 \NormalTok{5 }\KeywordTok{<title>}\NormalTok{\{\{ titre \}\}}\KeywordTok{</title>}
 \NormalTok{6 }\KeywordTok{</head>}
 \NormalTok{7 }\KeywordTok{<body>}
 \NormalTok{8 }\KeywordTok{<h1>}\NormalTok{\{\{ titre \}\}}\KeywordTok{</h1>}
 \NormalTok{9 }\KeywordTok{<ul>}
\NormalTok{10 \{% for ligne in lignes %\}}
\NormalTok{11 }\KeywordTok{<li>}\NormalTok{\{\{ ligne \}\}}\KeywordTok{</li>}
\NormalTok{12 \{% endfor %\}}
\NormalTok{13 }\KeywordTok{</ul>}
\NormalTok{14 }\KeywordTok{</body>}
\NormalTok{15 }\KeywordTok{</html>}
\end{Highlighting}
\end{Shaded}

Ouvrez une console dans \textbf{\emph{/isn-flask/}} puis tapez ***
python3.4 exemple.py***\\Ouvrez alors un navigateur internet à l'adresse
\url{http://localhost:5000} (celui qui n'est pas écrit pablo).

\paragraph{b Ajouter la date}\label{b-ajouter-la-date}

Il faudra importer la librairie \textbf{\emph{time}} puis utiliser le
décorateur \textbf{\emph{@app.context\_processor}} pour passer l'heure à
tous les templates.

\begin{Shaded}
\begin{Highlighting}[]
 \DecValTok{3} \CharTok{from} \NormalTok{flask }\CharTok{import} \NormalTok{Flask, render_template, url_for}
 \DecValTok{4} \CharTok{import} \NormalTok{time}
 \DecValTok{5} \NormalTok{app = Flask(}\OtherTok{__name__}\NormalTok{) }\CommentTok{# Initialise l'application Flask}
 \DecValTok{6} 
 \DecValTok{7} \NormalTok{@app.context_processor}
 \DecValTok{8} \KeywordTok{def} \NormalTok{passer_date_heure():}
 \DecValTok{9}     \NormalTok{date=time.strftime(}\StringTok{'}\OtherTok{%d}\StringTok{/%m/%Y'}\NormalTok{,time.localtime())}
\DecValTok{10}     \NormalTok{heure=time.strftime(}\StringTok{'%H'}\NormalTok{,time.localtime())}
\DecValTok{11}     \KeywordTok{return} \DataTypeTok{dict}\NormalTok{(date=date,heure=heure)}
\DecValTok{12} 
\DecValTok{13} \NormalTok{@app.route(}\StringTok{'/'}\NormalTok{)}
\end{Highlighting}
\end{Shaded}

Il faut maintenant modifier le template \textbf{\emph{accueil.html}}

\begin{Shaded}
\begin{Highlighting}[]
\NormalTok{13     }\KeywordTok{</ul>}
\NormalTok{14    }\KeywordTok{<footer>}\NormalTok{Nous somme le \{\{ date \}\} et il est \{\{ heure \}\} heures.}\KeywordTok{</footer>}
\NormalTok{15     }\KeywordTok{</body>}
\end{Highlighting}
\end{Shaded}

Question: Pourquoi est il peu judicieux d'utiliser le serveur python
pour afficher l'heure?

\ldots{}\ldots{}\ldots{}\ldots{}\ldots{}\ldots{}\ldots{}\ldots{}\ldots{}\ldots{}\ldots{}\ldots{}\ldots{}\ldots{}\ldots{}\ldots{}\ldots{}\ldots{}\ldots{}\ldots{}\ldots{}\ldots{}\ldots{}\ldots{}\ldots{}\ldots{}\ldots{}\ldots{}\ldots{}\ldots{}\ldots{}\ldots{}\ldots{}\ldots{}\ldots{}\ldots{}\ldots{}

\ldots{}\ldots{}\ldots{}\ldots{}\ldots{}\ldots{}\ldots{}\ldots{}\ldots{}\ldots{}\ldots{}\ldots{}\ldots{}\ldots{}\ldots{}\ldots{}\ldots{}\ldots{}\ldots{}\ldots{}\ldots{}\ldots{}\ldots{}\ldots{}\ldots{}\ldots{}\ldots{}\ldots{}\ldots{}\ldots{}\ldots{}\ldots{}\ldots{}\ldots{}\ldots{}\ldots{}\ldots{}

\paragraph{c Les fichiers css}\label{c-les-fichiers-css}

Nous allons utiliser un fichiers \emph{css} pour l'habillage
(présentation) de notre page web.

Modifions le fichier \textbf{\emph{templates/accueil.html}}.

\begin{Shaded}
\begin{Highlighting}[]
\NormalTok{4 }\KeywordTok{<title>} \NormalTok{titre }\KeywordTok{</title>}
\NormalTok{6 }\KeywordTok{<link}\OtherTok{ href=}\StringTok{"\{\{ url_for('static',filename='mon_style.css')\}\}"}\OtherTok{ rel=}\StringTok{"stylesheet"}\OtherTok{ type=}\StringTok{"text/css"} \KeywordTok{/>}
\NormalTok{7 }\KeywordTok{</head>}
\end{Highlighting}
\end{Shaded}

\begin{Shaded}
\begin{Highlighting}[]
\NormalTok{15    }\KeywordTok{<footer>}\NormalTok{Nous somme le }\KeywordTok{<em}\OtherTok{ class=}\StringTok{"Rouge"}\KeywordTok{>}\NormalTok{\{\{ date \}\}}\KeywordTok{</em>} \NormalTok{et il est }\KeywordTok{<em}\OtherTok{ class=}\StringTok{"Rouge"}\KeywordTok{>}\NormalTok{\{\{ heure \}\}}\KeywordTok{</em>} \NormalTok{heures.}\KeywordTok{</footer>}
\end{Highlighting}
\end{Shaded}

Créer ensuite le fichier \textbf{\emph{static/mon\_style.css}} avec
ceci:

\begin{Shaded}
\begin{Highlighting}[]
 \NormalTok{1 h1 }\KeywordTok{\{}
 \ErrorTok{2}   \KeywordTok{color:} \DataTypeTok{#ff00ff}\KeywordTok{;}
 \ErrorTok{3}   \KeywordTok{text-align:} \DataTypeTok{center}\KeywordTok{;}
 \ErrorTok{4} \KeywordTok{\}}
 \NormalTok{5}
 \NormalTok{6 footer }\KeywordTok{\{}
 \ErrorTok{7}   \KeywordTok{position:} \DataTypeTok{absolute}\KeywordTok{;}
 \ErrorTok{8}   \KeywordTok{bottom:} \DataTypeTok{0}\KeywordTok{;}
 \ErrorTok{9}   \KeywordTok{width:} \DataTypeTok{100%}\KeywordTok{;}
\ErrorTok{10}   \KeywordTok{height:} \DataTypeTok{20px}\KeywordTok{;}
\ErrorTok{11}   \KeywordTok{background-color:} \DataTypeTok{#f5f5f5}\KeywordTok{;}
\ErrorTok{12}   \KeywordTok{text-align:} \DataTypeTok{center}\KeywordTok{;}
\ErrorTok{13} \KeywordTok{\}}
\NormalTok{14}
\NormalTok{15 em}\FloatTok{.Rouge} \KeywordTok{\{color:} \DataTypeTok{#ff0000}\KeywordTok{;\}}
\end{Highlighting}
\end{Shaded}

Je met le contenu de la balise

\textless{}h1\textgreater{}

au centre et en couleur.\\Je place le `footer' au centre en bas de la
page avec un fond de couleur.\\Je met en rouge la classe Rouge.

\paragraph{d Le javascript}\label{d-le-javascript}

Nous allons maintenant utiliser un fichier \textbf{\emph{jascript
(.js)}} pour afficher l'heure. Le code javascript sera effectué par le
navigateur. Ce n'est pas le serveur qui exécute le code donc pas besoin
de rafraichir la page internet pour que le code soit exécuté.\\Il n'est
pas question d'expliquer le code javascript mais vous pouvez regarder à
quoi cela ressemble dans le ficher
\textbf{\emph{static/mes\_scripts.js}}.

Comme pour le fichier \emph{.css} Il faut indiquer le fichier \emph{.js}
aux templates.

\begin{itemize}
\item
  Modifions le fichier \textbf{\emph{templates/accueil.html}}.

\begin{Shaded}
\begin{Highlighting}[]
 \NormalTok{6 }\KeywordTok{<link}\OtherTok{ href=}\StringTok{" url_for('static', filename='mon_style.css') "}\OtherTok{ rel=}\StringTok{"stylesheet"}\OtherTok{ type=}\StringTok{"text/css"} \KeywordTok{/>}
 \NormalTok{7 }\KeywordTok{<script}\OtherTok{ type=}\StringTok{text/javascript}\OtherTok{ src=}\StringTok{"\{\{url_for('static', filename='mes_scripts.js') \}\}"}\KeywordTok{></script>}
 \NormalTok{8 }\KeywordTok{</head>}
\end{Highlighting}
\end{Shaded}

\begin{Shaded}
\begin{Highlighting}[]
\NormalTok{25 }\KeywordTok{</div>}
\NormalTok{26 }\KeywordTok{<footer>}\NormalTok{Nous sommes le }\KeywordTok{<em}\OtherTok{ class=}\StringTok{Rouge}\OtherTok{ id=}\StringTok{"date"}\KeywordTok{></em>} \KeywordTok{<script}\OtherTok{ type=}\StringTok{"text/javascript"}\KeywordTok{>} \OtherTok{window}\NormalTok{.}\FunctionTok{onload} \NormalTok{= }\FunctionTok{datejs}\NormalTok{(}\StringTok{'date'}\NormalTok{);<}\OtherTok{/script> et il est <em class=Rouge id="heure"></em}\NormalTok{><script type=}\StringTok{"text/javascript"}\NormalTok{>}\OtherTok{window}\NormalTok{.}\FunctionTok{onload} \NormalTok{= }\FunctionTok{heurejs}\NormalTok{(}\StringTok{'heure'}\NormalTok{);<}\OtherTok{/script>.</footer}\NormalTok{>}
\DecValTok{27}\NormalTok{<}\OtherTok{/body>}
\end{Highlighting}
\end{Shaded}
\item
  Modifions le fichier \textbf{\emph{templates/accueil.html}}.

\begin{Shaded}
\begin{Highlighting}[]
 \NormalTok{6 }\KeywordTok{<script}\OtherTok{ type=}\StringTok{text/javascript}\OtherTok{ src=}\StringTok{"url_for('static', filename='mes_scripts.js') "}\KeywordTok{></script>}
 \NormalTok{7 }\KeywordTok{<link}\OtherTok{ rel=}\StringTok{"shortcut icon"}\OtherTok{ href=}\StringTok{"\{\{ url_for('static', filename='favicon.ico') \}\}"}\KeywordTok{>}
 \NormalTok{8 }\KeywordTok{</head>}
\end{Highlighting}
\end{Shaded}

\begin{Shaded}
\begin{Highlighting}[]
\NormalTok{11 }\KeywordTok{<p>}\NormalTok{Bonjour }\KeywordTok{<b>} \NormalTok{prenom}
\NormalTok{12 }\KeywordTok{<footer>}\NormalTok{Nous sommes le }\KeywordTok{<em}\OtherTok{ class=}\StringTok{Rouge}\OtherTok{ id=}\StringTok{"date"}\KeywordTok{></em>} \KeywordTok{<script}\OtherTok{ type=}\StringTok{"text/javascript"}\KeywordTok{>}\OtherTok{window}\NormalTok{.}\FunctionTok{onload} \NormalTok{= }\FunctionTok{datejs}\NormalTok{(}\StringTok{'date'}\NormalTok{);<}\OtherTok{/script> et il est <em class=Rouge id="heure"></em}\NormalTok{><script type=}\StringTok{"text/javascript"}\NormalTok{>}\OtherTok{window}\NormalTok{.}\FunctionTok{onload} \NormalTok{= }\FunctionTok{heurejs}\NormalTok{(}\StringTok{'heure'}\NormalTok{);<}\OtherTok{/script>.</footer}\NormalTok{>}
\DecValTok{13} \NormalTok{<}\OtherTok{/body>}
\end{Highlighting}
\end{Shaded}
\end{itemize}

\paragraph{d Les images}\label{d-les-images}

Ajoutons un favicon. Vous savez c'est la petite images à gauche de
l'adresse dans un navigateur. Le nom de l'image doit être
\textbf{\emph{favicon.ico}}.

J'ai déjà mis deux images dans le dossier \textbf{\emph{static/}}.\\Il
suffit donc de rajouter dans \textbf{\emph{templates/accueil.html}}

\begin{Shaded}
\begin{Highlighting}[]
 \NormalTok{7 }\KeywordTok{<link}\OtherTok{ rel=}\StringTok{"shortcut icon"}\OtherTok{ href=}\StringTok{"\{\{ url_for('static' , filename='favicon.ico')\}\}"}\KeywordTok{>}
 \NormalTok{8 }\KeywordTok{</head>}
 \NormalTok{9 }\KeywordTok{<body>}
\NormalTok{10 }\KeywordTok{<h1>}\NormalTok{\{\{ titre \}\} }\KeywordTok{<img}\OtherTok{ src=}\StringTok{"\{\{ url_for('static',filename ='fleur.png') \}\}"}\OtherTok{ alt=}\StringTok{"fleur"}\OtherTok{ title=}\StringTok{"fleur"}\OtherTok{ border=}\StringTok{"0"}\KeywordTok{></h1>}
\NormalTok{11 }\KeywordTok{<ul>}
\end{Highlighting}
\end{Shaded}

\subsubsection{5.1.2 Les formulaires}\label{les-formulaires}

Modifions la page \textbf{\emph{templates/accueil.html}} pour mettre un
\emph{formulaire}

\begin{Shaded}
\begin{Highlighting}[]
\NormalTok{15  }\KeywordTok{</ul>}
\NormalTok{16  }\KeywordTok{<div}\OtherTok{ id=}\StringTok{"content"}\KeywordTok{>}
\NormalTok{17     }\KeywordTok{<form}\OtherTok{ method=}\StringTok{"post"}\OtherTok{ action=}\StringTok{"\{\{ url_for('hello') \}\}"}\KeywordTok{>}
\NormalTok{18     }\KeywordTok{<label}\OtherTok{ for=}\StringTok{"nom"}\KeywordTok{>}\NormalTok{Entrez votre nom:}\KeywordTok{</label>}
\NormalTok{19     }\KeywordTok{<input}\OtherTok{ type=}\StringTok{"text"}\OtherTok{ name=}\StringTok{"nom"} \KeywordTok{/><br} \KeywordTok{/>}
\NormalTok{20     }\ErrorTok{<}\NormalTok{label for="prenom">Entrez votre prénom:}\KeywordTok{</label>}
\NormalTok{21     }\KeywordTok{<input}\OtherTok{ type=}\StringTok{"text"}\OtherTok{ name=}\StringTok{"prenom"} \KeywordTok{/><br} \KeywordTok{/>}
\NormalTok{22     }\KeywordTok{<input}\OtherTok{ type=}\StringTok{"submit"} \KeywordTok{/>}
\NormalTok{23     }\KeywordTok{</form>}
\NormalTok{24 }\KeywordTok{</div>}
\end{Highlighting}
\end{Shaded}

pour le fichier \textbf{\emph{exemple.py}} on crée une nouvelle route.

\begin{Shaded}
\begin{Highlighting}[]
\DecValTok{13}  \NormalTok{@app.route(}\StringTok{'/'}\NormalTok{)}
\DecValTok{14}  \KeywordTok{def} \NormalTok{accueil():}
\DecValTok{15}      \NormalTok{lignes=[}\StringTok{'ligne \{\}'}\NormalTok{.}\DataTypeTok{format}\NormalTok{(i) }\KeywordTok{for} \NormalTok{i in }\DataTypeTok{range}\NormalTok{(}\DecValTok{1}\NormalTok{,}\DecValTok{10}\NormalTok{)]}
\DecValTok{16}      \KeywordTok{return} \NormalTok{render_template(}\StringTok{"accueil.html"}\NormalTok{, titre=}\StringTok{"Bienvenue !"}\NormalTok{,lignes=lignes, date=date, heure=heure)}
\DecValTok{17}
\DecValTok{18} \NormalTok{@app.route(}\StringTok{'/hello/'}\NormalTok{, methods=[}\StringTok{'POST'}\NormalTok{])}
\DecValTok{19} \KeywordTok{def} \NormalTok{hello():}
\DecValTok{20}     \NormalTok{nom=request.form[ nom ]}
\DecValTok{21}     \NormalTok{prenom=request.form[ prenom ]}
\DecValTok{22}     \KeywordTok{return} \NormalTok{render_template( page2.html ,titre=}\StringTok{"Page 2"}\NormalTok{, nom=nom, prenom=prenom,date=date,heure=heure)}
\end{Highlighting}
\end{Shaded}

\subsubsection{5.1.3 Une partie commune à toutes les
pages}\label{une-partie-commune-uxe0-toutes-les-pages}

Nous avons plusieurs templates qui utilisent le même bas de page.\\Il
est donc possible d'utiliser un seul fichier template et de l'inclure
dans les autres. Créer un fichier \emph{templates/footer.html}

\begin{Shaded}
\begin{Highlighting}[]
\NormalTok{<footer>Nous sommes le <em }\KeywordTok{class}\NormalTok{=Rouge }\DataTypeTok{id}\NormalTok{=}\StringTok{"date"}\NormalTok{></em> <script }\DataTypeTok{type}\NormalTok{=}\StringTok{"text/javascript"}\NormalTok{>window.onload = datejs(}\StringTok{'date'}\NormalTok{);</script> et il est <em }\KeywordTok{class}\NormalTok{=Rouge }\DataTypeTok{id}\NormalTok{=}\StringTok{"heure"}\NormalTok{></em><script }\DataTypeTok{type}\NormalTok{=}\StringTok{"text/javascript"}\NormalTok{>window.onload = heurejs(}\StringTok{'heure'}\NormalTok{);</script>.</footer>}
\end{Highlighting}
\end{Shaded}

Puis modifier les templates \textbf{\emph{templates/accueil.html}}

\begin{verbatim}
25 </div>
26 
27 </body>
\end{verbatim}

et \textbf{\emph{templates/page2.html}}.

\begin{verbatim}
12 <p>Bonjour <b>{{ prenom }} {{ nom }}.</b></p>
13 
14 </body>
\end{verbatim}

\subsubsection{5.1.4 Mettre le site sur
internet}\label{mettre-le-site-sur-internet}

Habituellement on s'héberge soit même mais il est possible de mettre son
site chez un hébergeur. Il en existe plusieurs qui autorise les scripts
python.

Ouvrez le navigateur pablo à l'adresse
\url{https://isn-flask.herokuapp.com/}\\Oui vous avez reconnu notre tp.
J'ai simplement affiché la \emph{date} et l'\emph{heure} avec du code du
javascript.

\subsubsection{5.1.5 Prolongement}\label{prolongement}

\begin{itemize}
\itemsep1pt\parskip0pt\parsep0pt
\item
  Utilisation d'une base de donnée
\item
  Créer une image aux couleurs aléatoire (module pil) et l'afficher.
\end{itemize}

\subsection{5.2 Exercices}\label{exercices}

\begin{enumerate}
\def\labelenumi{\arabic{enumi}.}
\item
  \begin{itemize}
  \itemsep1pt\parskip0pt\parsep0pt
  \item
    Créer un dossier \textbf{\emph{static/file/}} et mettre dans ce
    dossier quelques fichiers textes (.txt) et images (.png ou .jpg).\\
  \end{itemize}
\end{enumerate}

\begin{itemize}
\itemsep1pt\parskip0pt\parsep0pt
\item
  Créer dans le fichier \textbf{\emph{exemple.py}} une fonction
  \textbf{\emph{listdir()}} qui retourne la liste les noms des fichiers
  contenus dans le dossier \textbf{\emph{static/file/}} \emph{(On
  utilisera la librairie \textbf{os} pour lister)}.
\item
  modifier le \textbf{\emph{templates/accueil.html}} pour faire afficher
  le nom des fichiers.
\end{itemize}

\begin{enumerate}
\def\labelenumi{\arabic{enumi}.}
\setcounter{enumi}{1}
\itemsep1pt\parskip0pt\parsep0pt
\item
  Modifier alors la fonction \textbf{\emph{listdir()}} pour n'afficher
  que le nom des images.
\item
  Afficher cette fois les images dans la page internet.
\end{enumerate}


    % Add a bibliography block to the postdoc
    
    
    
    \end{document}
