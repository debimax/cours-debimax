
% Default to the notebook output style

    


% Inherit from the specified cell style.


    
\documentclass{article}

    
    
    \usepackage{graphicx} % Used to insert images
    \usepackage{adjustbox} % Used to constrain images to a maximum size 
    \usepackage{color} % Allow colors to be defined
    \usepackage{enumerate} % Needed for markdown enumerations to work
    \usepackage{geometry} % Used to adjust the document margins
    \usepackage{amsmath} % Equations
    \usepackage{amssymb} % Equations
    \usepackage[mathletters]{ucs} % Extended unicode (utf-8) support
    \usepackage[utf8x]{inputenc} % Allow utf-8 characters in the tex document
    \usepackage{fancyvrb} % verbatim replacement that allows latex
    \usepackage{grffile} % extends the file name processing of package graphics 
                         % to support a larger range 
    % The hyperref package gives us a pdf with properly built
    % internal navigation ('pdf bookmarks' for the table of contents,
    % internal cross-reference links, web links for URLs, etc.)
    \usepackage{hyperref}
    \usepackage{longtable} % longtable support required by pandoc >1.10
    \usepackage{booktabs}  % table support for pandoc > 1.12.2
    

    
    
    \definecolor{orange}{cmyk}{0,0.4,0.8,0.2}
    \definecolor{darkorange}{rgb}{.71,0.21,0.01}
    \definecolor{darkgreen}{rgb}{.12,.54,.11}
    \definecolor{myteal}{rgb}{.26, .44, .56}
    \definecolor{gray}{gray}{0.45}
    \definecolor{lightgray}{gray}{.95}
    \definecolor{mediumgray}{gray}{.8}
    \definecolor{inputbackground}{rgb}{.95, .95, .85}
    \definecolor{outputbackground}{rgb}{.95, .95, .95}
    \definecolor{traceback}{rgb}{1, .95, .95}
    % ansi colors
    \definecolor{red}{rgb}{.6,0,0}
    \definecolor{green}{rgb}{0,.65,0}
    \definecolor{brown}{rgb}{0.6,0.6,0}
    \definecolor{blue}{rgb}{0,.145,.698}
    \definecolor{purple}{rgb}{.698,.145,.698}
    \definecolor{cyan}{rgb}{0,.698,.698}
    \definecolor{lightgray}{gray}{0.5}
    
    % bright ansi colors
    \definecolor{darkgray}{gray}{0.25}
    \definecolor{lightred}{rgb}{1.0,0.39,0.28}
    \definecolor{lightgreen}{rgb}{0.48,0.99,0.0}
    \definecolor{lightblue}{rgb}{0.53,0.81,0.92}
    \definecolor{lightpurple}{rgb}{0.87,0.63,0.87}
    \definecolor{lightcyan}{rgb}{0.5,1.0,0.83}
    
    % commands and environments needed by pandoc snippets
    % extracted from the output of `pandoc -s`
    \DefineVerbatimEnvironment{Highlighting}{Verbatim}{commandchars=\\\{\}}
    % Add ',fontsize=\small' for more characters per line
    \newenvironment{Shaded}{}{}
    \newcommand{\KeywordTok}[1]{\textcolor[rgb]{0.00,0.44,0.13}{\textbf{{#1}}}}
    \newcommand{\DataTypeTok}[1]{\textcolor[rgb]{0.56,0.13,0.00}{{#1}}}
    \newcommand{\DecValTok}[1]{\textcolor[rgb]{0.25,0.63,0.44}{{#1}}}
    \newcommand{\BaseNTok}[1]{\textcolor[rgb]{0.25,0.63,0.44}{{#1}}}
    \newcommand{\FloatTok}[1]{\textcolor[rgb]{0.25,0.63,0.44}{{#1}}}
    \newcommand{\CharTok}[1]{\textcolor[rgb]{0.25,0.44,0.63}{{#1}}}
    \newcommand{\StringTok}[1]{\textcolor[rgb]{0.25,0.44,0.63}{{#1}}}
    \newcommand{\CommentTok}[1]{\textcolor[rgb]{0.38,0.63,0.69}{\textit{{#1}}}}
    \newcommand{\OtherTok}[1]{\textcolor[rgb]{0.00,0.44,0.13}{{#1}}}
    \newcommand{\AlertTok}[1]{\textcolor[rgb]{1.00,0.00,0.00}{\textbf{{#1}}}}
    \newcommand{\FunctionTok}[1]{\textcolor[rgb]{0.02,0.16,0.49}{{#1}}}
    \newcommand{\RegionMarkerTok}[1]{{#1}}
    \newcommand{\ErrorTok}[1]{\textcolor[rgb]{1.00,0.00,0.00}{\textbf{{#1}}}}
    \newcommand{\NormalTok}[1]{{#1}}
    
    % Define a nice break command that doesn't care if a line doesn't already
    % exist.
    \def\br{\hspace*{\fill} \\* }
    % Math Jax compatability definitions
    \def\gt{>}
    \def\lt{<}
    % Document parameters
    \title{slicing}
    
    
    

    % Pygments definitions
    
\makeatletter
\def\PY@reset{\let\PY@it=\relax \let\PY@bf=\relax%
    \let\PY@ul=\relax \let\PY@tc=\relax%
    \let\PY@bc=\relax \let\PY@ff=\relax}
\def\PY@tok#1{\csname PY@tok@#1\endcsname}
\def\PY@toks#1+{\ifx\relax#1\empty\else%
    \PY@tok{#1}\expandafter\PY@toks\fi}
\def\PY@do#1{\PY@bc{\PY@tc{\PY@ul{%
    \PY@it{\PY@bf{\PY@ff{#1}}}}}}}
\def\PY#1#2{\PY@reset\PY@toks#1+\relax+\PY@do{#2}}

\expandafter\def\csname PY@tok@gd\endcsname{\def\PY@tc##1{\textcolor[rgb]{0.63,0.00,0.00}{##1}}}
\expandafter\def\csname PY@tok@gu\endcsname{\let\PY@bf=\textbf\def\PY@tc##1{\textcolor[rgb]{0.50,0.00,0.50}{##1}}}
\expandafter\def\csname PY@tok@gt\endcsname{\def\PY@tc##1{\textcolor[rgb]{0.00,0.27,0.87}{##1}}}
\expandafter\def\csname PY@tok@gs\endcsname{\let\PY@bf=\textbf}
\expandafter\def\csname PY@tok@gr\endcsname{\def\PY@tc##1{\textcolor[rgb]{1.00,0.00,0.00}{##1}}}
\expandafter\def\csname PY@tok@cm\endcsname{\let\PY@it=\textit\def\PY@tc##1{\textcolor[rgb]{0.25,0.50,0.50}{##1}}}
\expandafter\def\csname PY@tok@vg\endcsname{\def\PY@tc##1{\textcolor[rgb]{0.10,0.09,0.49}{##1}}}
\expandafter\def\csname PY@tok@m\endcsname{\def\PY@tc##1{\textcolor[rgb]{0.40,0.40,0.40}{##1}}}
\expandafter\def\csname PY@tok@mh\endcsname{\def\PY@tc##1{\textcolor[rgb]{0.40,0.40,0.40}{##1}}}
\expandafter\def\csname PY@tok@go\endcsname{\def\PY@tc##1{\textcolor[rgb]{0.53,0.53,0.53}{##1}}}
\expandafter\def\csname PY@tok@ge\endcsname{\let\PY@it=\textit}
\expandafter\def\csname PY@tok@vc\endcsname{\def\PY@tc##1{\textcolor[rgb]{0.10,0.09,0.49}{##1}}}
\expandafter\def\csname PY@tok@il\endcsname{\def\PY@tc##1{\textcolor[rgb]{0.40,0.40,0.40}{##1}}}
\expandafter\def\csname PY@tok@cs\endcsname{\let\PY@it=\textit\def\PY@tc##1{\textcolor[rgb]{0.25,0.50,0.50}{##1}}}
\expandafter\def\csname PY@tok@cp\endcsname{\def\PY@tc##1{\textcolor[rgb]{0.74,0.48,0.00}{##1}}}
\expandafter\def\csname PY@tok@gi\endcsname{\def\PY@tc##1{\textcolor[rgb]{0.00,0.63,0.00}{##1}}}
\expandafter\def\csname PY@tok@gh\endcsname{\let\PY@bf=\textbf\def\PY@tc##1{\textcolor[rgb]{0.00,0.00,0.50}{##1}}}
\expandafter\def\csname PY@tok@ni\endcsname{\let\PY@bf=\textbf\def\PY@tc##1{\textcolor[rgb]{0.60,0.60,0.60}{##1}}}
\expandafter\def\csname PY@tok@nl\endcsname{\def\PY@tc##1{\textcolor[rgb]{0.63,0.63,0.00}{##1}}}
\expandafter\def\csname PY@tok@nn\endcsname{\let\PY@bf=\textbf\def\PY@tc##1{\textcolor[rgb]{0.00,0.00,1.00}{##1}}}
\expandafter\def\csname PY@tok@no\endcsname{\def\PY@tc##1{\textcolor[rgb]{0.53,0.00,0.00}{##1}}}
\expandafter\def\csname PY@tok@na\endcsname{\def\PY@tc##1{\textcolor[rgb]{0.49,0.56,0.16}{##1}}}
\expandafter\def\csname PY@tok@nb\endcsname{\def\PY@tc##1{\textcolor[rgb]{0.00,0.50,0.00}{##1}}}
\expandafter\def\csname PY@tok@nc\endcsname{\let\PY@bf=\textbf\def\PY@tc##1{\textcolor[rgb]{0.00,0.00,1.00}{##1}}}
\expandafter\def\csname PY@tok@nd\endcsname{\def\PY@tc##1{\textcolor[rgb]{0.67,0.13,1.00}{##1}}}
\expandafter\def\csname PY@tok@ne\endcsname{\let\PY@bf=\textbf\def\PY@tc##1{\textcolor[rgb]{0.82,0.25,0.23}{##1}}}
\expandafter\def\csname PY@tok@nf\endcsname{\def\PY@tc##1{\textcolor[rgb]{0.00,0.00,1.00}{##1}}}
\expandafter\def\csname PY@tok@si\endcsname{\let\PY@bf=\textbf\def\PY@tc##1{\textcolor[rgb]{0.73,0.40,0.53}{##1}}}
\expandafter\def\csname PY@tok@s2\endcsname{\def\PY@tc##1{\textcolor[rgb]{0.73,0.13,0.13}{##1}}}
\expandafter\def\csname PY@tok@vi\endcsname{\def\PY@tc##1{\textcolor[rgb]{0.10,0.09,0.49}{##1}}}
\expandafter\def\csname PY@tok@nt\endcsname{\let\PY@bf=\textbf\def\PY@tc##1{\textcolor[rgb]{0.00,0.50,0.00}{##1}}}
\expandafter\def\csname PY@tok@nv\endcsname{\def\PY@tc##1{\textcolor[rgb]{0.10,0.09,0.49}{##1}}}
\expandafter\def\csname PY@tok@s1\endcsname{\def\PY@tc##1{\textcolor[rgb]{0.73,0.13,0.13}{##1}}}
\expandafter\def\csname PY@tok@kd\endcsname{\let\PY@bf=\textbf\def\PY@tc##1{\textcolor[rgb]{0.00,0.50,0.00}{##1}}}
\expandafter\def\csname PY@tok@sh\endcsname{\def\PY@tc##1{\textcolor[rgb]{0.73,0.13,0.13}{##1}}}
\expandafter\def\csname PY@tok@sc\endcsname{\def\PY@tc##1{\textcolor[rgb]{0.73,0.13,0.13}{##1}}}
\expandafter\def\csname PY@tok@sx\endcsname{\def\PY@tc##1{\textcolor[rgb]{0.00,0.50,0.00}{##1}}}
\expandafter\def\csname PY@tok@bp\endcsname{\def\PY@tc##1{\textcolor[rgb]{0.00,0.50,0.00}{##1}}}
\expandafter\def\csname PY@tok@c1\endcsname{\let\PY@it=\textit\def\PY@tc##1{\textcolor[rgb]{0.25,0.50,0.50}{##1}}}
\expandafter\def\csname PY@tok@kc\endcsname{\let\PY@bf=\textbf\def\PY@tc##1{\textcolor[rgb]{0.00,0.50,0.00}{##1}}}
\expandafter\def\csname PY@tok@c\endcsname{\let\PY@it=\textit\def\PY@tc##1{\textcolor[rgb]{0.25,0.50,0.50}{##1}}}
\expandafter\def\csname PY@tok@mf\endcsname{\def\PY@tc##1{\textcolor[rgb]{0.40,0.40,0.40}{##1}}}
\expandafter\def\csname PY@tok@err\endcsname{\def\PY@bc##1{\setlength{\fboxsep}{0pt}\fcolorbox[rgb]{1.00,0.00,0.00}{1,1,1}{\strut ##1}}}
\expandafter\def\csname PY@tok@mb\endcsname{\def\PY@tc##1{\textcolor[rgb]{0.40,0.40,0.40}{##1}}}
\expandafter\def\csname PY@tok@ss\endcsname{\def\PY@tc##1{\textcolor[rgb]{0.10,0.09,0.49}{##1}}}
\expandafter\def\csname PY@tok@sr\endcsname{\def\PY@tc##1{\textcolor[rgb]{0.73,0.40,0.53}{##1}}}
\expandafter\def\csname PY@tok@mo\endcsname{\def\PY@tc##1{\textcolor[rgb]{0.40,0.40,0.40}{##1}}}
\expandafter\def\csname PY@tok@kn\endcsname{\let\PY@bf=\textbf\def\PY@tc##1{\textcolor[rgb]{0.00,0.50,0.00}{##1}}}
\expandafter\def\csname PY@tok@mi\endcsname{\def\PY@tc##1{\textcolor[rgb]{0.40,0.40,0.40}{##1}}}
\expandafter\def\csname PY@tok@gp\endcsname{\let\PY@bf=\textbf\def\PY@tc##1{\textcolor[rgb]{0.00,0.00,0.50}{##1}}}
\expandafter\def\csname PY@tok@o\endcsname{\def\PY@tc##1{\textcolor[rgb]{0.40,0.40,0.40}{##1}}}
\expandafter\def\csname PY@tok@kr\endcsname{\let\PY@bf=\textbf\def\PY@tc##1{\textcolor[rgb]{0.00,0.50,0.00}{##1}}}
\expandafter\def\csname PY@tok@s\endcsname{\def\PY@tc##1{\textcolor[rgb]{0.73,0.13,0.13}{##1}}}
\expandafter\def\csname PY@tok@kp\endcsname{\def\PY@tc##1{\textcolor[rgb]{0.00,0.50,0.00}{##1}}}
\expandafter\def\csname PY@tok@w\endcsname{\def\PY@tc##1{\textcolor[rgb]{0.73,0.73,0.73}{##1}}}
\expandafter\def\csname PY@tok@kt\endcsname{\def\PY@tc##1{\textcolor[rgb]{0.69,0.00,0.25}{##1}}}
\expandafter\def\csname PY@tok@ow\endcsname{\let\PY@bf=\textbf\def\PY@tc##1{\textcolor[rgb]{0.67,0.13,1.00}{##1}}}
\expandafter\def\csname PY@tok@sb\endcsname{\def\PY@tc##1{\textcolor[rgb]{0.73,0.13,0.13}{##1}}}
\expandafter\def\csname PY@tok@k\endcsname{\let\PY@bf=\textbf\def\PY@tc##1{\textcolor[rgb]{0.00,0.50,0.00}{##1}}}
\expandafter\def\csname PY@tok@se\endcsname{\let\PY@bf=\textbf\def\PY@tc##1{\textcolor[rgb]{0.73,0.40,0.13}{##1}}}
\expandafter\def\csname PY@tok@sd\endcsname{\let\PY@it=\textit\def\PY@tc##1{\textcolor[rgb]{0.73,0.13,0.13}{##1}}}

\def\PYZbs{\char`\\}
\def\PYZus{\char`\_}
\def\PYZob{\char`\{}
\def\PYZcb{\char`\}}
\def\PYZca{\char`\^}
\def\PYZam{\char`\&}
\def\PYZlt{\char`\<}
\def\PYZgt{\char`\>}
\def\PYZsh{\char`\#}
\def\PYZpc{\char`\%}
\def\PYZdl{\char`\$}
\def\PYZhy{\char`\-}
\def\PYZsq{\char`\'}
\def\PYZdq{\char`\"}
\def\PYZti{\char`\~}
% for compatibility with earlier versions
\def\PYZat{@}
\def\PYZlb{[}
\def\PYZrb{]}
\makeatother


    % Exact colors from NB
    \definecolor{incolor}{rgb}{0.0, 0.0, 0.5}
    \definecolor{outcolor}{rgb}{0.545, 0.0, 0.0}



    
    % Prevent overflowing lines due to hard-to-break entities
    \sloppy 
    % Setup hyperref package
    \hypersetup{
      breaklinks=true,  % so long urls are correctly broken across lines
      colorlinks=true,
      urlcolor=blue,
      linkcolor=darkorange,
      citecolor=darkgreen,
      }
    % Slightly bigger margins than the latex defaults
    
    \geometry{verbose,tmargin=1in,bmargin=1in,lmargin=1in,rmargin=1in}
    \author{Meilland}
    

    \begin{document}
    
    
    \maketitle
    
    

    

    \section{2 Le slicing et les structures de liste}


    \subsection{2.1 Complément pour la chaîne de
caractères.}\label{compluxe9ment-pour-la-chauxeene-de-caractuxe8res.}

Une chaîne de caractères est une liste particulière ne contenant que des
caractères.\\Comme pour une liste, on peut accéder à ses éléments (les
caractères) en spécifiant un indice :

    \begin{Verbatim}[commandchars=\\\{\}]
{\color{incolor}In [{\color{incolor}}]:} \PY{n}{chaine}\PY{o}{=} \PY{l+s}{\PYZdq{}}\PY{l+s}{Lycée Pablo Neruda}\PY{l+s}{\PYZdq{}}
       \PY{n}{chaine}\PY{p}{[}\PY{l+m+mi}{0}\PY{p}{]}
\end{Verbatim}

    Par contre, il est impossible de modifier une chaîne de caractères ! On
dit alors qu'il s'agit d'une liste non modifiable :

    \begin{Verbatim}[commandchars=\\\{\}]
{\color{incolor}In [{\color{incolor}}]:} \PY{n}{chaine}\PY{p}{[}\PY{l+m+mi}{1}\PY{p}{]}\PY{o}{=} \PY{l+s}{\PYZdq{}}\PY{l+s}{a}\PY{l+s}{\PYZdq{}}
\end{Verbatim}

    Si vous décidiez de lui ajouter des caractères en fin de chaîne à l'aide
d'une concaténation du type suivant :

    \begin{Verbatim}[commandchars=\\\{\}]
{\color{incolor}In [{\color{incolor}}]:} \PY{n}{chaine}\PY{o}{=}\PY{n}{chaine}\PY{o}{+}\PY{l+s}{\PYZdq{}}\PY{l+s}{!}\PY{l+s}{\PYZdq{}}
       \PY{n}{chaine}
\end{Verbatim}

    Il existe de nombreuse méthode pour les chaînes de caratères. Ouvrez
ipython3 puis tapez

    \begin{Verbatim}[commandchars=\\\{\}]
{\color{incolor}In [{\color{incolor}}]:} \PY{n}{x}\PY{o}{=}\PY{l+s}{\PYZsq{}}\PY{l+s}{Pablo Neruda}\PY{l+s}{\PYZsq{}}
\end{Verbatim}

    tapez alors juste \textbf{\emph{x. (n'oubliez pas le point) puis appuyez
sur la tabulation}}, vous devez voir toutes les méthodes.

\begin{longtable}[c]{@{}cccc@{}}
\toprule\addlinespace
a & b & c & d
\\\addlinespace
\midrule\endhead
x.capitalize & x.isalnum & x.join & x.rsplit
\\\addlinespace
x.casefold & x.isalpha & x.ljust & x.rstrip
\\\addlinespace
x.center & x.isdecimal & x.lower & x.split
\\\addlinespace
x.count & x.isdigit & x.lstrip & x.splitlines
\\\addlinespace
x.encode & x.isidentifier & x.maketrans & x.startswith
\\\addlinespace
x.endswith & x.islower & x.partition & x.strip
\\\addlinespace
x.expandtabs & x.isnumeric & x.replace & x.swapcase
\\\addlinespace
x.find & x.isprintable & x.rfind & x.title
\\\addlinespace
x.format & x.isspace & x.rindex & x.translate
\\\addlinespace
x.format\_map & x.istitle & x.rjust & x.upper
\\\addlinespace
x.index & x.isupper & x.rpartition & x.zfill
\\\addlinespace
\bottomrule
\end{longtable}

    Inutile de toutes les connaître. Je vous propose de voir ici les
fonctions les plus utiles.

\subsubsection{a. Couper et joindre}\label{a.-couper-et-joindre}

La fonction \textbf{\emph{split()}} permet de \textbf{\emph{découper}}
la chaîne de caractères qui lui est passée en paramètre suivant un ou
des caractère(s) de séparation et renvoie une liste des chaînes
découpées. Les caractères de séparation lui sont également passés en
paramètre et, si ce n'est pas le cas, ce sera le caractère espace qui
sera utilisé :

    \begin{Verbatim}[commandchars=\\\{\}]
{\color{incolor}In [{\color{incolor}}]:} \PY{n+nb}{str}\PY{o}{=}\PY{l+s}{\PYZdq{}}\PY{l+s}{Pablo Neruda Saint martin d}\PY{l+s}{\PYZsq{}}\PY{l+s}{hère}\PY{l+s}{\PYZdq{}}
       \PY{n+nb}{str}\PY{o}{.}\PY{n}{split}\PY{p}{(}\PY{p}{)}
\end{Verbatim}

    \begin{Verbatim}[commandchars=\\\{\}]
{\color{incolor}In [{\color{incolor}}]:} \PY{n+nb}{str}\PY{o}{.}\PY{n}{split}\PY{p}{(}\PY{l+s}{\PYZsq{}}\PY{l+s}{ }\PY{l+s}{\PYZsq{}}\PY{p}{,}\PY{l+m+mi}{2}\PY{p}{)}
\end{Verbatim}

    L'opération inverse s'appelle \textbf{\emph{join()}}. Elle consiste a
rendre une liste de chaînes de caractères pour former une chaîne en
concaténant tous les éléments et en les assemblant à l'aide d'un
caractère.

Cette méthode prend en paramètre une liste de caractères et s'applique à
une chaîne de caractères désignant le ou les caractère(s) de liaison :

    \begin{Verbatim}[commandchars=\\\{\}]
{\color{incolor}In [{\color{incolor}}]:} \PY{n}{l}\PY{o}{=}\PY{n+nb}{str}\PY{o}{.}\PY{n}{split}\PY{p}{(}\PY{p}{)}
       \PY{n}{l}
\end{Verbatim}

    \begin{Verbatim}[commandchars=\\\{\}]
{\color{incolor}In [{\color{incolor}}]:} \PY{l+s}{\PYZsq{}}\PY{l+s}{ \PYZhy{} }\PY{l+s}{\PYZsq{}}\PY{o}{.}\PY{n}{join}\PY{p}{(}\PY{n}{l}\PY{p}{)}
\end{Verbatim}

    On peut faire aussi la transformation directement

    \begin{Verbatim}[commandchars=\\\{\}]
{\color{incolor}In [{\color{incolor}}]:} \PY{l+s}{\PYZsq{}}\PY{l+s}{ \PYZhy{} }\PY{l+s}{\PYZsq{}}\PY{o}{.}\PY{n}{join}\PY{p}{(}\PY{l+s}{\PYZdq{}}\PY{l+s}{Pablo Neruda Saint martin d}\PY{l+s}{\PYZsq{}}\PY{l+s}{hère}\PY{l+s}{\PYZdq{}}\PY{o}{.}\PY{n}{split}\PY{p}{(}\PY{p}{)}\PY{p}{)}
\end{Verbatim}

    \subsubsection{b. Majuscule et
minuscule}\label{b.-majuscule-et-minuscule}

Deux autres méthodes standards peuvent être utiles:

\textbf{\emph{lower()}} et \textbf{\emph{upper()}} permettant
respectivement de convertir les caractères d'une chaîne en minuscules ou
en majuscules.\\Attention, bien que parlant de \textless{}\textless{}
conversion \textgreater{}\textgreater{}, ces méthodes ne modifient pas
la chaîne de départ mais renvoient une nouvelle chaîne :

    \begin{Verbatim}[commandchars=\\\{\}]
{\color{incolor}In [{\color{incolor}}]:} \PY{l+s}{\PYZsq{}}\PY{l+s}{Pablo Neruda}\PY{l+s}{\PYZsq{}}\PY{o}{.}\PY{n}{lower}\PY{p}{(}\PY{p}{)}
\end{Verbatim}

    \begin{Verbatim}[commandchars=\\\{\}]
{\color{incolor}In [{\color{incolor}}]:} \PY{l+s}{\PYZsq{}}\PY{l+s}{Pablo Neruda}\PY{l+s}{\PYZsq{}}\PY{o}{.}\PY{n}{upper}\PY{p}{(}\PY{p}{)}
\end{Verbatim}

    La fonction \textbf{\emph{capitalize()}} permet de ne mettre en
majuscule que la première lettre d'une chaîne :

    \begin{Verbatim}[commandchars=\\\{\}]
{\color{incolor}In [{\color{incolor}}]:} \PY{l+s}{\PYZsq{}}\PY{l+s}{pablo neruda}\PY{l+s}{\PYZsq{}}\PY{o}{.}\PY{n}{capitalize}\PY{p}{(}\PY{p}{)}
\end{Verbatim}

    Si maintenant nous souhaitons que la première lettre de chaque mot soit
une majuscule, que pouvons-nous faire?\\En utilisant les quelques
fonctions vues précédemment, cela est tout à fait réalisable :

    \begin{Verbatim}[commandchars=\\\{\}]
{\color{incolor}In [{\color{incolor}}]:} \PY{n}{p}\PY{o}{=}\PY{l+s}{\PYZsq{}}\PY{l+s}{une petite phrase de test}\PY{l+s}{\PYZsq{}}
       \PY{n}{l}\PY{o}{=}\PY{n}{p}\PY{o}{.}\PY{n}{split}\PY{p}{(}\PY{p}{)}
       \PY{n}{l}
\end{Verbatim}

    \begin{Verbatim}[commandchars=\\\{\}]
{\color{incolor}In [{\color{incolor}}]:} \PY{n}{i}\PY{o}{=}\PY{l+m+mi}{0}
       \PY{k}{for} \PY{n}{s} \PY{o+ow}{in} \PY{n}{l}\PY{p}{:}
           \PY{n}{l}\PY{p}{[}\PY{n}{i}\PY{p}{]} \PY{o}{=} \PY{n}{s}\PY{o}{.}\PY{n}{capitalize}\PY{p}{(}\PY{p}{)}
           \PY{n}{i} \PY{o}{+}\PY{o}{=} \PY{l+m+mi}{1}
       \PY{l+s}{\PYZsq{}}\PY{l+s}{ }\PY{l+s}{\PYZsq{}}\PY{o}{.}\PY{n}{join}\PY{p}{(}\PY{n}{l}\PY{p}{)}
\end{Verbatim}

    On peut même le faire beaucoup plus \textless{}\textless{} simplement
\textgreater{}\textgreater{}, en utilisant la compréhension de listes

    \begin{Verbatim}[commandchars=\\\{\}]
{\color{incolor}In [{\color{incolor}}]:} \PY{l+s}{\PYZsq{}}\PY{l+s}{ }\PY{l+s}{\PYZsq{}}\PY{o}{.}\PY{n}{join}\PY{p}{(}\PY{p}{[}\PY{n}{s}\PY{o}{.}\PY{n}{capitalize}\PY{p}{(}\PY{p}{)} \PY{k}{for} \PY{n}{s} \PY{o+ow}{in} \PY{n}{p}\PY{o}{.}\PY{n}{split}\PY{p}{(}\PY{p}{)}\PY{p}{]}\PY{p}{)}
\end{Verbatim}

    On reverra les compréhensions de listes un peu plus tard

\subsubsection{c. rechercher un caractère, une position etc\ldots{} dans
une
chaîne.}\label{c.-rechercher-un-caractuxe8re-une-position-etc-dans-une-chauxeene.}

    \begin{itemize}
\itemsep1pt\parskip0pt\parsep0pt
\item
  La fonction \textbf{\emph{len()}} permet de compter le nombre de
  caractères.
\end{itemize}

    len(`Pablo Neruda')

    \begin{itemize}
\itemsep1pt\parskip0pt\parsep0pt
\item
  La méthode \textbf{\emph{count()}} permet de compter le nombre
  d'occurrences d'une sous chaîne dans une chaîne de caractères.\\Le
  premier paramètre est la chaîne dans laquelle effectuer la recherche
  et le second paramètre est la sous chaîne :
\end{itemize}

    \begin{Verbatim}[commandchars=\\\{\}]
{\color{incolor}In [{\color{incolor}}]:} \PY{l+s}{\PYZsq{}}\PY{l+s}{Pablo Neruda}\PY{l+s}{\PYZsq{}}\PY{o}{.}\PY{n}{count}\PY{p}{(}\PY{l+s}{\PYZsq{}}\PY{l+s}{a}\PY{l+s}{\PYZsq{}}\PY{p}{)}
\end{Verbatim}

    \begin{itemize}
\itemsep1pt\parskip0pt\parsep0pt
\item
  La fonction \textbf{\emph{find()}} permet de trouver l'indice de la
  première occurrence d'une sous chaîne. Les paramètres sont les mêmes
  que pour la fonction \textbf{\emph{count()}}\\En cas d'échec,
  \textbf{\emph{find()}} renvoie la valeur -1 ( 0 correspond à l'indice
  du premier caractère): On utilisera \textbf{\emph{rfind()}} pour la
  dernière ocurrence.
\end{itemize}

    \begin{Verbatim}[commandchars=\\\{\}]
{\color{incolor}In [{\color{incolor}}]:} \PY{l+s}{\PYZsq{}}\PY{l+s}{Pablo Neruda}\PY{l+s}{\PYZsq{}}\PY{o}{.}\PY{n}{find}\PY{p}{(}\PY{l+s}{\PYZsq{}}\PY{l+s}{a}\PY{l+s}{\PYZsq{}}\PY{p}{)}
\end{Verbatim}

    \begin{Verbatim}[commandchars=\\\{\}]
{\color{incolor}In [{\color{incolor}}]:} \PY{l+s}{\PYZsq{}}\PY{l+s}{Pablo Neruda}\PY{l+s}{\PYZsq{}}\PY{o}{.}\PY{n}{find}\PY{p}{(}\PY{l+s}{\PYZsq{}}\PY{l+s}{aa}\PY{l+s}{\PYZsq{}}\PY{p}{)}
\end{Verbatim}

    \begin{Verbatim}[commandchars=\\\{\}]
{\color{incolor}In [{\color{incolor}}]:} \PY{l+s}{\PYZsq{}}\PY{l+s}{Pablo Neruda}\PY{l+s}{\PYZsq{}}\PY{o}{.}\PY{n}{rfind}\PY{p}{(}\PY{l+s}{\PYZsq{}}\PY{l+s}{a}\PY{l+s}{\PYZsq{}}\PY{p}{)}
\end{Verbatim}

    \begin{itemize}
\itemsep1pt\parskip0pt\parsep0pt
\item
  \textbf{\emph{index()}} est identique à \textbf{\emph{find()}} mais
  retourne une erreur en cas d'échec
\end{itemize}

    \begin{Verbatim}[commandchars=\\\{\}]
{\color{incolor}In [{\color{incolor}}]:} \PY{l+s}{\PYZsq{}}\PY{l+s}{Pablo Neruda}\PY{l+s}{\PYZsq{}}\PY{o}{.}\PY{n}{index}\PY{p}{(}\PY{l+s}{\PYZsq{}}\PY{l+s}{a}\PY{l+s}{\PYZsq{}}\PY{p}{)}
\end{Verbatim}

    \begin{Verbatim}[commandchars=\\\{\}]
{\color{incolor}In [{\color{incolor}}]:} \PY{l+s}{\PYZsq{}}\PY{l+s}{Pablo Neruda}\PY{l+s}{\PYZsq{}}\PY{o}{.}\PY{n}{index}\PY{p}{(}\PY{l+s}{\PYZsq{}}\PY{l+s}{aa}\PY{l+s}{\PYZsq{}}\PY{p}{)}
\end{Verbatim}

    Avec les méthode qui renvoie une \textbf{\emph{ValueError}} on utlilise
généralement la ``condition'' try except.\\

try: Quelquechose except: S'il y a une erreur alors autre chose 

    \begin{Verbatim}[commandchars=\\\{\}]
{\color{incolor}In [{\color{incolor}}]:} \PY{k}{try}\PY{p}{:} \PY{k}{print}\PY{p}{(}\PY{l+s}{\PYZsq{}}\PY{l+s}{Pablo Neruda}\PY{l+s}{\PYZsq{}}\PY{o}{.}\PY{n}{index}\PY{p}{(}\PY{l+s}{\PYZsq{}}\PY{l+s}{aa}\PY{l+s}{\PYZsq{}}\PY{p}{)}\PY{p}{)}
       \PY{k}{except}\PY{p}{:}  \PY{k}{print}\PY{p}{(}\PY{l+s}{\PYZdq{}}\PY{l+s}{Il n}\PY{l+s}{\PYZsq{}}\PY{l+s}{y a pas de aa}\PY{l+s}{\PYZdq{}}\PY{p}{)}
\end{Verbatim}

    \begin{itemize}
\itemsep1pt\parskip0pt\parsep0pt
\item
  La fonction \textbf{\emph{replace()}} permet, comme son nom l'indique,
  de remplacer une sous chaîne par une autre à l'intérieur d'une chaîne
  de caractères.\\Les paramètres sont, dans l'ordre : la chaîne de
  caractères à modifier, la sous chaîne à remplacer, la sous chaîne de
  remplacement,et, éventuellement, le nombre maximum d'occurrences à
  remplacer (si non spécifié, toutes les occurrences seront remplacées).
\end{itemize}

    \begin{Verbatim}[commandchars=\\\{\}]
{\color{incolor}In [{\color{incolor}}]:} \PY{l+s}{\PYZsq{}}\PY{l+s}{Pablo Neruda}\PY{l+s}{\PYZsq{}}\PY{o}{.}\PY{n}{replace}\PY{p}{(}\PY{l+s}{\PYZsq{}}\PY{l+s}{a}\PY{l+s}{\PYZsq{}}\PY{p}{,}\PY{l+s}{\PYZsq{}}\PY{l+s}{1}\PY{l+s}{\PYZsq{}}\PY{p}{)}
\end{Verbatim}

    Pour des recerches un peu plus complète on utilisera le module
\textbf{\emph{re}} (regex)

\subsubsection{d. Convertion}\label{d.-convertion}

\textbf{\emph{str.encode(encoding=``utf-8'', errors=``strict'')}}
Retourne une version codée de la chaîne comme un objet
\textbf{\emph{bytes}}. L'encodage par défaut est ``utf-8''.

    \begin{Verbatim}[commandchars=\\\{\}]
{\color{incolor}In [{\color{incolor}}]:} \PY{l+s}{\PYZsq{}}\PY{l+s}{Pablo Neruda}\PY{l+s}{\PYZsq{}}\PY{o}{.}\PY{n}{encode}\PY{p}{(}\PY{l+s}{\PYZsq{}}\PY{l+s}{utf8}\PY{l+s}{\PYZsq{}}\PY{p}{)}
\end{Verbatim}

    *\#\# À retenir:

Il faut savoir utiliser \textbf{\emph{split()}} , \textbf{\emph{join()}}
, \textbf{\emph{len()}} , \textbf{\emph{count()}} ,
\textbf{\emph{find()}} , \textbf{\emph{index()}} ,
\textbf{\emph{rfind()}}, \textbf{\emph{replace()}}.

\subsubsection{b. Exercice string}\label{b.-exercice-string}

\textbf{Exercice 1}:\\Écrire un programme qui dans une phrase compte:

\begin{itemize}
\item
  \begin{enumerate}
  \def\labelenumi{\alph{enumi})}
  \itemsep1pt\parskip0pt\parsep0pt
  \item
    le nombre de voyelle.\\ Entrée: `Un matin'\\ Sortie: 3
  \end{enumerate}
\item
  \begin{enumerate}
  \def\labelenumi{\alph{enumi})}
  \setcounter{enumi}{1}
  \itemsep1pt\parskip0pt\parsep0pt
  \item
    Le nombre de caratères (on utilisera la méthode \textbf{isalnum()}
    qui retourne True si le caracère est une lettre ou un chiffres)\\
    Entrée: ``l'avis n°1''\\ Sortie: 7
  \end{enumerate}
\end{itemize}

    \emph{\textbf{Exercice 2}:\\Écrire un programme qui teste si la chaîne
}str* est une adresse mail. On va simplifier sur le fait que \emph{str}
doit se terminer par @****.**\\- Un seul caractère @ - Un seul caractère
.(point) après @.

    \textbf{\emph{Exercice 3:}}\\Écrire un programme qui détermine si une
chaîne de caractères est un palindrome (un mot qui pour lequel la
signification est la même dans les deux sens de lecture, par exemple
\emph{ressasser}.

    \textbf{\emph{Exercice 4:}}\\Demandez la saisie d'un message puis
afficher ce dernier de façon indentée.

Exemple: si str=`toto'\\t\\to\\tot\\toto

    \subsection{2.2 Les spécificités des
tuples}\label{les-spuxe9cificituxe9s-des-tuples}

Les tuples sont également des listes non modifiables (les chaînes de
caractères sont donc en fait des tuples particuliers) :

    \begin{Verbatim}[commandchars=\\\{\}]
{\color{incolor}In [{\color{incolor}}]:} \PY{n}{t}\PY{o}{=}\PY{p}{(}\PY{l+m+mi}{1}\PY{p}{,}\PY{l+m+mi}{2}\PY{p}{,}\PY{l+m+mi}{3}\PY{p}{)}
       \PY{n}{t}\PY{p}{[}\PY{l+m+mi}{0}\PY{p}{]}
\end{Verbatim}

    \begin{Verbatim}[commandchars=\\\{\}]
{\color{incolor}In [{\color{incolor}}]:} \PY{n}{t}\PY{p}{[}\PY{l+m+mi}{0}\PY{p}{]}\PY{o}{=}\PY{l+m+mi}{0}
\end{Verbatim}

    Attention de ne pas utiliser le mot-clé tuple comme nom de variable: ce
dernier permet de créer un tuple de manière explicite. Il vous faudra
utiliser en paramètre. . . une liste. Il permet donc d'effectuer une
conversion :

    \begin{Verbatim}[commandchars=\\\{\}]
{\color{incolor}In [{\color{incolor}}]:} \PY{n}{l}\PY{o}{=}\PY{p}{[}\PY{l+m+mi}{1}\PY{p}{,}\PY{l+m+mi}{2}\PY{p}{,}\PY{l+m+mi}{3}\PY{p}{]}
       \PY{n}{t}\PY{o}{=}\PY{n+nb}{tuple}\PY{p}{(}\PY{n}{l}\PY{p}{)}
       \PY{n}{t}
\end{Verbatim}

    \begin{Verbatim}[commandchars=\\\{\}]
{\color{incolor}In [{\color{incolor}}]:} \PY{n}{l}
\end{Verbatim}

    \begin{Verbatim}[commandchars=\\\{\}]
{\color{incolor}In [{\color{incolor}}]:} \PY{n}{t}\PY{o}{=}\PY{n+nb}{tuple}\PY{p}{(}\PY{l+s}{\PYZsq{}}\PY{l+s}{1234}\PY{l+s}{\PYZsq{}}\PY{p}{)}
       \PY{n}{t}
\end{Verbatim}

    Tout comme avec les chaînes, il sera possible de concaténer des tuples
et avoir la sensation d'avoir modifié un tuple alors que nous aurons
simplement réaffecté une variable :

    \begin{Verbatim}[commandchars=\\\{\}]
{\color{incolor}In [{\color{incolor}}]:} \PY{n}{t}\PY{o}{=}\PY{p}{(}\PY{l+m+mi}{1}\PY{p}{,}\PY{l+m+mi}{2}\PY{p}{,}\PY{l+m+mi}{3}\PY{p}{)}
       \PY{n}{t}\PY{o}{=}\PY{n}{t}\PY{o}{+}\PY{p}{(}\PY{l+m+mi}{4}\PY{p}{,}\PY{l+m+mi}{5}\PY{p}{,}\PY{l+m+mi}{6}\PY{p}{)}
       \PY{n}{t}
\end{Verbatim}

    Pour rappel, un tuple ne contenant qu'un seul élément est noté entre
parenthèses, mais avec une virgule précédant la dernière
parenthèse:\\Quel peut être l'intérêt d'utiliser ce type plutôt qu'une
liste ?

Tout d'abord, les données ne peuvent pas être modifiées par erreur mais
surtout, leur implémentation en machine fait qu'elles occupent moins
d'espace mémoire et que leur traitement par l'interpréteur est plus
rapide que pour des valeurs identiques mais stockées sous forme de
listes.

De plus, de par leur aspect non modifiable, les valeurs contenues dans
un tuple peuvent être utilisées en tant que clé pour accéder à une
valeur dans un dictionnaire (alors que c'est beaucoup plus dangereux
avec les valeurs contenues dans une liste) :

    \begin{Verbatim}[commandchars=\\\{\}]
{\color{incolor}In [{\color{incolor}}]:} \PY{n}{l}\PY{o}{=}\PY{p}{[}\PY{l+s}{\PYZsq{}}\PY{l+s}{cle1}\PY{l+s}{\PYZsq{}}\PY{p}{,}\PY{l+s}{\PYZsq{}}\PY{l+s}{cle2}\PY{l+s}{\PYZsq{}}\PY{p}{,}\PY{l+s}{\PYZsq{}}\PY{l+s}{cle3}\PY{l+s}{\PYZsq{}}\PY{p}{]}
       \PY{n}{t}\PY{o}{=}\PY{p}{(}\PY{l+s}{\PYZsq{}}\PY{l+s}{cle1}\PY{l+s}{\PYZsq{}}\PY{p}{,}\PY{l+s}{\PYZsq{}}\PY{l+s}{cle2}\PY{l+s}{\PYZsq{}}\PY{p}{,}\PY{l+s}{\PYZsq{}}\PY{l+s}{cle3}\PY{l+s}{\PYZsq{}}\PY{p}{)}
       \PY{n}{d}\PY{o}{=}\PY{p}{\PYZob{}}\PY{l+s}{\PYZsq{}}\PY{l+s}{cle1}\PY{l+s}{\PYZsq{}}\PY{p}{:}\PY{l+m+mi}{1}\PY{p}{,}\PY{l+s}{\PYZsq{}}\PY{l+s}{cle2}\PY{l+s}{\PYZsq{}}\PY{p}{:}\PY{l+m+mi}{2}\PY{p}{,}\PY{l+s}{\PYZsq{}}\PY{l+s}{cle3}\PY{l+s}{\PYZsq{}}\PY{p}{:}\PY{l+m+mi}{3}\PY{p}{\PYZcb{}}
       \PY{n}{d}\PY{p}{[}\PY{n}{l}\PY{p}{[}\PY{l+m+mi}{0}\PY{p}{]}\PY{p}{]}
\end{Verbatim}

    \begin{Verbatim}[commandchars=\\\{\}]
{\color{incolor}In [{\color{incolor}}]:} \PY{n}{l}\PY{p}{[}\PY{l+m+mi}{0}\PY{p}{]}\PY{o}{=} \PY{l+s}{\PYZsq{}}\PY{l+s}{cle\PYZus{}modifiee}\PY{l+s}{\PYZsq{}}
       \PY{n}{d}\PY{p}{[}\PY{n}{l}\PY{p}{[}\PY{l+m+mi}{0}\PY{p}{]}\PY{p}{]}
\end{Verbatim}

    Les tuples dispose de deux méthodes \textbf{\emph{count()}} et
\textbf{\emph{t.index}}. Leur utilisation est semblable aux chaînes de
caractère.

\subsubsection{À retenir}\label{uxe0-retenir}

Les tuples s'écrivent avec des parenthèses et ne sont pas modifiables.

    \subsection{2.3 Les spécificités des
listes}\label{les-spuxe9cificituxe9s-des-listes}

Nous avons vu que les listes étaient des éléments modifiables pouvant
contenir différents types de données.\\Il est bien sûr possible de
modifier les éléments d'une liste:

    \begin{Verbatim}[commandchars=\\\{\}]
{\color{incolor}In [{\color{incolor}}]:} \PY{n}{l}\PY{o}{=}\PY{p}{[}\PY{l+m+mi}{1}\PY{p}{,}\PY{l+m+mi}{2}\PY{p}{,}\PY{l+m+mi}{3}\PY{p}{,}\PY{l+m+mi}{4}\PY{p}{]}
       \PY{n}{l}
\end{Verbatim}

    \begin{Verbatim}[commandchars=\\\{\}]
{\color{incolor}In [{\color{incolor}}]:} \PY{n}{l}\PY{p}{[}\PY{l+m+mi}{0}\PY{p}{]}\PY{o}{=}\PY{l+m+mi}{0}
       \PY{n}{l}
\end{Verbatim}

    Plusieurs \textbf{méthodes} peuvent être employées pour ajouter des
éléments a une liste existante (sans réaffectation).\\Voici la liste des
méthodes pour les listes.

   \begin{longtable}[c]{@{}cccccc@{}}
\toprule\addlinespace
a & b & c & d & e & f
\\\addlinespace
\midrule\endhead
l.append & l.copy & l.extend & l.insert & l.remove & l.sort
\\\addlinespace
l.clear & l.count & l.index & l.pop & l.reverse
\\\addlinespace
\bottomrule
\end{longtable}

    \subsubsection{a. Ajouter des
éléments}\label{a.-ajouter-des-uxe9luxe9ments}

La méthode \textbf{\emph{append(x)}} permet de rajouter l'élément x à la
fin de la liste.

    \begin{Verbatim}[commandchars=\\\{\}]
{\color{incolor}In [{\color{incolor}}]:} \PY{n}{l}\PY{o}{=}\PY{p}{[}\PY{l+m+mi}{1}\PY{p}{,}\PY{l+m+mi}{2}\PY{p}{,}\PY{l+m+mi}{3}\PY{p}{,}\PY{l+m+mi}{4}\PY{p}{]}
       \PY{n}{l}\PY{o}{.}\PY{n}{append}\PY{p}{(}\PY{l+m+mi}{5}\PY{p}{)}
       \PY{n}{l}
\end{Verbatim}

    La méthode \textbf{\emph{insert(n,x)}} permet d'insérer l'élément x dans
la liste à l'indice n.

    \begin{Verbatim}[commandchars=\\\{\}]
{\color{incolor}In [{\color{incolor}}]:} \PY{n}{l}\PY{o}{=}\PY{p}{[}\PY{l+m+mi}{1}\PY{p}{,}\PY{l+m+mi}{2}\PY{p}{,}\PY{l+m+mi}{3}\PY{p}{,}\PY{l+m+mi}{4}\PY{p}{]}
       \PY{n}{l}\PY{o}{.}\PY{n}{insert}\PY{p}{(}\PY{l+m+mi}{1}\PY{p}{,}\PY{l+m+mi}{8}\PY{p}{)}
       \PY{n}{l}
\end{Verbatim}

    La méthode \textbf{\emph{extend(l2)}} permet de réaliser la
concaténation de deux listes sans réaffectation.\\Cette opération est
réalisable avec réaffectation en utilisant l'opérateur \textbf{+} :

    \begin{Verbatim}[commandchars=\\\{\}]
{\color{incolor}In [{\color{incolor}}]:} \PY{n}{l1}\PY{o}{=}\PY{p}{[}\PY{l+m+mi}{1}\PY{p}{,}\PY{l+m+mi}{2}\PY{p}{,}\PY{l+m+mi}{3}\PY{p}{]}
       \PY{n}{l2}\PY{o}{=}\PY{p}{[}\PY{l+m+mi}{4}\PY{p}{,}\PY{l+m+mi}{5}\PY{p}{,}\PY{l+m+mi}{6}\PY{p}{]}
       \PY{n}{l1}\PY{o}{.}\PY{n}{extend}\PY{p}{(}\PY{n}{l2}\PY{p}{)}
       \PY{n}{l1}
\end{Verbatim}

    \begin{Verbatim}[commandchars=\\\{\}]
{\color{incolor}In [{\color{incolor}}]:} \PY{n}{l1}\PY{o}{=}\PY{p}{[}\PY{l+m+mi}{1}\PY{p}{,}\PY{l+m+mi}{2}\PY{p}{,}\PY{l+m+mi}{3}\PY{p}{]}
       \PY{n}{l1}\PY{o}{=}\PY{n}{l1}\PY{o}{+}\PY{n}{l2}  \PY{c}{\PYZsh{}On peur remplacer par l1+=l2}
       \PY{n}{l1}
\end{Verbatim}

    \subsubsection{b. Supprimer des
éléments}\label{b.-supprimer-des-uxe9luxe9ments}

    \begin{itemize}
\itemsep1pt\parskip0pt\parsep0pt
\item
  La méthode \textbf{\emph{clear()}} (depuis python3.3) permet d'effacer
  la liste
\end{itemize}

    \begin{Verbatim}[commandchars=\\\{\}]
{\color{incolor}In [{\color{incolor}}]:} \PY{n}{l}\PY{o}{.}\PY{n}{clear}\PY{p}{(}\PY{p}{)}
       \PY{n}{l}
\end{Verbatim}

    Si la version de python est inférieur à 3.2 alors on utilisera

    \begin{Verbatim}[commandchars=\\\{\}]
{\color{incolor}In [{\color{incolor}}]:} \PY{n}{l}\PY{o}{=}\PY{p}{[}\PY{p}{]}
\end{Verbatim}

    \begin{itemize}
\itemsep1pt\parskip0pt\parsep0pt
\item
  \textbf{\emph{remove(x)}} supprime la première occurrence de x.
\end{itemize}

    \begin{Verbatim}[commandchars=\\\{\}]
{\color{incolor}In [{\color{incolor}}]:} \PY{n}{l}\PY{o}{=}\PY{p}{[}\PY{l+m+mi}{1}\PY{p}{,}\PY{l+m+mi}{2}\PY{p}{,}\PY{l+m+mi}{3}\PY{p}{,}\PY{l+m+mi}{1}\PY{p}{]}
       \PY{n}{l}\PY{o}{.}\PY{n}{remove}\PY{p}{(}\PY{l+m+mi}{1}\PY{p}{)}
       \PY{n}{l}
\end{Verbatim}

    \begin{Verbatim}[commandchars=\\\{\}]
{\color{incolor}In [{\color{incolor}}]:} \PY{n}{l}\PY{o}{.}\PY{n}{remove}\PY{p}{(}\PY{l+m+mi}{1}\PY{p}{)}
       \PY{n}{l}
\end{Verbatim}

    \begin{Verbatim}[commandchars=\\\{\}]
{\color{incolor}In [{\color{incolor}}]:} \PY{n}{l}\PY{o}{.}\PY{n}{remove}\PY{p}{(}\PY{l+m+mi}{1}\PY{p}{)}
\end{Verbatim}

    \begin{itemize}
\itemsep1pt\parskip0pt\parsep0pt
\item
  La méthode \textbf{\emph{l.pop({[}i{]})}} Retourne l'élément de la
  liste l d'indice i et supprime cet élément de la liste.
\end{itemize}

    \begin{Verbatim}[commandchars=\\\{\}]
{\color{incolor}In [{\color{incolor}}]:} \PY{n}{l}\PY{o}{=}\PY{p}{[}\PY{l+m+mi}{1}\PY{p}{,}\PY{l+m+mi}{2}\PY{p}{,}\PY{l+m+mi}{3}\PY{p}{,}\PY{l+m+mi}{1}\PY{p}{]}
       \PY{n}{a}\PY{o}{=}\PY{n}{l}\PY{o}{.}\PY{n}{pop}\PY{p}{(}\PY{l+m+mi}{2}\PY{p}{)}
       \PY{k}{print}\PY{p}{(}\PY{n}{a}\PY{p}{)}
       \PY{k}{print}\PY{p}{(}\PY{n}{l}\PY{p}{)}
\end{Verbatim}

    \begin{itemize}
\itemsep1pt\parskip0pt\parsep0pt
\item
  La fonction \textbf{\emph{del}} supprime aussi un élément précis en
  fonction de son indice :
\end{itemize}

    \begin{Verbatim}[commandchars=\\\{\}]
{\color{incolor}In [{\color{incolor}}]:} \PY{n}{l}\PY{o}{=}\PY{p}{[}\PY{l+m+mi}{1}\PY{p}{,}\PY{l+m+mi}{2}\PY{p}{,}\PY{l+m+mi}{3}\PY{p}{,}\PY{l+m+mi}{1}\PY{p}{]}
       \PY{k}{del} \PY{n}{l}\PY{p}{[}\PY{l+m+mi}{2}\PY{p}{]}
       \PY{n}{l}
\end{Verbatim}

    \subsubsection{c. Autres méthodes}\label{c.-autres-muxe9thodes}

\begin{itemize}
\itemsep1pt\parskip0pt\parsep0pt
\item
  Pour savoir si un élément appartient bien à une liste on utilise le
  mot-clé \textbf{\emph{in}}.\\Si l'élément testé est dans la liste, la
  valeur retournée sera \textbf{\emph{True}} et sinon ce sera
  \textbf{\emph{False}}:
\end{itemize}

    \begin{Verbatim}[commandchars=\\\{\}]
{\color{incolor}In [{\color{incolor}}]:} \PY{n}{distrib} \PY{o}{=} \PY{p}{[}\PY{l+s}{\PYZsq{}}\PY{l+s}{debian}\PY{l+s}{\PYZsq{}}\PY{p}{,}\PY{l+s}{\PYZsq{}}\PY{l+s}{ubuntu}\PY{l+s}{\PYZsq{}}\PY{p}{,}\PY{l+s}{\PYZsq{}}\PY{l+s}{fedora}\PY{l+s}{\PYZsq{}}\PY{p}{]}
       \PY{l+s}{\PYZsq{}}\PY{l+s}{debian}\PY{l+s}{\PYZsq{}} \PY{o+ow}{in} \PY{n}{distrib}
\end{Verbatim}

    \begin{Verbatim}[commandchars=\\\{\}]
{\color{incolor}In [{\color{incolor}}]:} \PY{l+s}{\PYZsq{}}\PY{l+s}{mandriva}\PY{l+s}{\PYZsq{}} \PY{o+ow}{in} \PY{n}{distrib}
\end{Verbatim}

    \begin{itemize}
\itemsep1pt\parskip0pt\parsep0pt
\item
  Il est également possible d'obtenir l'indice de la première occurrence
  d'un élément par la méthode \textbf{\emph{index()}}:
\end{itemize}

    \begin{Verbatim}[commandchars=\\\{\}]
{\color{incolor}In [{\color{incolor}}]:} \PY{n}{distrib}\PY{o}{.}\PY{n}{index}\PY{p}{(}\PY{l+s}{\PYZsq{}}\PY{l+s}{ubuntu}\PY{l+s}{\PYZsq{}}\PY{p}{)}
\end{Verbatim}

    \begin{itemize}
\itemsep1pt\parskip0pt\parsep0pt
\item
  Pour connaître la taille d'une liste (nombre d'éléments qu'elle
  contient), on pourra utiliser comme pour les chaîne de caractère la
  fonction \textbf{\emph{len()}}:
\end{itemize}

Mais attention : cette fonction ne compte que les éléments de
\emph{\textless{}\textless{} premier niveau
\textgreater{}\textgreater{}} ! Si vous avez une structure de liste
complexe contenant des sous-listes, chaque sous-liste ne comptera que
comme un seul élément:

    \begin{Verbatim}[commandchars=\\\{\}]
{\color{incolor}In [{\color{incolor}}]:} \PY{n}{l}\PY{o}{=}\PY{p}{[}\PY{l+m+mi}{1}\PY{p}{,}\PY{l+m+mi}{2}\PY{p}{,}\PY{p}{[}\PY{l+s}{\PYZsq{}}\PY{l+s}{a}\PY{l+s}{\PYZsq{}}\PY{p}{,}\PY{l+s}{\PYZsq{}}\PY{l+s}{b}\PY{l+s}{\PYZsq{}}\PY{p}{,}\PY{p}{[}\PY{l+s}{\PYZsq{}}\PY{l+s}{000}\PY{l+s}{\PYZsq{}}\PY{p}{,}\PY{l+s}{\PYZsq{}}\PY{l+s}{001}\PY{l+s}{\PYZsq{}}\PY{p}{,}\PY{l+s}{\PYZsq{}}\PY{l+s}{010}\PY{l+s}{\PYZsq{}}\PY{p}{]}\PY{p}{]}\PY{p}{]}
       \PY{n+nb}{len}\PY{p}{(}\PY{n}{l}\PY{p}{)}
\end{Verbatim}

    \begin{Verbatim}[commandchars=\\\{\}]
{\color{incolor}In [{\color{incolor}}]:} \PY{n}{l}\PY{p}{[}\PY{l+m+mi}{2}\PY{p}{]}
\end{Verbatim}

    \begin{Verbatim}[commandchars=\\\{\}]
{\color{incolor}In [{\color{incolor}}]:} \PY{n}{l}\PY{p}{[}\PY{l+m+mi}{2}\PY{p}{]}\PY{p}{[}\PY{l+m+mi}{2}\PY{p}{]}
\end{Verbatim}

    \begin{Verbatim}[commandchars=\\\{\}]
{\color{incolor}In [{\color{incolor}}]:} \PY{n}{l}\PY{p}{[}\PY{l+m+mi}{2}\PY{p}{]}\PY{p}{[}\PY{l+m+mi}{2}\PY{p}{]}\PY{p}{[}\PY{l+m+mi}{1}\PY{p}{]}
\end{Verbatim}

    \subsubsection{d. Les compréhensions de
listes}\label{d.-les-compruxe9hensions-de-listes}

Nous souhaitons obtenir la liste des carrés pour i allant de 0 à
20.\\Une première méthode peut être celle-ci

    \begin{Verbatim}[commandchars=\\\{\}]
{\color{incolor}In [{\color{incolor}}]:} \PY{n}{carre}\PY{o}{=}\PY{p}{[}\PY{p}{]}                 \PY{c}{\PYZsh{} On créer une liste vide}
       \PY{k}{for} \PY{n}{i} \PY{o+ow}{in} \PY{n+nb}{range}\PY{p}{(}\PY{l+m+mi}{21}\PY{p}{)}\PY{p}{:}      \PY{c}{\PYZsh{} pour i allant de 0 à 20}
           \PY{n}{carre}\PY{o}{.}\PY{n}{append}\PY{p}{(}\PY{n}{i}\PY{o}{*}\PY{o}{*}\PY{l+m+mi}{2}\PY{p}{)}   \PY{c}{\PYZsh{} On ajoute le carré}
       \PY{k}{print}\PY{p}{(}\PY{n}{carre}\PY{p}{)}             \PY{c}{\PYZsh{} On affiche la liste}
\end{Verbatim}

    Python implémente un mécanisme appelé \textless{}\textless{}
\textbf{\emph{compréhension de listes}} \textgreater{}\textgreater{},
permettant d'utiliser une fonction qui sera appliquée sur chacun des
éléments d'une liste.

    \begin{Verbatim}[commandchars=\\\{\}]
{\color{incolor}In [{\color{incolor}}]:} \PY{n}{l}\PY{o}{=}\PY{p}{[}\PY{n}{i}\PY{o}{*}\PY{o}{*}\PY{l+m+mi}{2} \PY{k}{for} \PY{n}{i} \PY{o+ow}{in} \PY{n+nb}{range}\PY{p}{(}\PY{l+m+mi}{21}\PY{p}{)}\PY{p}{]}
       \PY{k}{print}\PY{p}{(}\PY{n}{l}\PY{p}{)}
\end{Verbatim}

    À l'aide de l'instruction \emph{for i in range(21)}, on récupère les
entiers i de 0 à 20, et on place les carré (i**2 ) dans la liste.\\Ce
mécanisme peut produire des résultats plus complexes et on peut aussi
appliquer une condition aux éléments \emph{i} à utiliser.

    \begin{Verbatim}[commandchars=\\\{\}]
{\color{incolor}In [{\color{incolor}}]:} \PY{n}{carre}\PY{o}{=}\PY{p}{[}\PY{n}{i}\PY{o}{*}\PY{o}{*}\PY{l+m+mi}{2} \PY{k}{for} \PY{n}{i} \PY{o+ow}{in} \PY{n+nb}{range}\PY{p}{(}\PY{l+m+mi}{21}\PY{p}{)} \PY{k}{if} \PY{n}{i}\PY{o}{\PYZpc{}}\PY{k}{2}\PY{o}{==}\PY{l+m+mi}{0} \PY{o+ow}{and} \PY{n}{i}\PY{o}{\PYZgt{}}\PY{l+m+mi}{2}\PY{p}{]}  \PY{c}{\PYZsh{} si i est pair et strictement supérieur à 2}
       \PY{n}{carre}
\end{Verbatim}

    \begin{Verbatim}[commandchars=\\\{\}]
{\color{incolor}In [{\color{incolor}}]:} \PY{k+kn}{import} \PY{n+nn}{random}
       \PY{n}{l}\PY{o}{=}\PY{p}{[}\PY{n}{random}\PY{o}{.}\PY{n}{randint}\PY{p}{(}\PY{l+m+mi}{1}\PY{p}{,}\PY{l+m+mi}{6}\PY{p}{)} \PY{k}{for} \PY{n}{i} \PY{o+ow}{in} \PY{n+nb}{range}\PY{p}{(}\PY{l+m+mi}{20}\PY{p}{)}\PY{p}{]}
       \PY{n}{l} \PY{c}{\PYZsh{} liste de 20 nombres aléatoires entre 1 et 6}
\end{Verbatim}

    Il y a donc différentes façon de construire une liste

    ou par compréhension

    les compréhensions de listes peuvent également être utilisés avec les
dictionnaires.

    \subsubsection{e. À retenir}\label{e.-uxe0-retenir}

Il faut savoir ajouter, rechercher , enlever un élément dans une liste
Les méthodes à connaitre sont \emph{append(), insert(), remove(),
count(), index(), pop()} ou la fonction \emph{del}

\subsubsection{f. Exercices}\label{f.-exercices}

\textbf{Exercice 1}: Soit $(u_n)$ la suite définie par $ \left\lbrace  \begin{array}{l}
u_0=1,2 \\
u_{n+1}=-2u_n+1
  \end{array}
\right.$

Demander à l'utilisateur de saisir un entier n puis afficher dans une
liste tous les termes de la suite $(u_n)$.\\Afficher enfin la somme avec
la fonction \emph{sum()}.

    \textbf{Exercice 2}: On met un grain de riz sur la 1° case d'un
échiquier, deux sur la deuxième, 4 sur la troisième, 8 sur la quatrième
etc\ldots{}.\\Combien y a t il de grain de grain sur l'échiquier? On
fera apparaitre dans une liste le nombre de nombre de grain de chaque
case.\\idem pour un damier

    \textbf{Exercice 3}:\\Écrire un programme qui inverse les mots (séparés
par un espace) d'une phrase.

Entrée: `Un matin nous partons, le cerveau plein de flamme'\\sortie:
`flamme de plein cerveau le partons, nous matin un'

    \textbf{\emph{Exercice 4:}}\\Soit $(u_n)$ et $(v_n)$ les suites définies
par $u_0=2,\quad \forall n \in N \ v_{n+1}=\dfrac{2}{u_n}$ et
$u_{n+1}=\dfrac{u_n+v_n}{2}$.

Déterminer dans les listes U et V les termes des suites $(u_n)$ et
$(v_n)$

    \textbf{\emph{Exercice 5:}} \emph{Une marche aléatoire}

Un point $M$ peut se déplacer sur un quadrillage d'un pas dans l'une des
quatre directions. Les quatre déplacements possibles se font au hasard
(ils sont donc équiprobables). La position de $M$ est repérée par ses
coordonnées $(x;y)$ entières dans le repère indiqué sur la figure. Au
départ $M$ est en $(0,0)$.

On continue les déplacements jusqu'à ce que $M$ sorte du cercle de
centre $O$ et de rayon 10.

On appelle $N$ le nombre de pas duquel $M$ sort du cercle pour la
première fois \emph{($N$ est donc une variable aléatoire)}.

Le but de l'exercice est d'essayer, en utilisant des simulations,
d'avoir une idée des probabilités des évènements suivants:

\begin{itemize}
\itemsep1pt\parskip0pt\parsep0pt
\item
  A: \textless{}\textless{} $M$ sort du cercle en moins de 20 pas
  \textgreater{}\textgreater{} (soit $N\leq 20)$,
\item
  B: \textless{}\textless{} $M$ sort du cercle après un nombre de pas
  compris entre 21 et 30 \textgreater{}\textgreater{} (soit
  $20<N\leq 30)$,
\item
  C: \textless{}\textless{} $M$ sort du cercle après un nombre de pas
  compris entre 31 et 40 \textgreater{}\textgreater{} (soit
  $30<N\leq 40)$,
\item
  D: \textless{}\textless{} $M$ sort du cercle en plus de 40 pas
  \textgreater{}\textgreater{} (soit $40<N)$,
\end{itemize}

Simuler un programme pour faire une simulation, puis pour faire 2000
simulations.

On Mettra les nombres N dans une liste L.

    \subsection{2.4 Les spécificités des
dictionnaires}\label{les-spuxe9cificituxe9s-des-dictionnaires}

Comme nous l'avons vu, les dictionnaires sont des listes d'éléments
indicés par des clés.\\Comme pour les listes, la commande
\textbf{\emph{del}} permet de supprimer un élément et son mot clé ,
\textless{}\textless{} \textbf{\emph{in}} \textgreater{}\textgreater{}
permet de vérifier l'existence d'une clé :

    \begin{Verbatim}[commandchars=\\\{\}]
{\color{incolor}In [{\color{incolor}}]:} \PY{n}{d}\PY{o}{=}\PY{p}{\PYZob{}}\PY{l+s}{\PYZsq{}}\PY{l+s}{cle1}\PY{l+s}{\PYZsq{}}\PY{p}{:}\PY{l+m+mi}{1}\PY{p}{,} \PY{l+s}{\PYZsq{}}\PY{l+s}{cle2}\PY{l+s}{\PYZsq{}}\PY{p}{:}\PY{l+m+mi}{2}\PY{p}{,} \PY{l+s}{\PYZsq{}}\PY{l+s}{cle3}\PY{l+s}{\PYZsq{}}\PY{p}{:}\PY{l+m+mi}{3}\PY{p}{\PYZcb{}}
       \PY{k}{del} \PY{n}{d}\PY{p}{[}\PY{l+s}{\PYZsq{}}\PY{l+s}{cle2}\PY{l+s}{\PYZsq{}}\PY{p}{]}
       \PY{n}{d}
\end{Verbatim}

    \begin{Verbatim}[commandchars=\\\{\}]
{\color{incolor}In [{\color{incolor}}]:} \PY{l+s}{\PYZsq{}}\PY{l+s}{cle1}\PY{l+s}{\PYZsq{}} \PY{o+ow}{in} \PY{n}{d}
\end{Verbatim}

    \begin{Verbatim}[commandchars=\\\{\}]
{\color{incolor}In [{\color{incolor}}]:} \PY{l+s}{\PYZsq{}}\PY{l+s}{cle2}\PY{l+s}{\PYZsq{}} \PY{o+ow}{in} \PY{n}{d}
\end{Verbatim}

    Vous aurez pu remarquer lors de l'affichage du dictionnaire
\textbf{\emph{d}}, que les couples \emph{clé/valeur} n'étaient pas
affichés dans l'ordre de création.\\C'est tout à fait normal car les
dictionnaires ne sont pas ordonnés !\\Plusieurs méthodes permettent de
récupérer des listes de clés, de valeurs et de couples \emph{clé/valeur}
:

    \begin{Verbatim}[commandchars=\\\{\}]
{\color{incolor}In [{\color{incolor}}]:} \PY{n}{courses} \PY{o}{=}\PY{p}{\PYZob{}}\PY{l+s}{\PYZsq{}}\PY{l+s}{pommes}\PY{l+s}{\PYZsq{}}\PY{p}{:}\PY{l+m+mi}{3}\PY{p}{,} \PY{l+s}{\PYZsq{}}\PY{l+s}{poires}\PY{l+s}{\PYZsq{}}\PY{p}{:}\PY{l+m+mi}{5} \PY{p}{,} \PY{l+s}{\PYZsq{}}\PY{l+s}{kiwis}\PY{l+s}{\PYZsq{}}\PY{p}{:}\PY{l+m+mi}{7}\PY{p}{\PYZcb{}}
       \PY{n}{courses}\PY{o}{.}\PY{n}{keys}\PY{p}{(}\PY{p}{)}
\end{Verbatim}

    \begin{Verbatim}[commandchars=\\\{\}]
{\color{incolor}In [{\color{incolor}}]:} \PY{n}{courses}\PY{o}{.}\PY{n}{values}\PY{p}{(}\PY{p}{)}
\end{Verbatim}

    \begin{Verbatim}[commandchars=\\\{\}]
{\color{incolor}In [{\color{incolor}}]:} \PY{n}{courses}\PY{o}{.}\PY{n}{items}\PY{p}{(}\PY{p}{)}
\end{Verbatim}

    Ces méthodes renvoient \textbf{des objets itérables}: ce ne sont pas des
listes et ils ne consomment donc pas autant de place mémoire que pour
stocker une liste, on ne stocke qu'un \emph{pointeur} vers l'élément
courant.\\Avec ces structures vous pourrez toujours utiliser les boucles
\textbf{\emph{for}} classiques, par contre vous n'aurez plus un accès
direct aux valeurs en spécifiant leur indice.

Dans un dictionnaire, pour obtenir la valeur correspondant à une clé on
peut bien entendu utiliser la notation classique
\textbf{\emph{dictionnaire{[}clé{]}}}, mais on peut aussi utiliser la
méthode \textbf{\emph{get()}} qui renvoie la valeur associée a la clé
passée en premier paramètre ou, si elle n'existe pas, la valeur passée
en second paramétré :

    \begin{Verbatim}[commandchars=\\\{\}]
{\color{incolor}In [{\color{incolor}}]:} \PY{n}{courses} \PY{o}{=}\PY{p}{\PYZob{}} \PY{l+s}{\PYZsq{}}\PY{l+s}{pommes}\PY{l+s}{\PYZsq{}}\PY{p}{:}\PY{l+m+mi}{3}\PY{p}{,} \PY{l+s}{\PYZsq{}}\PY{l+s}{poires}\PY{l+s}{\PYZsq{}}\PY{p}{:}\PY{l+m+mi}{5} \PY{p}{,} \PY{l+s}{\PYZsq{}}\PY{l+s}{kiwis}\PY{l+s}{\PYZsq{}}\PY{p}{:}\PY{l+m+mi}{7}\PY{p}{\PYZcb{}}
       \PY{n}{courses}\PY{o}{.}\PY{n}{get}\PY{p}{(} \PY{l+s}{\PYZsq{}}\PY{l+s}{pommes}\PY{l+s}{\PYZsq{}} \PY{p}{,} \PY{l+s}{\PYZsq{}}\PY{l+s}{stock epuise}\PY{l+s}{\PYZsq{}} \PY{p}{)}
\end{Verbatim}

    \begin{Verbatim}[commandchars=\\\{\}]
{\color{incolor}In [{\color{incolor}}]:} \PY{n}{courses}\PY{o}{.}\PY{n}{get}\PY{p}{(}\PY{l+s}{\PYZsq{}}\PY{l+s}{salades}\PY{l+s}{\PYZsq{}} \PY{p}{,} \PY{l+s}{\PYZsq{}}\PY{l+s}{stock epuise}\PY{l+s}{\PYZsq{}}\PY{p}{)}
\end{Verbatim}

    On peut fusionner deux dictionnaires avec la méthode
\textbf{\emph{update()}} : Notez que si une clé existe déjà, la valeur
stockée sera écrasée:

    \begin{Verbatim}[commandchars=\\\{\}]
{\color{incolor}In [{\color{incolor}}]:} \PY{n}{courses} \PY{o}{=}\PY{p}{\PYZob{}} \PY{l+s}{\PYZsq{}}\PY{l+s}{pommes}\PY{l+s}{\PYZsq{}}\PY{p}{:}\PY{l+m+mi}{3}\PY{p}{,} \PY{l+s}{\PYZsq{}}\PY{l+s}{poires}\PY{l+s}{\PYZsq{}}\PY{p}{:}\PY{l+m+mi}{5} \PY{p}{,} \PY{l+s}{\PYZsq{}}\PY{l+s}{kiwis}\PY{l+s}{\PYZsq{}}\PY{p}{:}\PY{l+m+mi}{7}\PY{p}{\PYZcb{}}
       \PY{n}{courses2}\PY{o}{=}\PY{p}{\PYZob{}} \PY{l+s}{\PYZsq{}}\PY{l+s}{kiwis}\PY{l+s}{\PYZsq{}}\PY{p}{:}\PY{l+m+mi}{3}\PY{p}{,} \PY{l+s}{\PYZsq{}}\PY{l+s}{salades}\PY{l+s}{\PYZsq{}}\PY{p}{:}\PY{l+m+mi}{2}\PY{p}{\PYZcb{}}
       \PY{n}{courses}\PY{o}{.}\PY{n}{update}\PY{p}{(}\PY{n}{courses2}\PY{p}{)}
       \PY{n}{courses}
\end{Verbatim}

    \subsection{2.5 Le slicing}\label{le-slicing}

Le \textbf{\emph{slicing}} est une méthode applicable à tous les objets
de type liste ordonnée (donc pas aux dictionnaires). ll s'agit d'un
\textless{}\textless{} découpage en tranches
\textgreater{}\textgreater{} des éléments d'une liste de manière à
récupérer des objets respectant cette découpe.

Pour cela, nous devrons spécifier l'indice de l'élément de départ,
l'indice de l'élément d'arrivée (qui ne sera pas compris dans la plage)
et le pas de déplacement. Pour une variable v donnée, l'écriture se fera
en utilisant la notation entre crochets et en séparant chacun des
paramètres par le caractère deux-points:
\textbf{\emph{v{[}début:fin:pas{]}}}. Cette écriture peut se traduire
par : \textless{}\textless{} les caractères de la variable
\textbf{\emph{v}} depuis l'indice \textbf{\emph{début}} jusqu'à l'indice
\textbf{\emph{fin}} non compris avec un déplacement de
\textbf{\emph{pas}} caractère(s) \textgreater{}\textgreater{}.

Pour bien comprendre le fonctionnement du slicing, nous commencerons par
l'appliquer aux chaînes de caractères avant de voir les listes et les
tuples.

\subsubsection{a) Le slicing sur une chaîne de
caratère}\label{a-le-slicing-sur-une-chauxeene-de-caratuxe8re}

Voici un premier exemple simple

    \begin{Verbatim}[commandchars=\\\{\}]
{\color{incolor}In [{\color{incolor}}]:} \PY{n}{c}\PY{o}{=}\PY{l+s}{\PYZsq{}}\PY{l+s}{Pablo Neruda}\PY{l+s}{\PYZsq{}}
       \PY{n}{c}\PY{p}{[}\PY{p}{:}\PY{l+m+mi}{5}\PY{p}{]}
\end{Verbatim}

    \begin{Verbatim}[commandchars=\\\{\}]
{\color{incolor}In [{\color{incolor}}]:} \PY{n}{c}\PY{p}{[}\PY{l+m+mi}{6}\PY{p}{:}\PY{l+m+mi}{12}\PY{p}{]}
\end{Verbatim}

    Vous avez remarque que dans ce code aucune indication de pas n'a été
donnée : c'est la valeur par défaut qui est alors utilisée, c'est-a-dire
\textbf{\emph{1}}. De même, si la valeur de début est omise, la valeur
par défaut utilisée sera \textbf{\emph{0}} et si la valeur de fin est
omise, la valeur par défaut utilisée sera la \emph{taille de la
chaîne+1}.

    \begin{Verbatim}[commandchars=\\\{\}]
{\color{incolor}In [{\color{incolor}}]:} \PY{n}{c}\PY{p}{[}\PY{p}{:}\PY{l+m+mi}{5}\PY{p}{]} \PY{c}{\PYZsh{} équivaut à c[0:5], équivaut à c[0:5:1]}
\end{Verbatim}

    \begin{Verbatim}[commandchars=\\\{\}]
{\color{incolor}In [{\color{incolor}}]:} \PY{n}{c}\PY{p}{[}\PY{l+m+mi}{6}\PY{p}{:}\PY{p}{]} \PY{c}{\PYZsh{} équivaut à c[6:12], équivaut à c[6:12:1]}
\end{Verbatim}

    Du fait de ces valeurs par défaut, que pensez vous que l'on sélectionne
en tapant: c{[}:{]} ou encore c{[}::{]}?\\La chaîne entière, bien sûr :

    \begin{Verbatim}[commandchars=\\\{\}]
{\color{incolor}In [{\color{incolor}}]:} \PY{n}{c}\PY{p}{[}\PY{p}{:}\PY{p}{]}
\end{Verbatim}

    Il devient alors très simple d'inverser une chaîne en utilisant le pas :

    \begin{Verbatim}[commandchars=\\\{\}]
{\color{incolor}In [{\color{incolor}}]:} \PY{n}{c}\PY{p}{[}\PY{p}{:}\PY{p}{:}\PY{o}{\PYZhy{}}\PY{l+m+mi}{1}\PY{p}{]}
\end{Verbatim}

    Une première solution pour effectuer une copie peut être d'utiliser le
slicing.\\En effet, l'opération {[}:{]} renvoie une nouvelle liste, ce
qui résout le problème des pointeurs vers la même zone mémoire, comme le
montrent l'exemple suivant :

    \begin{Verbatim}[commandchars=\\\{\}]
{\color{incolor}In [{\color{incolor}}]:} \PY{n}{listea} \PY{o}{=}\PY{p}{[}\PY{l+m+mi}{1}\PY{p}{,} \PY{l+m+mi}{2}\PY{p}{,} \PY{l+m+mi}{3}\PY{p}{]}
       \PY{n}{listeb} \PY{o}{=}\PY{n}{listea}\PY{p}{[}\PY{p}{:}\PY{p}{]}
       \PY{n}{listea}
\end{Verbatim}

    \begin{Verbatim}[commandchars=\\\{\}]
{\color{incolor}In [{\color{incolor}}]:} \PY{n}{listeb}
\end{Verbatim}

    \begin{Verbatim}[commandchars=\\\{\}]
{\color{incolor}In [{\color{incolor}}]:} \PY{n}{listeb}\PY{p}{[}\PY{l+m+mi}{0}\PY{p}{]}\PY{o}{=}\PY{l+m+mi}{0}
       \PY{n}{listeb}
\end{Verbatim}

    \begin{Verbatim}[commandchars=\\\{\}]
{\color{incolor}In [{\color{incolor}}]:} \PY{n}{listea}
\end{Verbatim}

    Cette copie peut paraître satisfaisante\ldots{} et elle l'est, mais à
condition de ne manipuler que des listes de premier niveau ne comportant
aucune sous-liste :

    \begin{Verbatim}[commandchars=\\\{\}]
{\color{incolor}In [{\color{incolor}}]:} \PY{n}{listea}\PY{o}{=}\PY{p}{[}\PY{l+m+mi}{1}\PY{p}{,}\PY{l+m+mi}{2}\PY{p}{,}\PY{l+m+mi}{3}\PY{p}{,}\PY{p}{[}\PY{l+m+mi}{4}\PY{p}{,}\PY{l+m+mi}{5}\PY{p}{,}\PY{l+m+mi}{6}\PY{p}{]}\PY{p}{]}
       \PY{n}{listeb} \PY{o}{=}\PY{n}{listea}\PY{p}{[}\PY{p}{:}\PY{p}{]}
       \PY{n}{listea}
\end{Verbatim}

    \begin{Verbatim}[commandchars=\\\{\}]
{\color{incolor}In [{\color{incolor}}]:} \PY{n}{listeb}
\end{Verbatim}

    \begin{Verbatim}[commandchars=\\\{\}]
{\color{incolor}In [{\color{incolor}}]:} \PY{n}{listeb}\PY{p}{[}\PY{l+m+mi}{3}\PY{p}{]}\PY{p}{[}\PY{l+m+mi}{0}\PY{p}{]}\PY{o}{=}\PY{l+m+mi}{0}
       \PY{n}{listeb}
\end{Verbatim}

    \begin{Verbatim}[commandchars=\\\{\}]
{\color{incolor}In [{\color{incolor}}]:} \PY{n}{listea}
\end{Verbatim}

    En effet, le slicing n'effectue pas de copie récursive: en cas de
sous-liste, on retombe dans la problématique des pointeurs mémoire. La
solution est alors d'utiliser un module spécifique, le module
\textbf{\emph{copy}}, qui permet d'effectuer une copie récursive. Bien
que n'ayant pas encore approfondi la manipulation des modules, voici
comment réaliser une copie de liste par cette méthode:

    \begin{Verbatim}[commandchars=\\\{\}]
{\color{incolor}In [{\color{incolor}}]:} \PY{k+kn}{from} \PY{n+nn}{copy}   \PY{k+kn}{import} \PY{o}{*}
       \PY{n}{listea}\PY{o}{=}\PY{p}{[}\PY{l+m+mi}{1}\PY{p}{,}\PY{l+m+mi}{2}\PY{p}{,}\PY{l+m+mi}{3}\PY{p}{,}\PY{p}{[}\PY{l+m+mi}{4}\PY{p}{,}\PY{l+m+mi}{5}\PY{p}{,}\PY{l+m+mi}{6}\PY{p}{]}\PY{p}{]}
       \PY{n}{listeb}\PY{o}{=}\PY{n}{deepcopy}\PY{p}{(}\PY{n}{listea}\PY{p}{)}
       \PY{n}{listea}
\end{Verbatim}

    \begin{Verbatim}[commandchars=\\\{\}]
{\color{incolor}In [{\color{incolor}}]:} \PY{n}{listeb}
\end{Verbatim}

    \begin{Verbatim}[commandchars=\\\{\}]
{\color{incolor}In [{\color{incolor}}]:} \PY{n}{listeb}\PY{p}{[}\PY{l+m+mi}{3}\PY{p}{]}\PY{p}{[}\PY{l+m+mi}{0}\PY{p}{]}\PY{o}{=}\PY{l+m+mi}{0}
       \PY{n}{listeb}
\end{Verbatim}

    \begin{Verbatim}[commandchars=\\\{\}]
{\color{incolor}In [{\color{incolor}}]:} \PY{n}{listea}
\end{Verbatim}

    Le problème est exactement le même avec la copie de dictionnaires. Il
faudra utiliser ici la méthode \textbf{\emph{copy()}}:

    \begin{Verbatim}[commandchars=\\\{\}]
{\color{incolor}In [{\color{incolor}}]:} \PY{n}{dicoa}\PY{o}{=}\PY{p}{\PYZob{}} \PY{l+s}{\PYZsq{}}\PY{l+s}{cle1}\PY{l+s}{\PYZsq{}}\PY{p}{:}\PY{l+m+mi}{1}\PY{p}{,} \PY{l+s}{\PYZsq{}}\PY{l+s}{cle2}\PY{l+s}{\PYZsq{}}\PY{p}{:}\PY{l+m+mi}{2}\PY{p}{,} \PY{l+s}{\PYZsq{}}\PY{l+s}{cle3}\PY{l+s}{\PYZsq{}}\PY{p}{:}\PY{l+m+mi}{3}\PY{p}{\PYZcb{}}
       \PY{n}{dicob}\PY{o}{=}\PY{n}{dicoa}\PY{o}{.}\PY{n}{copy}\PY{p}{(}\PY{p}{)}
       \PY{n}{dicoa}
\end{Verbatim}

    \begin{Verbatim}[commandchars=\\\{\}]
{\color{incolor}In [{\color{incolor}}]:} \PY{n}{dicob}
\end{Verbatim}

    \begin{Verbatim}[commandchars=\\\{\}]
{\color{incolor}In [{\color{incolor}}]:} \PY{n}{dicob}\PY{p}{[}\PY{l+s}{\PYZsq{}}\PY{l+s}{cle1}\PY{l+s}{\PYZsq{}}\PY{p}{]}\PY{o}{=}\PY{l+m+mi}{0}
\end{Verbatim}

    \begin{Verbatim}[commandchars=\\\{\}]
{\color{incolor}In [{\color{incolor}}]:} \PY{n}{dicob}
\end{Verbatim}

    \begin{Verbatim}[commandchars=\\\{\}]
{\color{incolor}In [{\color{incolor}}]:} \PY{n}{dicoa}
\end{Verbatim}

    \subsection{Conclusion}\label{conclusion}

Les structures de liste en Python sont particulièrement fournies et il
existe pour chacune d'elles de nombreuses méthodes permettant de
manipuler leurs éléments.\\Le slicing représente un raccourci efficace
pour récupérer ou modifier des éléments d'une liste: il masque la
complexité des opérations sous une syntaxe claire et lisible\ldots{}
pour peu que l'on ait pris le temps de se familiariser avec elle.


    % Add a bibliography block to the postdoc
    
    
    
    \end{document}
