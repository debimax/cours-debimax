
%Fichier: tuto-lightty-python.tex
%Crée le 17 avril 2012
%Dernière modification: 16 mars 2019 13:19:44

\documentclass[a4paper,10pt,usenames]{article}
\usepackage[utf8]{inputenc}
\usepackage[T1]{fontenc}

%les fontes
\usepackage{lmodern}
%\usepackage{frcursive} %pour la fonte cursuive
\usepackage{pifont} % pour la fonte ding
\usepackage{textcomp} %%%Pour autoriser les caractères ° etc...

%mise en page
%\usepackage{multicol}
%\usepackage{lscape} %%% pour localement utiliser \begin{landscape}...\end{landscape}

%couleur
%\usepackage{colortbl}%%%pour colorer les cases d'un tableau
\usepackage[usenames,dvipsnames]{xcolor} % utiliser par exemple black!20
\definecolor{Bleulighttpd}{RGB}{75,105,131}
\definecolor{Rougelighttpd}{RGB}{204,0,0}
\definecolor{Jaunelighttpd}{RGB}{255,255,204}
\definecolor{Grispythoninfo}{RGB}{242,242,242}
\definecolor{Jaunepythoninfo}{RGB}{255,255,204}
%\usepackage[dvips,ps2pdf]{hyperref} %%%Pour les liens
\usepackage{graphicx} %%%%
\usepackage{colortbl}

%Encadrement etc etc
%\usepackage{fancybox} %%% pour encadrer du texte
%\usepackage{framed} %%% pour encadrer les formules etc ..
%\usepackage[amsmath,thmmarks,hyperref,framed]{ntheorem} %%%% Pour encadrer les thm, propriétés et définitions

\usepackage{array}
%\usepackage{enumerate} %%% pour les numerotation de liste \begin{enumerate}[i] ... %%%

% Pour les algorithmes
%\usepackage[french,boxed,vlined]{algorithm2e}  %,linesnumbered,inoutnumbered
%\usepackage{listings} %%% pour un environnement listings %%%%
%\input{mes_listings.tex} %%% pour les listings R-cran R etc... %%%%

%\usepackage{alltt} %%% pour un environnement programme bof %%%%

%\usepackage{pstricks-add}
%\usepackage{pst-eucl} %%%%
%\usepackage{pst-tree} %%%%


\usepackage[a4paper,tmargin=1.1cm,bmargin=2.4cm,lmargin=2cm,rmargin=1.7cm]{geometry} % Fixer les dimensions d'une page

\usepackage[francais]{babel}

%%% pour mes bas de pages %%%
%\usepackage{fancyhdr} %%%% Pour gérer les hauts et bas de pages.
%\pagestyle{fancy} %on met par défaut la page fancy
%\fancyhf{} %On efface tous pour définir ce que l'on veut
%\rfoot{\textit{Page \thepage}}
%\lfoot{\textit{Seconde}}
%\renewcommand{\footrulewidth}{0.4pt}
%\renewcommand{\headrulewidth}{0pt} %%%

% mon fichier listing se télécharge ic http://megamaths.free.fr/pdf/mes_listings.tex
\input{mes_listings.tex} %%% pour les listings R-cran xcas etc... %%%%
\usepackage{verbatim}

\usepackage[dvips,ps2pdf]{hyperref} %%%Pour les liens

%informations pour le lecteur pdf
\hypersetup{
	bookmarksopen=true,
	bookmarksopenlevel=0,
	pdfauthor=Meilland jean claude,
	pdftitle=tuoriel raspberry lighttpd fastcgi python  flask,
}

\setlength{\parskip}{0pt}% espace entre paragraphe
\usepackage{titlesec}
\titlespacing{\section}{0pt}{3pt}{1pt}


\title{Installer un serveur web  configuré pour python. (lighttpd+fastcgi+Flask)}
\author{Meilland jean claude}
%\date{}

\begin{document}
\maketitle


Avant de commencer je précise que vous avez un tutoriel  pour \href{https://raspbian-france.fr/installer-serveur-web-raspberry-lamp/}{Apache + PHP + Mysql} sur raspbian-france



\section{Installation}

\begin{list}{$\bullet$}{}
\item \textit{\textbf{lighttpd}} est un serveur léger donc idéal pour notre raspberry.
\item \textit{\textbf{Flask}} est un framwork  pour réaliser un site internet en python.
\item \textit{\textbf{Flup}} est un module python pour  \textit{Web Server Gateway Interface} (WSGI)
\end{list}

Commençons par installer le serveur et flask.

\lstset{title={},caption={}, style=bash}
\begin{lstlisting}
sudo apt-get install  lighttpd
sudo apt-get install  python3-flask
sudo pip3 install flup
\end{lstlisting}

Les différents modules que l'on peux charger se trouvent dans le dossier \textit{/etc/lighttpd/conf-available/}

\lstset{title={},caption={}, style=bash}
\begin{lstlisting}
ls /etc/lighttpd/conf-available/
\end{lstlisting}

Nous utiliseront les modules \textit{\textbf{10-fastcgi.conf,  10-rewrite.conf}} et \textbf{\textit{10-userdir.conf}}.

\lstset{title={},caption={}, style=bash}
\begin{lstlisting}
sudo lighttpd-enable-mod  fastcgi
sudo lighttpd-enable-mod  rewrite
sudo lighttpd-enable-mod  userdir
\end{lstlisting}

Les modules chargés se trouvent dans le dossier  \textit{\textbf{/etc/lighttpd/conf-enabled/}}

\lstset{title={},caption={}, style=bash}
\begin{lstlisting}
ls /etc/lighttpd/conf-enabled/
\end{lstlisting}

Il ne reste plus qu'à relancer \textit{\textbf{lighttpd}}

\lstset{title={},caption={}, style=bash}
\begin{lstlisting}
sudo service  lighttpd restart
\end{lstlisting}

\section{Vérification}


\begin{list}{\ding{212}}{}
\item Vérifions que le \textit{\textbf{lighttpd}} soit fonctionnel.

Ouvrez un navigateur à l'adresse \url{http://localhost/}.  Vous devez voir cette page 


%\fbox{\begin{minipage}[t]{\linewidth}
%
%\colorbox{Bleulighttpd}{\begin{minipage}[t]{0.98\linewidth}
%	\begin{center}
%		\huge\textcolor{white}{\textbf{Placeholder page}}
%
%\vspace{0.3cm}
%\scriptsize
%%\footnotesize
%\textcolor{white}{The owner of this web site has not put up any web pages yet. Please come back later.}
%	\end{center}
%\end{minipage}
%}
%
%\scriptsize
%%\footnotesize
%\textbf{You should replace this page with your own web pages as soon as possible.}
%
%
%Unless you changed its configuration, your new server is configured as follows:
%
%\begin{list}{$\bullet$}{}
%\item Configuration files can be found in /etc/lighttpd. Please read /etc/lighttpd/conf-available/README file.
%\item The DocumentRoot, which is the directory under which all your HTML files should exist, is set to /var/www/html.
%\item CGI scripts are looked for in /usr/www/cgi-bin, which is where Debian packages will place their scripts. You can enable cgi module by using command "lighty-enable-mod cgi".
%\item Log files are placed in /var/log/lighttpd, and will be rotated weekly. The frequency of rotation can be easily changed by editing /etc/logrotate.d/lighttpd.
%\item The default directory index is index.html, meaning that requests for a directory /foo/bar/ will give the contents of the file /var/www/foo/bar/index.html if it exists (assuming that /var/www is your DocumentRoot).
%\item You can enable user directories by using command "\textbf{lighty-enable-mod userdir}"
%\end{list}
%
%About this page
%This is a placeholder page installed by the Debian release of the \colorbox{Jaunelighttpd}{\textcolor{Rougelighttpd}{Lighttpd server package}}.
%
%This computer has installed the Debian GNU/Linux operating system, but it has nothing to do with the Debian Project. Please do not contact the Debian Project about it.
%
%If you find a bug in this Lighttpd package, or in Lighttpd itself, please file a bug report on it. Instructions on doing this, and the list of known bugs of this package, can be found in the \colorbox{Jaunelighttpd}{\textcolor{Rougelighttpd}{\textbf{Debian Bug Tracking System.}}}
%\end{minipage}}

\begin{center}
	\raisebox{-0.1cm}{\includegraphics[width=0.8\linewidth]{images/placeholder}}
\end{center}
\normalsize 

\item Idem pour le module \textit{\textbf{userdir}}\\
	Ouvrez un navigateur à l'adresse \url{http://localhost/\string~pi/}, vous devez voir la même page.\\
Évidemment remplacez \textbf{\textit{pi}} par votre \textbf{\textit{\$USER}}.\\
	Le module \textit{\textbf{userdir}}  permet d'écrire les programmes dans le dossier \textbf{\textit{/home/pi/public\string_html/}}
\item Vérifions le module  \textit{\textbf{Flask}}. Commençons par tester un mini site \textit{Pythoninfo}  (comme \textbf{\textit{phpinfo}}).

\lstset{title={},caption={}, style=bash}
\begin{lstlisting}
mkdir ~/public_html/pythoninfo
sudo pip3 install pyinfo
\end{lstlisting}

%On utilises le dossier \textit{static} pour mettre les images, fichiers css etc\ldots et le dossier templates pour mettre évidement les templates (squelettes) html.

Créer le fichier \textit{\textbf{\string~/public\string_html/pythoninfo/pythoninfo.py}}

\lstset{title={},caption={}, style=PYTHON}
\begin{lstlisting}
#!/usr/bin/python3
# -*- coding: utf-8 -*-
from flask import Flask
app = Flask(__name__)

@app.route('/pyinfo')
@app.route('/')
def info():
    import pyinfo
    return pyinfo.info_as_html()

if __name__ == '__main__':
    app.run(debug=True)
\end{lstlisting}

On peut déjà tester ce fichier en local

\lstset{title={},caption={}, style=bash}
\begin{lstlisting}
python3 pythoninfo.py
\end{lstlisting}

Ouvrez votre navigateur à l'adresse \url{http://localhost:5000} Vous devez voir ceci

 
\footnotesize
\begin{center}
	\fbox{\begin{minipage}[t]{12cm}

	\begin{center}
		
	\begin{tabular}[]{|>{\raggedright}m{3cm} >{\raggedleft}m{8cm} |}
		\hline \textbf{Python 3.5.3} & \raisebox{-0.1cm}{\includegraphics[width=2cm]{images/python-logo}}\tabularnewline
		\hline 
	\end{tabular} 
	\bigskip

	\textbf{System}

	\begin{tabular}[]{| >{\raggedright \bfseries \columncolor{Jaunepythoninfo}}m{3cm} | >{\raggedleft}m{7cm} |}
%	\begin{tabular}[]{|>{ \bfseries \columncolor{Jaunepythoninfo}}l| >{ \small}m{5cm}<{.}|}
\hline Python version & 3.5.3\tabularnewline
\hline OS Version & Linux 4.9.59-v7+ (Debian 9.3) \tabularnewline
\hline Executable & /usr/bin/python3 \tabularnewline
\hline Build Date & Jan 19 2017 14:11:04 \tabularnewline
\hline Compiler & GCC 6.3.0 20170124 \tabularnewline
\hline Python API & 1013 \tabularnewline
\hline
		\end{tabular} 
	\end{center}
\end{minipage}}
\end{center}
\normalsize

\item Configurez \textbf{\textit{ligthttpd}} pour notre site internet \textit{\textbf{pythoninfo}}.

Créer le fichier \textit{\textbf{\string~/public\string_html/pythoninfo/pythoninfo.fcgi}}

\lstset{title={},caption={}, style=PYTHON}
\begin{lstlisting}
#!/usr/bin/python3
# -*- coding: utf-8 -*-

from flup.server.fcgi import WSGIServer
from pythoninfo import app

if __name__ == '__main__':
    WSGIServer(app).run()
\end{lstlisting}

Rendez ce fichier exécutable.

\lstset{title={},caption={}, style=bash}
\begin{lstlisting}
chmod +x ~/public_html/pythoninfo/pythoninfo.fcgi
\end{lstlisting}


Ouvrez avec les droits root le fichier   \textit{\textbf{/etc/lighttpd/conf-enabled/10-fastcgi.conf}} 

\lstset{title={},caption={}, style=bash}
\begin{lstlisting}
sudo editor /etc/lighttpd/conf-enabled/10-fastcgi.conf
\end{lstlisting}

Ajouter au bas du fichier

\lstset{title={},caption={}, style=PYTHON}
\begin{lstlisting}
# http//localhost/~pi/pythoninfo
fastcgi.server += ( "/~pi/pythoninfo" =>
    ((
        "socket" => "/tmp/pythoninfo-user-fcgi.sock",
        "bin-path" => "/home/pi/public_html/pythoninfo/pythoninfo.fcgi",
        "check-local" => "disable",
        "max-procs"  => 1
    ))
)

url.rewrite-once += (
    "^/~pi/pythoninfo$" => "/~pi/pythoninfo/"
)
\end{lstlisting}


vérifiez que la syntaxe soit correcte avec la commande suivante:


\lstset{title={},caption={}, style=bash}
\begin{lstlisting}
lighttpd -t -f /etc/lighttpd/lighttpd.conf
\end{lstlisting}


Il faut relancer le servide lighttpd.  

\lstset{title={},caption={}, style=bash}
\begin{lstlisting}
sudo service lighttpd restart
\end{lstlisting}

Il ne reste plus qu'à vérifier. Ouvrez votre navigateur à l'adresse \url{http://localhost/~pi/pythoninfo/}  ou à l'adresse \url{http://localhost/~pi/pythoninfo/pyinfo}
\end{list}


\section{Un exemple de formulaire avec flask}

\subsection{Formulaire avec redirection sur une deuxième page}


Vous pouvez télécharger directement les fichiers \url{http://megamaths.hd.free.fr/~pi/statics/testflask.zip}

%Voici un exemple de  formulaire avec une redirection vers une deuxième page.\\


\begin{list}{\ding{212}}{}
\item On créer les dossiers puis on télécharge deux images pour le site.

\lstset{title={},caption={}, style=bash}
\begin{lstlisting}
mkdir -p   ~/public_html/testflask/templates
mkdir   ~/public_html/testflask/static
cd ~/public_html/testflask/static/
wget http://megamaths.hd.free.fr/~pi/statics/favicon.ico
wget http://megamaths.hd.free.fr/~pi/statics/fleur.png
\end{lstlisting}

\item Créer le fichier  \textit{\textbf{\string~/public\string_html/testflask/testflask.py}}

\lstset{title={},caption={}, style=PYTHON}
\begin{lstlisting}
#!/usr/bin/python3
# -*- coding: utf-8 -*-
from flask import Flask, render_template, url_for, request
app = Flask(__name__)   # Initialise l'application Flask

@app.route('/') # C'est un décorateur, on donne la route ici "/"  l'adresse sera donc localhost:5000/
def page1():
    return render_template("formulaire1_page1.html", titre="Formulaire 1")


@app.route('/hello', methods=['POST'])  
def hello():
    Nom=request.form['nom']
    Prenom=request.form['prenom']
    return render_template('formulaire1_page2.html' ,titre="Page 2", nom=Nom, prenom=Prenom)

if __name__ == '__main__':
    app.run(host='0.0.0.0',debug=True)
\end{lstlisting}



\item Créer le fichier  \textit{\textbf{\string~/public\string_html/testflask/formulaire1\string_page1.html}}

\lstset{title={},caption={}, language=HTML}
\begin{lstlisting}
<html>
<head>
<meta charset="utf-8" />
<title>{{ titre }}</title>
<!-- On importe notre frichier css -->
<link href="{{ url_for('static', filename='mon_style.css') }}" rel="stylesheet" type="text/css" />
<!-- Les images -->
<link rel="shortcut icon" href="{{ url_for('static', filename='favicon.ico') }}">
</head>

<body>
<header><h1>{{ titre }} <img src="{{ url_for('static',filename ='fleur.png') }}" alt="fleur" title="fleur" border="0"></h1></header>
<section>
<div>
    <form method="post" action="{{ url_for('hello') }}">
    <label for="nom">Entrez votre nom:</label>
    <input type="text" name="nom" /><br />
    <label for="prenom">Entrez votre prénom:</label>
    <input type="text" name="prenom" /><br />
    <input type="submit" type="submit" value="Envoyer" />
    </form>
</div>
</section>
</body>
</html>
\end{lstlisting}


\item Créer le fichier  \textit{\textbf{\string~/public\string_html/testflask/formulaire1\string_page2.html}}

\lstset{title={},caption={}, language=HTML}
\begin{lstlisting}
<html>
<head>
<meta charset="utf-8" />
<title>{{ titre }}</title>
<!-- On importe notre frichier css -->
<link href="{{ url_for('static', filename='mon_style.css') }}" rel="stylesheet" type="text/css" />
<!-- Les images -->
<link rel="shortcut icon" href="{{ url_for('static', filename='favicon.ico') }}">
</head>

<body>
<header class="site-header"><h1>{{ titre }} <img src="{{ url_for('static',filename ='fleur.png') }}" alt="fleur" title="fleur" border="0"></h1></header>
<section>
<div
<p>bonjour <b><em>{{ prenom }} {{ nom }}</em></b></p>
<p><a href="{{ url_for('page1') }}" >Lien pour revenir sur la 1° page</a></p>  
<!-- Attention le lien redirige sur la fonction python, pas sur le template html -->
</div>
</section>
</body>
</html>
\end{lstlisting}

\item N'oublions pas le fichier \textit{\textbf{public\string_html/testflask/static/mon\string_style.css}}


\lstset{title={},caption={}, language=css}
\begin{lstlisting}
h1 {
    color: red;
}
\end{lstlisting}

\item Éxécuter le fichier python

\lstset{title={},caption={}, style=bash}
\begin{lstlisting}
python3 ~/public_html/testflask.py
\end{lstlisting}

Lancer votre navigateur à l'adresse \url{http://localhost:5000}

\begin{center}
	\raisebox{-0.1cm}{\includegraphics[height=5cm]{images/formulaire1a}}\\
	\raisebox{-0.1cm}{\includegraphics[height=5cm]{images/formulaire1b}}\\
\end{center}
\item Configurer \textit{\textbf{ligthttpd}} pour notre site internet \textit{\textbf{testflask}}.

Créer le fichier \textit{\textbf{\string~/public\string_html/testflask/testflask.fcgi}}

\lstset{title={},caption={}, style=PYTHON}
\begin{lstlisting}
#!/usr/bin/python3
# -*- coding: utf-8 -*-
from flup.server.fcgi import WSGIServer
from testflask import app

if __name__ == '__main__':
    WSGIServer(app).run()
\end{lstlisting}

Rendez ce fichier exécutable.

\lstset{title={},caption={}, style=bash}
\begin{lstlisting}
chmod +x ~/public_html/pythoninfo/testflask.fcgi
\end{lstlisting}


Ouvrez avec les droits root le fichier   \textit{/etc/lighttpd/conf-enabled/10-fastcgi.conf} 

\lstset{title={},caption={}, style=bash}
\begin{lstlisting}
sudo editor /etc/lighttpd/conf-enabled/10-fastcgi.conf
\end{lstlisting}

Ajouter au bas du fichier

\lstset{title={},caption={}, style=PYTHON}
\begin{lstlisting}
# http//localhost/~pi/testflask
fastcgi.server += ( "/~pi/testflask" =>
    ((
        "socket" => "/tmp/testflask-user-fcgi.sock",
        "bin-path" => "/home/pi/public_html/testflask/testflask.fcgi",
        "check-local" => "disable",
        "max-procs"  => 1
    ))
)

url.rewrite-once += (
    "^/~pi/testflask$" => "/~pi/testflask/"
)
\end{lstlisting}

Vérifier la syntaxe et relancer le servide lighttpd.  

\lstset{title={},caption={}, style=bash}
\begin{lstlisting}
lighttpd -t -f /etc/lighttpd/lighttpd.conf
sudo service lighttpd restart
\end{lstlisting}

Il ne reste plus qu'à vérifier. Ouvrez votre navigateur à l'adresse \url{http://localhost/~pi/testflask}  ou à l'adresse \url{http://votrenomdedomaine/~pi/testflask}


\end{list}



\subsection{Formulaire avec redirection sur la même page}



\begin{list}{\ding{212}}{}
\item Sauvegardons notre fichier python.
\lstset{title={},caption={}, style=bash}
\begin{lstlisting}
cp ~/public_html/testflask/testflask.py ~/public_html/testflask/testflask.old.py
\end{lstlisting}

\item Modifier le fichier \textit{\textbf{\string~/public\string_html/testflask/testflask.py}}


\lstset{title={},caption={}, style=PYTHON}
\begin{lstlisting}
#!/usr/bin/python3
# -*- coding: utf-8 -*-
from flask import Flask, render_template, url_for, request
app = Flask(__name__)   # Initialise l'application Flask

@app.route('/', methods=['GET','POST'])  # On doit indiquer que l'on utilise les deux méthodes
def page1():
    try:
        nom=request.form['nom']
    except:
        nom=''
    try:
        prenom=request.form['prenom']
    except:
        prenom=''
    methode=format(request.method)
    return render_template("formulaire2.html", titre='Formulaire 2',methode=methode,nom=nom,prenom=prenom)

if __name__ == '__main__':
    app.run(host='0.0.0.0',debug=True)

\end{lstlisting}

\item Créer le fichier  \textit{\textbf{\string~/public\string_html/testflask/formulaire2.html}}

\lstset{title={},caption={}, language=HTML}
\begin{lstlisting}
<html>
<head>
<meta charset="utf-8" />
<title>{{ titre }}</title>
<!-- On importe notre frichier css -->
<link href="{{ url_for('static', filename='mon_style.css') }}" rel="stylesheet" type="text/css" />
<!-- Les images -->
<link rel="shortcut icon" href="{{ url_for('static', filename='favicon.ico') }}">
</head>

<body>
<header><h1>{{ titre }} <img src="{{ url_for('static',filename ='fleur.png') }}" alt="fleur" title="fleur" border="0"></h1></header>
<section>
	<p>Cette page a été obtenue avec la méthode  <b><em>{{ methode }}</em></b>.</p>

<div>
    <form method="post" action="{{ url_for('page1') }}">
    <label for="nom">Entrez votre nom:</label>
    <input type="text" name="nom" /><br />
    <label for="prenom">Entrez votre prénom:</label>
    <input type="text" name="prenom" /><br />
    <input type="submit" type="submit" value="Envoyer" />
    </form>
</div>

    
    <p>bonjour <b><em>{{ prenom }} {{ nom }}</em></b></p>
    

</section>
</body>
</html>
\end{lstlisting}

\item Relancer \textbf{\textit{lighttpd}}

\lstset{title={},caption={}, style=bash}
\begin{lstlisting}
sudo service lighttpd restart
\end{lstlisting}
\item Ouvrez votre navigateur à l'adresse \url{http://votrenomdedomaine/~pi/testflask}
\end{list}

\end{document}
