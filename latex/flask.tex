
% Default to the notebook output style

    


% Inherit from the specified cell style.




    
\documentclass{article}

    
    
    \usepackage{graphicx} % Used to insert images
    \usepackage{adjustbox} % Used to constrain images to a maximum size 
    \usepackage{color} % Allow colors to be defined
    \usepackage{enumerate} % Needed for markdown enumerations to work
    \usepackage{geometry} % Used to adjust the document margins
    \usepackage{amsmath} % Equations
    \usepackage{amssymb} % Equations
    \usepackage[mathletters]{ucs} % Extended unicode (utf-8) support
    \usepackage[utf8x]{inputenc} % Allow utf-8 characters in the tex document
    \usepackage{fancyvrb} % verbatim replacement that allows latex
    \usepackage{grffile} % extends the file name processing of package graphics 
                         % to support a larger range 
    % The hyperref package gives us a pdf with properly built
    % internal navigation ('pdf bookmarks' for the table of contents,
    % internal cross-reference links, web links for URLs, etc.)
    \usepackage{hyperref}
    \usepackage{longtable} % longtable support required by pandoc >1.10
    \usepackage{booktabs}  % table support for pandoc > 1.12.2
    

    
    
    \definecolor{orange}{cmyk}{0,0.4,0.8,0.2}
    \definecolor{darkorange}{rgb}{.71,0.21,0.01}
    \definecolor{darkgreen}{rgb}{.12,.54,.11}
    \definecolor{myteal}{rgb}{.26, .44, .56}
    \definecolor{gray}{gray}{0.45}
    \definecolor{lightgray}{gray}{.95}
    \definecolor{mediumgray}{gray}{.8}
    \definecolor{inputbackground}{rgb}{.95, .95, .85}
    \definecolor{outputbackground}{rgb}{.95, .95, .95}
    \definecolor{traceback}{rgb}{1, .95, .95}
    % ansi colors
    \definecolor{red}{rgb}{.6,0,0}
    \definecolor{green}{rgb}{0,.65,0}
    \definecolor{brown}{rgb}{0.6,0.6,0}
    \definecolor{blue}{rgb}{0,.145,.698}
    \definecolor{purple}{rgb}{.698,.145,.698}
    \definecolor{cyan}{rgb}{0,.698,.698}
    \definecolor{lightgray}{gray}{0.5}
    
    % bright ansi colors
    \definecolor{darkgray}{gray}{0.25}
    \definecolor{lightred}{rgb}{1.0,0.39,0.28}
    \definecolor{lightgreen}{rgb}{0.48,0.99,0.0}
    \definecolor{lightblue}{rgb}{0.53,0.81,0.92}
    \definecolor{lightpurple}{rgb}{0.87,0.63,0.87}
    \definecolor{lightcyan}{rgb}{0.5,1.0,0.83}
    
    % commands and environments needed by pandoc snippets
    % extracted from the output of `pandoc -s`
    \DefineVerbatimEnvironment{Highlighting}{Verbatim}{commandchars=\\\{\}}
    % Add ',fontsize=\small' for more characters per line
    \newenvironment{Shaded}{}{}
    \newcommand{\KeywordTok}[1]{\textcolor[rgb]{0.00,0.44,0.13}{\textbf{{#1}}}}
    \newcommand{\DataTypeTok}[1]{\textcolor[rgb]{0.56,0.13,0.00}{{#1}}}
    \newcommand{\DecValTok}[1]{\textcolor[rgb]{0.25,0.63,0.44}{{#1}}}
    \newcommand{\BaseNTok}[1]{\textcolor[rgb]{0.25,0.63,0.44}{{#1}}}
    \newcommand{\FloatTok}[1]{\textcolor[rgb]{0.25,0.63,0.44}{{#1}}}
    \newcommand{\CharTok}[1]{\textcolor[rgb]{0.25,0.44,0.63}{{#1}}}
    \newcommand{\StringTok}[1]{\textcolor[rgb]{0.25,0.44,0.63}{{#1}}}
    \newcommand{\CommentTok}[1]{\textcolor[rgb]{0.38,0.63,0.69}{\textit{{#1}}}}
    \newcommand{\OtherTok}[1]{\textcolor[rgb]{0.00,0.44,0.13}{{#1}}}
    \newcommand{\AlertTok}[1]{\textcolor[rgb]{1.00,0.00,0.00}{\textbf{{#1}}}}
    \newcommand{\FunctionTok}[1]{\textcolor[rgb]{0.02,0.16,0.49}{{#1}}}
    \newcommand{\RegionMarkerTok}[1]{{#1}}
    \newcommand{\ErrorTok}[1]{\textcolor[rgb]{1.00,0.00,0.00}{\textbf{{#1}}}}
    \newcommand{\NormalTok}[1]{{#1}}
    
    % Define a nice break command that doesn't care if a line doesn't already
    % exist.
    \def\br{\hspace*{\fill} \\* }
    % Math Jax compatability definitions
    \def\gt{>}
    \def\lt{<}
    % Document parameters
    \title{flask}
    
    
    

    % Pygments definitions
    
\makeatletter
\def\PY@reset{\let\PY@it=\relax \let\PY@bf=\relax%
    \let\PY@ul=\relax \let\PY@tc=\relax%
    \let\PY@bc=\relax \let\PY@ff=\relax}
\def\PY@tok#1{\csname PY@tok@#1\endcsname}
\def\PY@toks#1+{\ifx\relax#1\empty\else%
    \PY@tok{#1}\expandafter\PY@toks\fi}
\def\PY@do#1{\PY@bc{\PY@tc{\PY@ul{%
    \PY@it{\PY@bf{\PY@ff{#1}}}}}}}
\def\PY#1#2{\PY@reset\PY@toks#1+\relax+\PY@do{#2}}

\expandafter\def\csname PY@tok@bp\endcsname{\def\PY@tc##1{\textcolor[rgb]{0.00,0.50,0.00}{##1}}}
\expandafter\def\csname PY@tok@c1\endcsname{\let\PY@it=\textit\def\PY@tc##1{\textcolor[rgb]{0.25,0.50,0.50}{##1}}}
\expandafter\def\csname PY@tok@ni\endcsname{\let\PY@bf=\textbf\def\PY@tc##1{\textcolor[rgb]{0.60,0.60,0.60}{##1}}}
\expandafter\def\csname PY@tok@ss\endcsname{\def\PY@tc##1{\textcolor[rgb]{0.10,0.09,0.49}{##1}}}
\expandafter\def\csname PY@tok@il\endcsname{\def\PY@tc##1{\textcolor[rgb]{0.40,0.40,0.40}{##1}}}
\expandafter\def\csname PY@tok@kd\endcsname{\let\PY@bf=\textbf\def\PY@tc##1{\textcolor[rgb]{0.00,0.50,0.00}{##1}}}
\expandafter\def\csname PY@tok@s2\endcsname{\def\PY@tc##1{\textcolor[rgb]{0.73,0.13,0.13}{##1}}}
\expandafter\def\csname PY@tok@mi\endcsname{\def\PY@tc##1{\textcolor[rgb]{0.40,0.40,0.40}{##1}}}
\expandafter\def\csname PY@tok@gd\endcsname{\def\PY@tc##1{\textcolor[rgb]{0.63,0.00,0.00}{##1}}}
\expandafter\def\csname PY@tok@err\endcsname{\def\PY@bc##1{\setlength{\fboxsep}{0pt}\fcolorbox[rgb]{1.00,0.00,0.00}{1,1,1}{\strut ##1}}}
\expandafter\def\csname PY@tok@kr\endcsname{\let\PY@bf=\textbf\def\PY@tc##1{\textcolor[rgb]{0.00,0.50,0.00}{##1}}}
\expandafter\def\csname PY@tok@gi\endcsname{\def\PY@tc##1{\textcolor[rgb]{0.00,0.63,0.00}{##1}}}
\expandafter\def\csname PY@tok@gh\endcsname{\let\PY@bf=\textbf\def\PY@tc##1{\textcolor[rgb]{0.00,0.00,0.50}{##1}}}
\expandafter\def\csname PY@tok@vg\endcsname{\def\PY@tc##1{\textcolor[rgb]{0.10,0.09,0.49}{##1}}}
\expandafter\def\csname PY@tok@sh\endcsname{\def\PY@tc##1{\textcolor[rgb]{0.73,0.13,0.13}{##1}}}
\expandafter\def\csname PY@tok@mb\endcsname{\def\PY@tc##1{\textcolor[rgb]{0.40,0.40,0.40}{##1}}}
\expandafter\def\csname PY@tok@m\endcsname{\def\PY@tc##1{\textcolor[rgb]{0.40,0.40,0.40}{##1}}}
\expandafter\def\csname PY@tok@mo\endcsname{\def\PY@tc##1{\textcolor[rgb]{0.40,0.40,0.40}{##1}}}
\expandafter\def\csname PY@tok@sd\endcsname{\let\PY@it=\textit\def\PY@tc##1{\textcolor[rgb]{0.73,0.13,0.13}{##1}}}
\expandafter\def\csname PY@tok@nd\endcsname{\def\PY@tc##1{\textcolor[rgb]{0.67,0.13,1.00}{##1}}}
\expandafter\def\csname PY@tok@ge\endcsname{\let\PY@it=\textit}
\expandafter\def\csname PY@tok@nl\endcsname{\def\PY@tc##1{\textcolor[rgb]{0.63,0.63,0.00}{##1}}}
\expandafter\def\csname PY@tok@gu\endcsname{\let\PY@bf=\textbf\def\PY@tc##1{\textcolor[rgb]{0.50,0.00,0.50}{##1}}}
\expandafter\def\csname PY@tok@ne\endcsname{\let\PY@bf=\textbf\def\PY@tc##1{\textcolor[rgb]{0.82,0.25,0.23}{##1}}}
\expandafter\def\csname PY@tok@o\endcsname{\def\PY@tc##1{\textcolor[rgb]{0.40,0.40,0.40}{##1}}}
\expandafter\def\csname PY@tok@nn\endcsname{\let\PY@bf=\textbf\def\PY@tc##1{\textcolor[rgb]{0.00,0.00,1.00}{##1}}}
\expandafter\def\csname PY@tok@mf\endcsname{\def\PY@tc##1{\textcolor[rgb]{0.40,0.40,0.40}{##1}}}
\expandafter\def\csname PY@tok@kn\endcsname{\let\PY@bf=\textbf\def\PY@tc##1{\textcolor[rgb]{0.00,0.50,0.00}{##1}}}
\expandafter\def\csname PY@tok@gs\endcsname{\let\PY@bf=\textbf}
\expandafter\def\csname PY@tok@ow\endcsname{\let\PY@bf=\textbf\def\PY@tc##1{\textcolor[rgb]{0.67,0.13,1.00}{##1}}}
\expandafter\def\csname PY@tok@gt\endcsname{\def\PY@tc##1{\textcolor[rgb]{0.00,0.27,0.87}{##1}}}
\expandafter\def\csname PY@tok@si\endcsname{\let\PY@bf=\textbf\def\PY@tc##1{\textcolor[rgb]{0.73,0.40,0.53}{##1}}}
\expandafter\def\csname PY@tok@sr\endcsname{\def\PY@tc##1{\textcolor[rgb]{0.73,0.40,0.53}{##1}}}
\expandafter\def\csname PY@tok@nf\endcsname{\def\PY@tc##1{\textcolor[rgb]{0.00,0.00,1.00}{##1}}}
\expandafter\def\csname PY@tok@c\endcsname{\let\PY@it=\textit\def\PY@tc##1{\textcolor[rgb]{0.25,0.50,0.50}{##1}}}
\expandafter\def\csname PY@tok@nv\endcsname{\def\PY@tc##1{\textcolor[rgb]{0.10,0.09,0.49}{##1}}}
\expandafter\def\csname PY@tok@sx\endcsname{\def\PY@tc##1{\textcolor[rgb]{0.00,0.50,0.00}{##1}}}
\expandafter\def\csname PY@tok@cm\endcsname{\let\PY@it=\textit\def\PY@tc##1{\textcolor[rgb]{0.25,0.50,0.50}{##1}}}
\expandafter\def\csname PY@tok@cs\endcsname{\let\PY@it=\textit\def\PY@tc##1{\textcolor[rgb]{0.25,0.50,0.50}{##1}}}
\expandafter\def\csname PY@tok@kp\endcsname{\def\PY@tc##1{\textcolor[rgb]{0.00,0.50,0.00}{##1}}}
\expandafter\def\csname PY@tok@go\endcsname{\def\PY@tc##1{\textcolor[rgb]{0.53,0.53,0.53}{##1}}}
\expandafter\def\csname PY@tok@cp\endcsname{\def\PY@tc##1{\textcolor[rgb]{0.74,0.48,0.00}{##1}}}
\expandafter\def\csname PY@tok@k\endcsname{\let\PY@bf=\textbf\def\PY@tc##1{\textcolor[rgb]{0.00,0.50,0.00}{##1}}}
\expandafter\def\csname PY@tok@nb\endcsname{\def\PY@tc##1{\textcolor[rgb]{0.00,0.50,0.00}{##1}}}
\expandafter\def\csname PY@tok@sc\endcsname{\def\PY@tc##1{\textcolor[rgb]{0.73,0.13,0.13}{##1}}}
\expandafter\def\csname PY@tok@w\endcsname{\def\PY@tc##1{\textcolor[rgb]{0.73,0.73,0.73}{##1}}}
\expandafter\def\csname PY@tok@s\endcsname{\def\PY@tc##1{\textcolor[rgb]{0.73,0.13,0.13}{##1}}}
\expandafter\def\csname PY@tok@na\endcsname{\def\PY@tc##1{\textcolor[rgb]{0.49,0.56,0.16}{##1}}}
\expandafter\def\csname PY@tok@gr\endcsname{\def\PY@tc##1{\textcolor[rgb]{1.00,0.00,0.00}{##1}}}
\expandafter\def\csname PY@tok@vi\endcsname{\def\PY@tc##1{\textcolor[rgb]{0.10,0.09,0.49}{##1}}}
\expandafter\def\csname PY@tok@vc\endcsname{\def\PY@tc##1{\textcolor[rgb]{0.10,0.09,0.49}{##1}}}
\expandafter\def\csname PY@tok@sb\endcsname{\def\PY@tc##1{\textcolor[rgb]{0.73,0.13,0.13}{##1}}}
\expandafter\def\csname PY@tok@nt\endcsname{\let\PY@bf=\textbf\def\PY@tc##1{\textcolor[rgb]{0.00,0.50,0.00}{##1}}}
\expandafter\def\csname PY@tok@kc\endcsname{\let\PY@bf=\textbf\def\PY@tc##1{\textcolor[rgb]{0.00,0.50,0.00}{##1}}}
\expandafter\def\csname PY@tok@gp\endcsname{\let\PY@bf=\textbf\def\PY@tc##1{\textcolor[rgb]{0.00,0.00,0.50}{##1}}}
\expandafter\def\csname PY@tok@nc\endcsname{\let\PY@bf=\textbf\def\PY@tc##1{\textcolor[rgb]{0.00,0.00,1.00}{##1}}}
\expandafter\def\csname PY@tok@s1\endcsname{\def\PY@tc##1{\textcolor[rgb]{0.73,0.13,0.13}{##1}}}
\expandafter\def\csname PY@tok@kt\endcsname{\def\PY@tc##1{\textcolor[rgb]{0.69,0.00,0.25}{##1}}}
\expandafter\def\csname PY@tok@no\endcsname{\def\PY@tc##1{\textcolor[rgb]{0.53,0.00,0.00}{##1}}}
\expandafter\def\csname PY@tok@mh\endcsname{\def\PY@tc##1{\textcolor[rgb]{0.40,0.40,0.40}{##1}}}
\expandafter\def\csname PY@tok@se\endcsname{\let\PY@bf=\textbf\def\PY@tc##1{\textcolor[rgb]{0.73,0.40,0.13}{##1}}}

\def\PYZbs{\char`\\}
\def\PYZus{\char`\_}
\def\PYZob{\char`\{}
\def\PYZcb{\char`\}}
\def\PYZca{\char`\^}
\def\PYZam{\char`\&}
\def\PYZlt{\char`\<}
\def\PYZgt{\char`\>}
\def\PYZsh{\char`\#}
\def\PYZpc{\char`\%}
\def\PYZdl{\char`\$}
\def\PYZhy{\char`\-}
\def\PYZsq{\char`\'}
\def\PYZdq{\char`\"}
\def\PYZti{\char`\~}
% for compatibility with earlier versions
\def\PYZat{@}
\def\PYZlb{[}
\def\PYZrb{]}
\makeatother


    % Exact colors from NB
    \definecolor{incolor}{rgb}{0.0, 0.0, 0.5}
    \definecolor{outcolor}{rgb}{0.545, 0.0, 0.0}



    
    % Prevent overflowing lines due to hard-to-break entities
    \sloppy 
    % Setup hyperref package
    \hypersetup{
      breaklinks=true,  % so long urls are correctly broken across lines
      colorlinks=true,
      urlcolor=blue,
      linkcolor=darkorange,
      citecolor=darkgreen,
      }
    % Slightly bigger margins than the latex defaults
    
    \geometry{verbose,tmargin=1in,bmargin=1in,lmargin=1in,rmargin=1in}
    
    

    \begin{document}
    
    
    \maketitle
    
    

    
    \begin{Verbatim}[commandchars=\\\{\}]
{\color{incolor}In [{\color{incolor}5}]:} \PY{o}{\PYZpc{}\PYZpc{}}\PY{k}{javascript}
        \PY{n}{IPython}\PY{o}{.}\PY{n}{Cell}\PY{o}{.}\PY{n}{options\PYZus{}default}\PY{o}{.}\PY{n}{cm\PYZus{}config}\PY{o}{.}\PY{n}{lineNumbers} \PY{o}{=} \PY{n}{false}\PY{p}{;}
\end{Verbatim}

    
    \begin{verbatim}
<IPython.core.display.Javascript at 0x7fa0a8036f98>
    \end{verbatim}

    
    \begin{Verbatim}[commandchars=\\\{\}]
{\color{incolor}In [{\color{incolor}12}]:} \PY{c}{\PYZsh{} Charge ma feuille de style pour nbviewer}
         \PY{k+kn}{from} \PY{n+nn}{IPython} \PY{k+kn}{import} \PY{n}{utils}
         \PY{k+kn}{from} \PY{n+nn}{IPython.core.display} \PY{k+kn}{import} \PY{n}{HTML}
         \PY{k+kn}{import} \PY{n+nn}{os}
         \PY{k}{def} \PY{n+nf}{css\PYZus{}styling}\PY{p}{(}\PY{p}{)}\PY{p}{:}
             \PY{n}{base} \PY{o}{=} \PY{n}{utils}\PY{o}{.}\PY{n}{path}\PY{o}{.}\PY{n}{get\PYZus{}ipython\PYZus{}dir}\PY{p}{(}\PY{p}{)}
             \PY{n}{styles} \PY{o}{=} \PY{l+s}{\PYZdq{}}\PY{l+s}{\PYZlt{}style\PYZgt{}}\PY{l+s+se}{\PYZbs{}n}\PY{l+s+si}{\PYZpc{}s}\PY{l+s+se}{\PYZbs{}n}\PY{l+s}{\PYZlt{}/style\PYZgt{}}\PY{l+s}{\PYZdq{}} \PY{o}{\PYZpc{}} \PY{p}{(}\PY{n+nb}{open}\PY{p}{(}\PY{l+s}{\PYZsq{}}\PY{l+s}{./documents/custom.css}\PY{l+s}{\PYZsq{}}\PY{p}{,}\PY{l+s}{\PYZsq{}}\PY{l+s}{r}\PY{l+s}{\PYZsq{}}\PY{p}{)}\PY{o}{.}\PY{n}{read}\PY{p}{(}\PY{p}{)}\PY{p}{)}
             \PY{k}{return} \PY{n}{HTML}\PY{p}{(}\PY{n}{styles}\PY{p}{)}
         \PY{n}{css\PYZus{}styling}\PY{p}{(}\PY{p}{)}
\end{Verbatim}

            \begin{Verbatim}[commandchars=\\\{\}]
{\color{outcolor}Out[{\color{outcolor}12}]:} <IPython.core.display.HTML at 0x7fa0a8044e10>
\end{Verbatim}
        

    \section{IV Site internet en python avec flask~}


    Classiquement sur internet on utilise un serveur \textbf{\emph{lamp}}
:\\\textbf{l}inux (os), \textbf{a}pache (serveur), \textbf{m}ysql (base
de données), \textbf{p}hp (language de programmation pour avoir des
pages dynamiques).

Il existe de nombreux outils pour le web écrit en python: \emph{serveur
Web} (Zope, gunicorn) ; \emph{script cgi}; \emph{frameworks Web} (Flask,
Django, cherrypy etc \ldots{},

nous utiliserons le framework \textbf{\emph{flask}}. Pour se documenter
je vous conseille deux sites.\\- \url{http://flask.pocoo.org/docs} -
\url{http://openclassrooms.com/courses/creez-vos-applications-web-avec-flask}
-
\url{http://pub.phyks.me/sdz/sdz/creez-vos-applications-web-avec-flask.html}

Nous avons besoin de quelques fichiers pour commencer.

\begin{itemize}
\itemsep1pt\parskip0pt\parsep0pt
\item
  Télécharger le fichier
  \url{https://github.com/debimax/cours-debimax/raw/master/documents/isn-flask.tar.gz}.
\item
  Décompressez le (en console : tar zxvf isn-flask.tar.gz).
\end{itemize}

    \subsection{1) Flask et les templates}\label{flask-et-les-templates}

\subsubsection{a) Le principe}\label{a-le-principe}

Avec \emph{Flask} on pourrait mettre le code dans un seul
\emph{fichier.py} mais on utilise de préférence des \emph{templates}
(avec \emph{jinja2}).

L'arborescence d'un projet Flask sera :

\begin{verbatim}
   projet/script.py
   projet/static/
   projet/templates
   projet/templates/les_templates.html
\end{verbatim}

Les templates se mettent dans le dossier \textbf{\emph{templates/}}. Le
dossier \textbf{\emph{static/}} contiendra lui toutes les images,
fichiers css, js \ldots{}

Ouvrez avec \emph{geany} le fichier \textbf{\emph{exemple.py}} :

\begin{Shaded}
\begin{Highlighting}[]
\DecValTok{01} \CommentTok{#!/usr/bin/python3}
\DecValTok{02} \CommentTok{# -*- coding: utf-8 -*-}
\DecValTok{03} \CharTok{from} \NormalTok{flask }\CharTok{import} \NormalTok{Flask, render_template, url_for}
\DecValTok{04} \NormalTok{app = Flask(}\OtherTok{__name__}\NormalTok{)   }\CommentTok{# Initialise l'application Flask}
\DecValTok{05}
\NormalTok{06 @app.route(}\StringTok{'/'}\NormalTok{)  }\CommentTok{# C'est un décorateur, on donne la route ici "/"  l'adresse sera donc localhost:5000/}
\DecValTok{07} \KeywordTok{def} \NormalTok{accueil():}
\DecValTok{08}     \NormalTok{lignes=[}\StringTok{'ligne \{\}'}\NormalTok{.}\DataTypeTok{format}\NormalTok{(i) }\KeywordTok{for} \NormalTok{i in }\DataTypeTok{range}\NormalTok{(}\DecValTok{1}\NormalTok{,}\DecValTok{10}\NormalTok{)] }\CommentTok{# Que fait cette ligne?}
\DecValTok{09}     \KeywordTok{return} \NormalTok{render_template(}\StringTok{"accueil.html"}\NormalTok{, titre=}\StringTok{"Bienvenue !"}\NormalTok{,lignes=lignes) }\CommentTok{# On utilise le template accueil.html, avec les variables titre et lignes}
\DecValTok{10}
\DecValTok{11} \KeywordTok{if} \OtherTok{__name__} \NormalTok{== __main__ :}
\DecValTok{12}     \NormalTok{app.run(debug=}\OtherTok{True}\NormalTok{)}
\end{Highlighting}
\end{Shaded}

Dans le code ci-dessous, quel est le contenu de la variable
\textbf{\emph{lignes}}?

\begin{Shaded}
\begin{Highlighting}[]
 \NormalTok{liges=[}\StringTok{'ligne\{\}'}\NormalTok{.}\DataTypeTok{format}\NormalTok{(i) }\KeywordTok{for} \NormalTok{i in }\DataTypeTok{range}\NormalTok{(}\DecValTok{1}\NormalTok{,}\DecValTok{10}\NormalTok{)]}
\end{Highlighting}
\end{Shaded}

\ldots{}\ldots{}\ldots{}\ldots{}\ldots{}\ldots{}\ldots{}\ldots{}\ldots{}\ldots{}\ldots{}\ldots{}\ldots{}\ldots{}\ldots{}\ldots{}\ldots{}\ldots{}\ldots{}\ldots{}\ldots{}\ldots{}\ldots{}\ldots{}\ldots{}\ldots{}\ldots{}\ldots{}\ldots{}\ldots{}\ldots{}\ldots{}\ldots{}\ldots{}\ldots{}\ldots{}\ldots{}

Ouvrez maintenant le fichier \textbf{\emph{templates/accueil.html}}.

\begin{Shaded}
\begin{Highlighting}[]
\NormalTok{01 }\DataTypeTok{<!DOCTYPE }\NormalTok{html}\DataTypeTok{>}
\NormalTok{02 }\KeywordTok{<html>}
\NormalTok{03 }\KeywordTok{<head>}
\NormalTok{04 }\KeywordTok{<meta}\OtherTok{ charset=}\StringTok{"utf-8"} \KeywordTok{/>}
\NormalTok{05 }\KeywordTok{<title>}\NormalTok{\{\{ titre \}\}}\KeywordTok{</title>}
\NormalTok{06 }\KeywordTok{</head>}
\NormalTok{07 }\KeywordTok{<body>}
\NormalTok{08 }\KeywordTok{<header><h1>}\NormalTok{\{\{ titre \}\}}\KeywordTok{</h1></header>}  \CommentTok{<!-- On affiche le titre -->} 
\NormalTok{09 }\KeywordTok{<section>}
\NormalTok{10 }\KeywordTok{<ul>}
\NormalTok{11 \{% for ligne in lignes %\}   }\CommentTok{<!-- Il existe avec jinja des boucles des conditions ... -->} 
\NormalTok{12 }\KeywordTok{<li>}\NormalTok{\{\{ ligne \}\}}\KeywordTok{</li>}        \CommentTok{<!-- On affiche le contenu de la liste ligne -->} 
\NormalTok{13 \{% endfor %\}}
\NormalTok{14 }\KeywordTok{</ul>}
\NormalTok{15 }\KeywordTok{</section>}
\NormalTok{16 }\KeywordTok{</body>}
\NormalTok{17 }\KeywordTok{</html>}
\end{Highlighting}
\end{Shaded}

Exécuter le fichier \textbf{\emph{exemple.py}} depuis \emph{geany}.
Ouvrez un navigateur internet à l'adresse \url{http://localhost:5000}
(celui qui n'est pas écrit pablo car il ne doit pas utiliser de proxy).

Le fichier \emph{templates/accueil.html} s'appelle un
\textbf{\emph{template}} et le logiciel qui gère ces templates est
\textbf{\emph{jinja}} (on dit aussi moteur de templates).

    \subsubsection{b) Ajouter la date}\label{b-ajouter-la-date}

Il faudra importer la librairie \textbf{\emph{time}} puis utiliser le
décorateur \textbf{\emph{@app.context\_processor}} pour passer l'heure à
tous les templates.

\begin{Shaded}
\begin{Highlighting}[]
\DecValTok{03} \CharTok{from} \NormalTok{flask }\CharTok{import} \NormalTok{Flask, render_template, url_for}
\DecValTok{04} \CharTok{import} \NormalTok{time           }\CommentTok{# pour afficher l'heure }
\DecValTok{05} \NormalTok{app = Flask(}\OtherTok{__name__}\NormalTok{) }\CommentTok{# Initialise l'application Flask}
\DecValTok{06} 
\NormalTok{07 @app.context_processor   }\CommentTok{# ceci pour envoyer à tous les templates les variables date et heure}
\DecValTok{08} \KeywordTok{def} \NormalTok{passer_date_heure():}
\DecValTok{09}     \NormalTok{date=time.strftime(}\StringTok{'}\OtherTok{%d}\StringTok{/%m/%Y'}\NormalTok{,time.localtime())}
\DecValTok{10}     \NormalTok{heure=time.strftime(}\StringTok{"%H:%M:%S"}\NormalTok{,time.localtime())}
\DecValTok{11}     \KeywordTok{return} \DataTypeTok{dict}\NormalTok{(date=date,heure=heure)}
\DecValTok{12} 
\DecValTok{13} \NormalTok{@app.route(}\StringTok{'/'}\NormalTok{)}
\end{Highlighting}
\end{Shaded}

Il faut maintenant modifier le template \textbf{\emph{accueil.html}}

\begin{Shaded}
\begin{Highlighting}[]
\NormalTok{15     }\KeywordTok{</section>}
\NormalTok{16    }\KeywordTok{<footer>}\NormalTok{Nous sommes le \{\{ date \}\} et il est \{\{ heure \}\} heures.}\KeywordTok{</footer>}
\NormalTok{17     }\KeywordTok{</body>}
\end{Highlighting}
\end{Shaded}

Il suffit d'enregistrer le fichier \textbf{\emph{exemple.py}} s'il n'y a
pas d'erreur pour que le site soit mis à jour automatiquement.\\Observez
le résulat puis rechargez plusieurs fois la page.

Question: Pourquoi est il peu judicieux d'utiliser le serveur python
pour afficher
l'heure?\\\ldots{}\ldots{}\ldots{}\ldots{}\ldots{}\ldots{}\ldots{}\ldots{}\ldots{}\ldots{}\ldots{}\ldots{}\ldots{}\ldots{}\ldots{}\ldots{}\ldots{}\ldots{}\ldots{}\ldots{}\ldots{}\ldots{}\ldots{}\ldots{}\ldots{}\ldots{}\ldots{}\ldots{}\ldots{}\ldots{}\ldots{}\ldots{}\ldots{}\ldots{}\ldots{}\ldots{}\ldots{}\\\ldots{}\ldots{}\ldots{}\ldots{}\ldots{}\ldots{}\ldots{}\ldots{}\ldots{}\ldots{}\ldots{}\ldots{}\ldots{}\ldots{}\ldots{}\ldots{}\ldots{}\ldots{}\ldots{}\ldots{}\ldots{}\ldots{}\ldots{}\ldots{}\ldots{}\ldots{}\ldots{}\ldots{}\ldots{}\ldots{}\ldots{}\ldots{}\ldots{}\ldots{}\ldots{}\ldots{}\ldots{}

    \subsubsection{c) Les fichiers css}\label{c-les-fichiers-css}

Nous allons utiliser un fichiers \emph{css} pour l'habillage
(présentation) de notre page web.

Modifions le fichier \textbf{\emph{templates/accueil.html}}.

\begin{Shaded}
\begin{Highlighting}[]
\NormalTok{05 }\KeywordTok{<title>}\NormalTok{\{\{ titre \}\}}\KeywordTok{</title>}
\NormalTok{06 }\KeywordTok{<link}\OtherTok{ href=}\StringTok{"\{\{ url_for('static',filename='mon_style.css')\}\}"}\OtherTok{ rel=}\StringTok{"stylesheet"}\OtherTok{ type=}\StringTok{"text/css"} \KeywordTok{/>}
\NormalTok{07 }\KeywordTok{</head>}
\end{Highlighting}
\end{Shaded}

\begin{Shaded}
\begin{Highlighting}[]
\NormalTok{17    }\KeywordTok{<footer>}\NormalTok{Nous somme le }\KeywordTok{<em}\OtherTok{ class=}\StringTok{"Rouge"}\KeywordTok{>}\NormalTok{\{\{ date \}\}}\KeywordTok{</em>} \NormalTok{et il est }\KeywordTok{<em}\OtherTok{ class=}\StringTok{"Rouge"}\KeywordTok{>}\NormalTok{\{\{ heure \}\}}\KeywordTok{</em>} \NormalTok{heures.}\KeywordTok{</footer>}
\end{Highlighting}
\end{Shaded}

Dans le fichier \textbf{\emph{static/mon\_style.css}} il y a ceci:

\begin{Shaded}
\begin{Highlighting}[]
\NormalTok{01 html, body }\KeywordTok{\{}
\ErrorTok{02}    \KeywordTok{height:} \DataTypeTok{100%}\KeywordTok{;}
\ErrorTok{03}    \KeywordTok{margin:} \DataTypeTok{0}\KeywordTok{;}
\ErrorTok{04}    \KeywordTok{padding:} \DataTypeTok{0}\KeywordTok{;\}}
\NormalTok{05}
\NormalTok{06 body }\KeywordTok{\{}
\ErrorTok{07} \KeywordTok{display:}\DataTypeTok{table}\KeywordTok{;}
\ErrorTok{08} \KeywordTok{width:}\DataTypeTok{100%}\KeywordTok{;\}}
\NormalTok{09 }
\NormalTok{10 header,section,footer }\KeywordTok{\{display:}\DataTypeTok{table-row}\KeywordTok{;\}}
\NormalTok{11 }
\NormalTok{12 section }\KeywordTok{\{height:}\DataTypeTok{100%}\KeywordTok{;\}}
\NormalTok{13 }
\NormalTok{14 h1 }\KeywordTok{\{}
\ErrorTok{15} \KeywordTok{color:} \DataTypeTok{#ff00ff}\KeywordTok{;} 
\ErrorTok{16} \KeywordTok{text-align:} \DataTypeTok{center}\KeywordTok{;}\ErrorTok{19} \KeywordTok{\}}
\NormalTok{17 }
\NormalTok{18 footer }\KeywordTok{\{}
\ErrorTok{19} \KeywordTok{width:} \DataTypeTok{100%}\KeywordTok{;}
\ErrorTok{20} \KeywordTok{height:} \DataTypeTok{20px}\KeywordTok{;}
\ErrorTok{21} \KeywordTok{background-color:} \DataTypeTok{#f5f5f5}\KeywordTok{;}
\ErrorTok{22} \KeywordTok{text-align:} \DataTypeTok{center}\KeywordTok{;\}}
\NormalTok{23 }
\NormalTok{24 em}\FloatTok{.Rouge} \KeywordTok{\{color:} \DataTypeTok{#ff0000}\KeywordTok{;\}}
\end{Highlighting}
\end{Shaded}

\begin{itemize}
\itemsep1pt\parskip0pt\parsep0pt
\item
  Je met le contenu de la balise

  \textless{}h1\textgreater{}

  au centre et en couleur.\\
\item
  Je place le `footer' au centre en bas de la page avec un fond de
  couleur.\\
\item
  Je met en rouge avec la classe \textbf{\emph{em.Rouge}}.
\item
  Le reste c'est pour placer les sections et avoir un footer au bas de
  la page.
\end{itemize}

    \subsubsection{d) Le javascript}\label{d-le-javascript}

Coté client (le navigateur) on utilise aussi des scripts dans le code
HTML des pages web comme javascript, jquery etc\ldots{}

Nous allons maintenant utiliser un fichier \textbf{\emph{jascript
(.js)}} pour afficher l'heure. Le code javascript sera effectué par le
navigateur, ce n'est donc pas le serveur qui exécute le code et donc pas
besoin de rafraichir la page internet pour que le code soit exécuté.\\Il
n'est pas question d'expliquer le code javascript mais vous pouvez
regarder à quoi cela ressemble dans le ficher
\textbf{\emph{static/mes\_scripts.js}}.

Comme pour le fichier \emph{.css} Il faut indiquer le fichier \emph{.js}
aux templates.

Modifions le fichier \textbf{\emph{templates/accueil.html}}.

\begin{Shaded}
\begin{Highlighting}[]
 \NormalTok{6 }\KeywordTok{<link}\OtherTok{ href=}\StringTok{" url_for('static', filename='mon_style.css') "}\OtherTok{ rel=}\StringTok{"stylesheet"}\OtherTok{ type=}\StringTok{"text/css"} \KeywordTok{/>}
 \NormalTok{7 }\KeywordTok{<script}\OtherTok{ type=}\StringTok{text/javascript}\OtherTok{ src=}\StringTok{"\{\{url_for('static', filename='mes_scripts.js') \}\}"}\KeywordTok{></script>}
 \NormalTok{8 }\KeywordTok{</head>}
\end{Highlighting}
\end{Shaded}

\begin{Shaded}
\begin{Highlighting}[]
\NormalTok{17 }\KeywordTok{</ul>}
\NormalTok{18 }\KeywordTok{<footer>}\NormalTok{Nous sommes le }\KeywordTok{<em}\OtherTok{ class=}\StringTok{Rouge}\OtherTok{ id=}\StringTok{"date"}\KeywordTok{></em>} \KeywordTok{<script}\OtherTok{ type=}\StringTok{"text/javascript"}\KeywordTok{>} \OtherTok{window}\NormalTok{.}\FunctionTok{onload} \NormalTok{= }\FunctionTok{datejs}\NormalTok{(}\StringTok{'date'}\NormalTok{);<}\OtherTok{/script> et il est <em class=Rouge id="heure"></em}\NormalTok{><script type=}\StringTok{"text/javascript"}\NormalTok{>}\OtherTok{window}\NormalTok{.}\FunctionTok{onload} \NormalTok{= }\FunctionTok{heurejs}\NormalTok{(}\StringTok{'heure'}\NormalTok{);<}\OtherTok{/script>.</footer}\NormalTok{>}
\DecValTok{19} \NormalTok{<}\OtherTok{/body>}
\end{Highlighting}
\end{Shaded}

Si cela fonctionne alors on peut \textbf{enlever} dans
\textbf{\emph{accueuil.py}} la fonction \textbf{\emph{passer\_heure()}}
pour obtenir

\begin{Shaded}
\begin{Highlighting}[]
\DecValTok{03} \CharTok{from} \NormalTok{flask }\CharTok{import} \NormalTok{Flask, render_template, url_for}
\DecValTok{04} \NormalTok{app = Flask(}\OtherTok{__name__}\NormalTok{) }\CommentTok{# Initialise l'application Flask}
\DecValTok{05} 
\DecValTok{06} \NormalTok{@app.route(}\StringTok{'/'}\NormalTok{)}
\end{Highlighting}
\end{Shaded}

    \subsubsection{e) Les images}\label{e-les-images}

Ajoutons un favicon. Vous savez c'est la petite images à gauche de
l'adresse dans un navigateur. Le nom de l'image doit être
\textbf{\emph{favicon.ico}}.

J'ai déjà mis deux images dans le dossier \textbf{\emph{static/}}.\\Il
suffit donc de rajouter dans \textbf{\emph{templates/accueil.html}}

\begin{Shaded}
\begin{Highlighting}[]
\NormalTok{08 }\KeywordTok{<link}\OtherTok{ rel=}\StringTok{"shortcut icon"}\OtherTok{ href=}\StringTok{"\{\{ url_for('static' , filename='favicon.ico')\}\}"}\KeywordTok{>}
\NormalTok{09 }\KeywordTok{</head>}
\NormalTok{10 }\KeywordTok{<body>}
\NormalTok{11 }\KeywordTok{<header><h1>}\NormalTok{\{\{ titre \}\} }\KeywordTok{<img}\OtherTok{ src=}\StringTok{"\{\{ url_for('static',filename ='fleur.png') \}\}"}\OtherTok{ alt=}\StringTok{"fleur"}\OtherTok{ title=}\StringTok{"fleur"}\OtherTok{ border=}\StringTok{"0"}\KeywordTok{></h1></header>}
\NormalTok{12 }\KeywordTok{<section>}
\end{Highlighting}
\end{Shaded}

    \subsection{2) Les formulaires}\label{les-formulaires}

Dans le protocole \textbf{\emph{HTTP}}, une \textbf{\emph{méthode}} est
une Commande spécifiant un type de requête, c'est-à-dire qu'elle demande
au serveur d'effectuer une action. En général l'action concerne une
ressource identifiée par l'URL qui suit le nom de la méthode

\begin{itemize}
\itemsep1pt\parskip0pt\parsep0pt
\item
  La méthode \textbf{\emph{get}}:\\C'est la méthode la plus courante
  pour demander une ressource.
\item
  La méthode \textbf{\emph{post}}:\\Cette méthode est utilisée pour
  transmettre des données en vue d'un traitement à une ressource (le
  plus souvent depuis un formulaire HTML). L'URI fournie est l'URI d'une
  ressource à laquelle s'appliqueront les données envoyées. Le résultat
  peut être la création de nouvelles ressources ou la modification de
  ressources existantes.
\end{itemize}

\subsubsection{a) Exemple de formulaire utilisant la méthode
POST}\label{a-exemple-de-formulaire-utilisant-la-muxe9thode-post}

Une page internet utilise souvent des champs de formulaire avec la
balise \textbf{\emph{INPUT}} et il existe de nombreux champs de
formulaires. Regardez sur ce site quelques possibilités
\url{http://www.startyourdev.com/html/tag-html-balise-input}

Modifions la page \textbf{\emph{templates/accueil.html}} pour mettre un
\emph{formulaire}

\begin{Shaded}
\begin{Highlighting}[]
\NormalTok{17  }\KeywordTok{</ul>}
\NormalTok{18  }\KeywordTok{<div}\OtherTok{ id=}\StringTok{"content"}\KeywordTok{>}
\NormalTok{19     }\KeywordTok{<form}\OtherTok{ method=}\StringTok{"post"}\OtherTok{ action=}\StringTok{"\{\{ url_for('hello') \}\}"}\KeywordTok{>}
\NormalTok{20     }\KeywordTok{<label}\OtherTok{ for=}\StringTok{"nom"}\KeywordTok{>}\NormalTok{Entrez votre nom:}\KeywordTok{</label>}
\NormalTok{21     }\KeywordTok{<input}\OtherTok{ type=}\StringTok{"text"}\OtherTok{ name=}\StringTok{"nom"} \KeywordTok{/><br} \KeywordTok{/>}
\NormalTok{22     }\ErrorTok{<}\NormalTok{label for="prenom">Entrez votre prénom:}\KeywordTok{</label>}
\NormalTok{23     }\KeywordTok{<input}\OtherTok{ type=}\StringTok{"text"}\OtherTok{ name=}\StringTok{"prenom"} \KeywordTok{/><br} \KeywordTok{/>}
\NormalTok{24     }\KeywordTok{<input}\OtherTok{ type=}\StringTok{"submit"} \KeywordTok{/>}
\NormalTok{25     }\KeywordTok{</form>}
\NormalTok{26 }\KeywordTok{</div>}
\NormalTok{27 }\KeywordTok{</section>}
\end{Highlighting}
\end{Shaded}

pour le fichier \textbf{\emph{exemple.py}} on importe
\textbf{\emph{flask.request}}

\begin{Shaded}
\begin{Highlighting}[]
\DecValTok{03} \CharTok{from} \NormalTok{flask }\CharTok{import} \NormalTok{Flask, render_template, url_for, request}
\end{Highlighting}
\end{Shaded}

et on crée une nouvelle route.

\begin{Shaded}
\begin{Highlighting}[]
\DecValTok{09}      \KeywordTok{return} \NormalTok{render_template(}\StringTok{"accueil.html"}\NormalTok{, titre=}\StringTok{"Bienvenue !"}\NormalTok{,lignes=lignes)}
\DecValTok{10}
\DecValTok{11} \NormalTok{@app.route(}\StringTok{'/hello/'}\NormalTok{, methods=[}\StringTok{'POST'}\NormalTok{])}
\DecValTok{12} \KeywordTok{def} \NormalTok{hello():}
\DecValTok{13}     \NormalTok{nom=request.form[}\StringTok{'nom'}\NormalTok{]}
\DecValTok{14}     \NormalTok{prenom=request.form[}\StringTok{'prenom'}\NormalTok{]}
\DecValTok{15}     \KeywordTok{return} \NormalTok{render_template(}\StringTok{'page2.html'} \NormalTok{,titre=}\StringTok{"Page 2"}\NormalTok{, nom=nom, prenom=prenom)}
\end{Highlighting}
\end{Shaded}

Il ne reste plus qu'à créer le deuxième template
\textbf{\emph{templates/page2.html}} \emph{(pensez à copier les lignes
depuis templates/accueil.html)}.

\begin{Shaded}
\begin{Highlighting}[]
\NormalTok{01 }\DataTypeTok{<!DOCTYPE }\NormalTok{html}\DataTypeTok{>}
\NormalTok{02 }\KeywordTok{<html>}
\NormalTok{03 }\KeywordTok{<head>}
\NormalTok{04 }\KeywordTok{<meta}\OtherTok{ charset=}\StringTok{"utf-8"} \KeywordTok{/>}
\NormalTok{05 }\KeywordTok{<title>}\NormalTok{\{\{ titre \}\}}\KeywordTok{</title>}
\NormalTok{06 }\KeywordTok{<link}\OtherTok{ href=}\StringTok{"\{\{ url_for('static', filename='mon_style.css') \}\}"}\OtherTok{ rel=}\StringTok{"stylesheet"}\OtherTok{ type=}\StringTok{"text/css"} \KeywordTok{/>}
\NormalTok{07 }\KeywordTok{<script}\OtherTok{ type=}\StringTok{"text/javascript"}\OtherTok{ src=}\StringTok{"\{\{url_for('static', filename='mes_scripts.js') \}\}"}\KeywordTok{></script>}
\NormalTok{08 }\KeywordTok{<link}\OtherTok{ rel=}\StringTok{"shortcut icon"}\OtherTok{ href=}\StringTok{"\{\{ url_for('static', filename='favicon.ico') \}\}"}\KeywordTok{>}
\NormalTok{09 }\KeywordTok{</head>}
\NormalTok{10 }\KeywordTok{<body>}
\NormalTok{11 }\KeywordTok{<h1>}\NormalTok{\{\{ titre \}\} }\KeywordTok{<img}\OtherTok{ src=}\StringTok{"\{\{ url_for('static', filename = 'fleur.png') \}\}"}\OtherTok{  alt=}\StringTok{"fleur"}\OtherTok{ title=}\StringTok{"fleur"}\OtherTok{ border=}\StringTok{"0"}\KeywordTok{></h1>}
\NormalTok{12 }\KeywordTok{<section>}
\NormalTok{13 }\KeywordTok{<p>}\NormalTok{Bonjour }\KeywordTok{<b>}\NormalTok{\{\{ prenom \}\} \{\{ nom \}\}.}\KeywordTok{</b></p>}
\NormalTok{14 }\KeywordTok{</section>}
\NormalTok{15 }\KeywordTok{<footer>}\NormalTok{Nous sommes le }\KeywordTok{<em}\OtherTok{ class=}\StringTok{"Rouge"}\OtherTok{ id=}\StringTok{"date"}\KeywordTok{></em><script}\OtherTok{ type=}\StringTok{"text/javascript"}\KeywordTok{>}\OtherTok{window}\NormalTok{.}\FunctionTok{onload}\NormalTok{=}\FunctionTok{datejs}\NormalTok{(}\StringTok{'date'}\NormalTok{);<}\OtherTok{/script> et il est <em class="Rouge" id="heure"></em}\NormalTok{><script type=}\StringTok{"text/javascript"}\NormalTok{>}\OtherTok{window}\NormalTok{.}\FunctionTok{onload}\NormalTok{=}\FunctionTok{heurejs}\NormalTok{(}\StringTok{'heure'}\NormalTok{);<}\OtherTok{/script> heures. </footer}\NormalTok{>}
\DecValTok{16} \NormalTok{<}\OtherTok{/body>}
\OtherTok{17 </html}\NormalTok{>}
\end{Highlighting}
\end{Shaded}

    \subsubsection{b) Formulaire qui redirige vers la même
page}\label{b-formulaire-qui-redirige-vers-la-muxeame-page}

Il est nécessaire d'utiliser à la fois la méthode \textbf{\emph{POST}}
et la méthode \textbf{\emph{GET}}.\\La méthode \textbf{\emph{GET}} quand
on arrive sur la page, puis la méthode \textbf{\emph{POST}} quand on
transmet les informations.

Modifier le fichier \emph{exemple.py}

\begin{verbatim}
03 from flask import Flask, render_template, url_for, request
04 app = Flask(__name__) # Initialise l'application Flask
05 
06 @app.route('/', methods=['GET','POST'])  # On doit indiquer que l'on utilise les deux méthodes
07 def accueil():
08 lignes=['ligne {}'.format(i) for i in range(1,10)]
09 if request.method == 'GET':   # à la 1° connection c'est uné méthode GET
10     nom=''                    # La variable doit être définie
11     prenom=''
12     titre="Bienvenue !"
13 else:                         # à la 2° connection c'est une méthode POST
14     nom=request.form['nom']
15     prenom=request.form['prenom']
16     titre="Méthode POST"
17 return render_template("accueil.html", titre=titre,lignes=lignes,nom=nom,prenom=prenom)
18
19 if __name__ == '__main__':
20 app.run(debug=True)
\end{verbatim}

et le template acceuil.html

\begin{Shaded}
\begin{Highlighting}[]
\NormalTok{17 }\KeywordTok{</ul>}
\NormalTok{18 }
\NormalTok{19 \{% if nom == '' and prenom == '' %\}}
\NormalTok{20 }\KeywordTok{<div}\OtherTok{ id=}\StringTok{"content"}\KeywordTok{>}
\NormalTok{21      }\KeywordTok{<form}\OtherTok{ method=}\StringTok{"post"}\OtherTok{ action=}\StringTok{"\{\{ url_for('accueil') \}\}"}\KeywordTok{>}
\NormalTok{22      }\KeywordTok{<label}\OtherTok{ for=}\StringTok{"nom"}\KeywordTok{>}\NormalTok{Entrez votre nom:}\KeywordTok{</label>}
\NormalTok{23      }\KeywordTok{<input}\OtherTok{ type=}\StringTok{"text"}\OtherTok{ name=}\StringTok{"nom"} \KeywordTok{/><br} \KeywordTok{/>}
\NormalTok{24      }\ErrorTok{<}\NormalTok{label for="prenom">Entrez votre prénom:}\KeywordTok{</label>}
\NormalTok{25      }\KeywordTok{<input}\OtherTok{ type=}\StringTok{"text"}\OtherTok{ name=}\StringTok{"prenom"} \KeywordTok{/><br} \KeywordTok{/>}
\NormalTok{26      }\KeywordTok{<input}\OtherTok{ type=}\StringTok{"submit"} \KeywordTok{/>}
\NormalTok{27      }\KeywordTok{</form>}
\NormalTok{28 }\KeywordTok{</div>}
\NormalTok{29 \{% else %\}}
\NormalTok{30 }\KeywordTok{<p>}\NormalTok{bonjour \{\{ prenom \}\} \{\{ nom \}\}}\KeywordTok{</p>}
\NormalTok{31 \{% endif %\}}
\NormalTok{32 }\KeywordTok{</section>}
\end{Highlighting}
\end{Shaded}

    \subsection{3) Une partie commune à toutes les
pages}\label{une-partie-commune-uxe0-toutes-les-pages}

Nous avons plusieurs templates qui utilisent le même bas de page.\\Il
est donc possible d'utiliser un seul fichier template et de l'inclure
dans les autres. Créer un fichier \textbf{\emph{templates/footer.html}}

\begin{Shaded}
\begin{Highlighting}[]
\NormalTok{<footer>Nous sommes le <em }\KeywordTok{class}\NormalTok{=}\StringTok{"Rouge"} \DataTypeTok{id}\NormalTok{=}\StringTok{"date"}\NormalTok{></em> <script }\DataTypeTok{type}\NormalTok{=}\StringTok{"text/javascript"}\NormalTok{>window.onload = datejs(}\StringTok{'date'}\NormalTok{);</script> et il est <em }\KeywordTok{class}\NormalTok{=Rouge }\DataTypeTok{id}\NormalTok{=}\StringTok{"heure"}\NormalTok{></em><script }\DataTypeTok{type}\NormalTok{=}\StringTok{"text/javascript"}\NormalTok{>window.onload = heurejs(}\StringTok{'heure'}\NormalTok{);</script>.</footer>}
\end{Highlighting}
\end{Shaded}

Puis modifier les templates \textbf{\emph{templates/accueil.html}}

\begin{Shaded}
\begin{Highlighting}[]
\NormalTok{27 }\KeywordTok{</div>}
\NormalTok{28 \{% include 'footer.html' %\}}
\NormalTok{29 }\KeywordTok{</body>}
\end{Highlighting}
\end{Shaded}

et \textbf{\emph{templates/page2.html}}.

\begin{Shaded}
\begin{Highlighting}[]
\NormalTok{14 }\KeywordTok{<p>}\NormalTok{Bonjour }\KeywordTok{<b>}\NormalTok{\{\{ prenom \}\} \{\{ nom \}\}.}\KeywordTok{</b></p>}
\NormalTok{15 \{% include 'footer.html' %\}}
\NormalTok{16 }\KeywordTok{</body>}
\end{Highlighting}
\end{Shaded}

    \subsection{4) Mettre le site sur
internet}\label{mettre-le-site-sur-internet}

Habituellement on s'héberge soit même mais il est possible de mettre son
site chez un hébergeur. Il en existe plusieurs qui autorisent les
scripts python.

Ouvrez le navigateur pablo à l'adresse
\url{https://isn-flask.herokuapp.com/}\\Oui vous avez reconnu notre tp.

Pour héberger son site gratuitement on dispose de: *
\href{https://www.heroku.com/}{heroku} qui necessite l'utilisation de
\emph{git} pour poser le code. *
\href{https://www.pythonanywhere.com}{pythonanywhere}\\Je préfère
\textbf{\emph{heroku}} mais je conseillerai
\textbf{\emph{pythonanywhere}} pour les élèves d'isn pour sa simplicité
d'utilisation.

    \subsection{5) Prolongement}\label{prolongement}

\begin{itemize}
\itemsep1pt\parskip0pt\parsep0pt
\item
  Utilisation d'une base de donnée
\item
  Créer une image aux couleurs aléatoire (module pil) et l'afficher.
\end{itemize}

    \subsection{6) Exercices}\label{exercices}

\begin{enumerate}
\def\labelenumi{\arabic{enumi}.}
\item
  \begin{itemize}
  \itemsep1pt\parskip0pt\parsep0pt
  \item
    Créer un dossier \textbf{\emph{static/file/}} et mettre dans ce
    dossier quelques fichiers textes (.txt) et images (.png ou .jpg).\\
  \end{itemize}
\end{enumerate}

\begin{itemize}
\itemsep1pt\parskip0pt\parsep0pt
\item
  Créer dans le fichier \textbf{\emph{exemple.py}} une fonction
  \textbf{\emph{listdir()}} qui retourne la liste les noms des fichiers
  contenus dans le dossier \textbf{\emph{static/file/}} \emph{(On
  utilisera la librairie \textbf{os} pour lister)}.
\item
  modifier le \textbf{\emph{templates/accueil.html}} pour faire afficher
  le nom des fichiers.
\end{itemize}

\begin{enumerate}
\def\labelenumi{\arabic{enumi}.}
\setcounter{enumi}{1}
\itemsep1pt\parskip0pt\parsep0pt
\item
  Modifier alors la fonction \textbf{\emph{listdir()}} pour n'afficher
  que le nom des images.
\item
  Afficher cette fois les images dans la page internet.
\end{enumerate}

    \begin{Verbatim}[commandchars=\\\{\}]
{\color{incolor}In [{\color{incolor}}]:} 
\end{Verbatim}


    % Add a bibliography block to the postdoc
    
    
    
    \end{document}
