
%Fichier: algorithmique-python3-seconde.tex
%Crée le 05 juil. 2008
%Dernière modification: 06 nov. 2018 18:22:54

\documentclass[10pt,dvipsnames,  dvips]{article}
%%%%%%%% mes package %%%%%%%%%%%%%%%%%
\usepackage[utf8]{inputenc}
\usepackage[T1]{fontenc}
\usepackage{lmodern}
\usepackage{amssymb}  %  pour  leqslant 
\usepackage[verbose,a4paper,tmargin=2.2cm,bmargin=1.5cm,lmargin=1.3cm,rmargin=1.3cm]{geometry} %%%%pour fixer les marges du texte
%mise en page
%\usepackage{lscape} %%% pour localement utiliser \begin{landscape}...\end{landscape}
\usepackage{multicol} %%%%
\usepackage{ulem}
\usepackage{array}
\usepackage[dvips,ps2pdf]{hyperref}
\usepackage{verbatim}
\usepackage{textcomp} %%%Pour autoriser les caractères ° etc...
\usepackage{amsfonts,t1enc} %%% pour avoir les ensembles N Z Q R C
\usepackage{amsmath}%%%pour dfrac  etc...
\usepackage{pstricks,pst-plot}
\usepackage{xcolor} % utiliser par exemple black!20
\usepackage[dvips,ps2pdf]{hyperref} %%%Pour les liens
\usepackage{graphicx} %%%%



\usepackage{fancyhdr,fancybox}  %%%%pour les hauts et bas de pages
\usepackage{graphicx} %%%%pour inclure les graphiques
\usepackage{lastpage} %%%%pour inclure le le nombre total de page
%\usepackage{makeidx}
\usepackage[Bjornstrup]{fncychap}
\usepackage{alltt}

%les fontes
%\usepackage{lmodern}
%\usepackage{frcursive} %pour la fonte cursuive
\usepackage{pifont} % pour la fonte ding
\usepackage{textcomp} %%%Pour autoriser les caractères ° etc...

\usepackage{pst-eucl} %%% pour les dessin geométrique pstricks %%%
\usepackage{multido}
\newcounter{numeroexo}[section]
\newcommand{\exerciceun}[1]{\par\stepcounter{numeroexo}\hspace{-0.5cm}\underline{\textbf{Exercice \arabic{numeroexo}}}\quad \textit{#1}\:}





% mon fichier listing se télécharge ic http://megamaths.free.fr/pdf/mes_listings.tex
\input{mes_listings.tex} %%% pour les listings R-cran xcas etc... %%%%

\usepackage[frenchb]{babel}
%%%%%%%%            %%%%%%%%%%%%%%%%%
\hypersetup{
     backref=true,    %permet d'ajouter des liens dans...
     pagebackref=true,%...les bibliographies
     hyperindex=true, %ajoute des liens dans les index.
     colorlinks=true, %colorise les liens
     breaklinks=true, %permet le retour à la ligne dans les liens trop longs
     urlcolor= blue, %couleur des hyperliens
     linkcolor= blue, %couleur des liens internes
     bookmarks=true, %créé des signets
     bookmarksopen=false,  %si les signets Acrobat sont créés,
                          %les afficher complètement.
     pdftitle={aide mémoire}, %informations apparaissant dans
     pdfauthor={Meilland jean claude},     %dans les informations du document
     pdfsubject={Correspondance algorithmique python}       %sous Acrobat.
}



%\renewcommand{\vec}[1]{\overrightarrow{#1}}%écriture d'un vecteur
%\newcommand{\cc}[1]{\mathcal{C} _{#1}}%pour écrire Cf la courbe représentative
%\newcommand{\dd}[1]{\mathcal{D} _{#1}}%pour écrire Df le domaine de définition
%\newcommand{\R}{\mathbb{R}}
%\newcommand{\N}{\mathbb{N}}
%\newcommand{\C}{\mathbb{C}}
%\newcommand{\e}{\mathrm{e}}
\newcommand{\code}[1]{\fcolorbox{black}{cyan!10}{\lstinline!#1!}}

\newcounter{Chapter}
\newcounter{Sec}[Chapter]
\newcounter{Subsec}[Sec]
%\newcounter{Subsubsection}[Subsec]


%\newsavebox\boxofgeogebra

%\newcommand{\Chapter}[1]{\pdfbookmark[1]{#1}{chapter\theChapter}\stepcounter{Chapter}\chead{\colorbox{black!15}{ \Large\textbf{\Roman{Chapter} #1}}  }}
\newcommand{\Chapter}[1]{\stepcounter{Chapter}\chead{\colorbox{black!15}{ \Large\textbf{\Roman{Chapter} #1}}  }}


\newlength\taille
\newenvironment{Sec}[1]
{
\stepcounter{Sec}
\taille=\linewidth\advance\taille by -0.4cm % On réduit la largeur du cadre gris
\pdfbookmark[2]{#1}{Sec\theChapter-\theSec}
\begin{center}
\colorbox{black!15}{
	\begin{minipage}{\taille}
	\begin{flushleft}
	\textbf{
		\arabic{Sec}  #1
	}
	\end{flushleft}
	\end{minipage}\par
}\end{center}
}

\newcommand{\Subsec}[1]{\stepcounter{Subsec} \underline{\textbf{\alph{Subsec} #1} } \\}

%\setcounter{secnumdepth}{1}% enlève la numérotation après les sections
\usepackage{titlesec}


%redéfinition des sections
\titleformat{\section}
   {\large\selectfont\bfseries}% apparence commune au titre et au numéro  %\normalfont\fontsize{11pt}{13pt}
   {\thesection}% apparence du numéro
   {1em}% espacement numéro/texte
   {}% apparence du titre
\titlespacing{\subsubsection}{1pt}{1pt}{1pt}  %\titlespacing{\section}{espace horizontal}{espace avant}{espace après}


%redéfinition des subsubsections
\titleformat{\subsubsection}[hang]
   {\selectfont\bfseries}% apparence commune au titre et au numéro  %\normalfont\fontsize{11pt}{13pt}
   {\thesubsubsection}% apparence du numéro
   {0em}% espacement numéro/texte
   {\underline}% apparence du titre
\titlespacing{\subsubsection}{1pt}{0pt}{0pt}  %\titlespacing{\section}{espace horizontal}{espace avant}{espace après}

%\setlength{\parindent}{0pt}
\pagestyle{fancy} 
\lhead{\small\textsc{Lycée Pablo neruda}}
\chead{\small\textsc{Aide mémoire python seconde}}
\rhead{Page \thepage/\pageref{LastPage} }
\cfoot{}
\renewcommand{\headrulewidth}{1pt}
\renewcommand{\footrulewidth}{0pt}

% Pour les algorithmes


\usepackage[french,boxed,vlined]{algorithm2e}  %,linesnumbered,inoutnumbered
%SetKwData{fonction}{fonction: }
\SetKwInput{Fonction}{\textbf{fonction}}
\SetKwInput{Procedure}{\textbf{procédure} }
\SetKwInput{Variable}{\textbf{Variable}}
\SetKwInput{Constante}{\textbf{Constante}}
\SetKw{EntSor}{E/S}
\SetKw{Ent}{E}
\SetKw{Sor}{S}
%\SetKwInput{Variable}{Variable}
%\SetKw{fonction}{fonction:}
\SetKwSwitch{Selon}{Cas}{Autre}{selon}{}{cas où}{autres cas}{fin d'alternative}
\SetAlCapFnt{\color{red}}

%\setlength{\parskip}{2pt}
%Pour l'environnement multicolonne
%\setlength{\columnseprule}{0pt}
%\setlength{\columnsep}{0.5cm}

%\setlength{\parindent}{1cm} %
%\setlength{\parskip}{1cm plus4mm minus3mm}



%%%% debut macro figure à gauche \figgauche %%%%
 \newlength\jataille
 \newcommand{\figgauche}[3]%
 {\jataille=\linewidth\advance\jataille by -#1
 %\jataille=\textwidth\advance\jataille by -#1
 \advance\jataille by -1cm
 \begin{minipage}[t]{#1}
 #2
 \end{minipage}\hfill
 \begin{minipage}[t]{\jataille}
  #3
 \end{minipage}}
 %%%% fin macro %%%%

 %%%% debut macro figure à droite \figdroite%%%%
 \newlength\jatail
 \newcommand{\figdroite}[3]%
 {\jatail=\linewidth\advance\jatail by -#1
 \advance\jatail by -0.5cm
\begin{minipage}[t]{\jatail}
 #2
 \end{minipage}
 \hfill
\begin{minipage}[t]{#1}
  #3
\end{minipage}}

\setlength\parindent{0pt}%noident
\begin{document}
%Lextronic




\begin{list}{$\bullet$}{}
\item  Mes cours se trouvent à l'adresse \url{https://mybinder.org/v2/gh/debimax/cours-debimax/master}
\item Éditeur python en ligne:\url{https://repl.it/site/languages/python_turtle}
\end{list}

\section{Les bases de python3}

\subsubsection*{Les types}

Il y a  différents types:  entiers, flottants, complexes, chaînes de caractères, boléens, listes, tuples et dictionnaires.


%\begin{minipage}[3]{0.3\linewidth}

\begin{list}{-}{}
\item \textbf{\textit{entier (integer)}}: 3
\item \textbf{\textit{flotant (float)}}: 2.3
%\item \textbf{\textit{complexe}}: 1+2j
\item \textbf{\textit{chaînes de caractères (string)}}: 'ISN'
\item \textbf{\textit{Boléen}}: True
%\item \textbf{\textit{Listes}}: [0,1,2,3]
%\item \textbf{\textit{Tuples}}: (0,1,2,3)
%\item \textbf{\textit{Dictionnaires}}: {'zero':0, 'un':1, 'deux':2,'trois':3}
%\item \textbf{\textit{Byte}}: b'toto'
%\end{minipage}

\item  On aura toujours à l'esprit que $0.2+0.7=0.8999999999999999\neq 0.9$

 \item \textbf{\textit{int()}} permet de convertir, un nombre ou une chaîne de caractère en un entier.

 \item \textbf{\textit{float()}} permet de convertir, un nombre ou une chaîne de caractère en un flottant.
\item On utilise le caractère \# pour écrire un commentaire 
\end{list}

%Quelques opérateur déjà vu et très utiles




%\section{Les bases}

%\section{Les opérateurs}
%\underline{Les opérateurs}
\subsubsection*{Les opérateurs}

\begin{tabular}[]{| r |>{\raggedright}m{6 cm} | >{\raggedright}m{3cm} |}
\hline Opération & algorithme &  python\tabularnewline
\hline  Addition & 2+3 & 2+3\tabularnewline
\hline  Soustraction & 12-5 & 12-5\tabularnewline
\hline  Multiplication & 3*6 & 3*6\tabularnewline
\hline  Division & 7/2 & 7/2\tabularnewline
\hline  Quotient  de  la  division  euclidienne & 7 div  2  ou  div(7,2) & 7//2\tabularnewline
\hline  Reste  de  la  division euclidienne & 7  mod  2  ou  mod(7,2) & 7\%2\tabularnewline
\hline  puissance & 7\string^2 & 7**2\tabularnewline
\hline  racine  carrée & $\sqrt[]{2}$ ou  sqrt(2) & sqrt(2)\tabularnewline
\hline
\end{tabular}

\vspace{0.5cm}
\subsubsection*{Variables}

\begin{minipage}[t]{12cm}
\begin{list}{$\bullet$}{}
\item Python utilise le symbole $\boldsymbol{=}$ pour affecter une valeur à une variable.
\item Attention à ne pas confondre $a=12$ et $a==12$.
\end{list}
\end{minipage}
\hspace{0.5cm}
\begin{minipage}[t]{5cm}
\lstset{ style=PYTHON}
\vspace{-1cm}
\begin{lstlisting}
A=3; B=5
C='toto'
\end{lstlisting}
\end{minipage}

\subsubsection*{Entrées}


\begin{minipage}[t]{9cm}
\begin{list}{$\bullet$}{}
\item \textbf{\textit{input("texte")}} permet de saisir du texte pour un programme.
\item Il faudra éventuellement convertir ce texte dans le type voulu avec \textbf{\textit{int()}} ou \textbf{\textit{float()}}
\end{list}
\end{minipage}
\hspace{0.5cm}
\begin{minipage}[t]{8cm}
\lstset{ style=PYTHON}
\begin{lstlisting}
a=input("Saisir un texte a:  ")
b=int(input("Saisir un entier b:    "))
c=float(input("Saisir un réel c:    "))
\end{lstlisting}
\end{minipage}


\subsubsection*{Affichage}

\lstset{title={},caption={}, style=PYTHON}
\begin{list}{$\bullet$}{}
\item On utilise de préférence la méthode \textbf{\textit{format()}} pour afficher du texte et des variables.
\end{list}

\begin{lstlisting}
print("La valeur de la variable a est", a ,".")
print("La valeur de la variable a est " + str(a) + ".") # autre forme
print("La valeur de la variable a est {}.".format(a)) # je préfère
print("Le produit de {} par {} est {}".format(a,b,a*b))
print("{0}*{1}={2} et {0}/{1}={3}".format(a,b,a*b,a/b))
\end{lstlisting}

\subsubsection*{Connecteurs logiques}
\vspace{-0.3cm}\hspace{5cm}
%\begin{center}
\begin{minipage}[t]{5cm}
\begin{tabular}[]{|c |>{\centering}m{2cm} |}
\hline algorithmique & python \tabularnewline
\hline a = b  &   a==b\tabularnewline
\hline a$\neq $b    &  a!=b\tabularnewline
\hline a $\leqslant$ b  &   a<=b\tabularnewline
\hline
\end{tabular}
\end{minipage}\hspace{1cm}
\begin{minipage}[t]{5cm}
\begin{tabular}[]{|c |>{\centering}m{2cm} |}
\hline algorithmique & python \tabularnewline
\hline A et B &   A and B \tabularnewline
\hline A ou B  &  A or B \tabularnewline
%\hline A xor B  & A \string^ B \tabularnewline
\hline non A &  not(A)  \tabularnewline
\hline
\end{tabular}
\end{minipage}
%\end{center}


\subsubsection*{Fonction}
\begin{tabular}[]{>{\raggedright}m{0.35\linewidth} >{\raggedright}m{0.65\linewidth}}
Syntaxe: & 
Créer la fonction affine $f:x\longmapsto 3x+1$
et afficher les valeurs de $f(-5)$ à $f(5)$.\tabularnewline
%\lstset{title={ ''},caption={'syntaxe'}, style=PYTHON}
\begin{lstlisting}
def nomfonction(parametres):
	instructions
	instructions
	return valeur
\end{lstlisting}
 & 
 \lstset{title={},caption={}, style=PYTHON}
\begin{lstlisting}
def f(x):
	return 3*x+1
for i in range(-5,6):
	print(f(i))
\end{lstlisting}
 \tabularnewline
\end{tabular} 



\subsubsection*{Condition SI}

%\begin{minipage}[t]{5.6cm}
%\begin{algorithm}[H]
%\Deb{
%\Si{<~condition~>}{<~instructions~>}
%}
%\end{algorithm}
%\end{minipage}
%\hspace{0.5cm}
%\begin{minipage}[t]{5.6cm}
%\begin{algorithm}[H]
%\Si{<~condition~>}
%{<~instructions~>}
%\Sinon{<~autre instructions~>}
%\end{algorithm}
%\end{minipage}
%\hspace{0.5cm}
%\begin{minipage}[t]{5.6cm}
%\begin{algorithm}[H]
%\Deb{
%\Si{<~condition 1~>}
%{<~instructions 1~>}
%\eSi{<~condition 2~>}{<~instructions 2~>}{<~autre instructions~>}
%}
%\end{algorithm}
%\end{minipage}\vspace{0.2cm}

\hspace{-0.5cm}
\begin{minipage}[t]{5.6cm}
\begin{lstlisting}
if x<0:
	print(']-inf;0[')
\end{lstlisting}
\end{minipage}
\hspace{0.1cm}
\begin{minipage}[t]{6cm}
\begin{lstlisting}
if x<0:
	print(']-inf;0[')
else:
	print('[0;+inf[')
\end{lstlisting}
\end{minipage}
\hspace{0.1cm}
\begin{minipage}[t]{6.7cm}
\begin{lstlisting}
if x<0:
	print(']-inf;0[')
elif 0<=x<=20:
	print('[0;20]')
else:
	print(']20;+inf[')
\end{lstlisting}
\end{minipage}

%\newpage
Exemple:\\
\begin{lstlisting}
print("Saisissez deux valeurs numériques")
a=float(input("Saisir a: "))
b=float(input("Saisir b: "))
if a==b :
	print("Vous avez saisi deux fois la même valeur, à savoir {}.".format(a))
else :
	print("Vous avez saisi deux valeurs différentes {} et {}.".format(a,b))
\end{lstlisting}


\subsubsection*{Boucle pour}

\begin{lstlisting}
for i in range(7):     # pour i allant de de 0 à 6
	print(i)
for i in range(1,7):     # pour i allant de de 1 à 6
	print(i)
for i in range(1,6,2):    # pour i allant de de 1 à 6 par pas de 2 donc: 1 3 5
	print(i)
\end{lstlisting}

La syntaxe générale est \textbf{\textit{for i in range(m,n,p)}}:

\textbf{\textit{i}} prend alors toutes les valeurs de \textbf{\textit{m à n-1 par pas de p}}

\subsubsection*{Tant que}

\begin{lstlisting}
i=1
while i<=5:
	print(i)
	i=i+1 #où en plus concis i+=1
# Affichage: 1 2 3 4 5
# À la sortie de la boucle i=6
\end{lstlisting}




\subsubsection*{Turtle}


\begin{tabular}[]{| r | >{\raggedright}m{12cm} |}
\hline \textbf{reset()} & On efface tout et on recommence  \tabularnewline
\hline \textbf{goto(x,y)} & Aller à l’endroit de coordonnées x et y  \tabularnewline
\hline \textbf{forward(distance)} & Avancer d’une distance donnée\tabularnewline
\hline \textbf{backward(distance)} & Reculer  \tabularnewline
\hline \textbf{up()}  &  Relever le crayon (pour pouvoir avancer sans dessiner)  \tabularnewline
\hline \textbf{down()}  & Abaisser le crayon (pour pouvoir recommencer à dessiner)  \tabularnewline
\hline \textbf{color(couleur)} &  Couleur peut être une chaîne prédéfinie (’red’, ’blue’, ’green’, etc.)  \tabularnewline
\hline \textbf{left(angle)}  & Tourner à gauche d’un angle donné (exprimé en degré)  \tabularnewline
\hline \textbf{right(angle)}  & Tourner à droite  \tabularnewline
\hline \textbf{width(épaisseur)}  & Choisir l’épaisseur du tracé  \tabularnewline
\hline \textbf{fill(1)}  & Remplir un contour fermé à l’aide de la couleur sélectionnée (on termine la construction par fill(0))  \tabularnewline
\hline \textbf{write(texte)}  &  texte doit être une chaîne de caractères délimitée avec des " ou des ’\tabularnewline
\hline
\end{tabular} 




\newpage
\section{Turtle}

\begin{list}{$\bullet$}{}
\item Importer le module \textbf{turtle}  
\lstset{title={},caption={}, style=PYTHON}
\begin{lstlisting}
import turtle
\end{lstlisting}
\item Création d'une fenètre de largeur  (\textbf{width})  600px, de hauteur (\textbf{height}) 400px, \\
	\lstset{title={},caption={}, style=PYTHON}
\begin{lstlisting}
turtle.setup(640, 480) 
\end{lstlisting}
\end{list}
	%la largeur (\textbf{width}) de notre fenêtre, sa hauteur (\textbf{height}), la position en largeur (\textbf{startx})% puis la position en hauteur (\textbf{starty}) du coin en haut à gauche de notre fenêtre par rapport au coin en haut à gauche de l'écran.

\begin{center}
\raisebox{-0.1cm}{\includegraphics[width=6cm]{images/turtle}}
\end{center}

\subsection{ Mouvements de la tortue}

\begin{list}{$\bullet$}{}
\item  \textbf{forward(d)} : fait avancer la tortue de d (en pixel), le trait est dessiné si le crayon est baissé ;
\item  \textbf{backward(d)} : fait reculer la tortue de d (en pixel), le trait est dessiné si le crayon est baissé ;
\item  \textbf{left(a)} : fait pivoter la tortue d’un angle de a degrés vers la gauche ;
\item  \textbf{right(a)} : fait pivoter la tortue d’un angle de a degrés vers la droite ;
\item  \textbf{goto(x,y)} : la tortue va se positionner au point de coordonnées ( x ; y );
\item  \textbf{circle(r)} : trace un cercle de rayon r , le point de départ de la tortue appartient au cercle (attention il n’est
pas centré sur la position de la tortue) ;
\item  \textbf{circle(r,s)} : trace une portion du cercle correspondant à s degrés ;
\end{list}

\subsection{Contrôle du stylo}

\begin{list}{$\bullet$}{}
\item  \textbf{up()} : lève le crayon ;
\item  \textbf{down()} : baisse le crayon ;
\item  \textbf{pensize()} ou \textbf{width()} : fixe la largeur du trait (en pixel) ;
\item  \textbf{reset()} : nettoie la fenêtre de dessin, réinitialise la tortue ; elle est située alors au centre de l’écran de
dessin tournée vers la droite.
\item  \textbf{pencolor(c)} : la couleur par défaut est le noir, on peut la changer en mettant une couleur prédéfinie
"red" , "green" , "blue" , "yellow" , . . . ;
\item  \textbf{color(c1,c2)} : modifie la couleur du trait c1 et la couleur du remplissage c2 . On peut aussi les
modifier séparement avec pencolor(c) et fillcolor(c) .
\item  \textbf{begin\string_fill()} et \textbf{end\_fill()} permettent de commencer et de terminer le remplissage d’une
figure géométrique.
\end{list}

\newpage
\subsection{TP}


\exerciceun{}


\begin{enumerate}
\item Dessiner un triangle équilatéral de coté 100 pixel.
\item Écrire la fonction \textbf{\textit{triangle1(cote)}} qui dessine un triangle équilatéral dont les côtés sont de longueur \textit{cote} et qui a la pointe vers le haut.
\item Écrire la fonction \textbf{\textit{triangle2(cote)}} qui dessine un triangle équilatéral dont les côtés sont de longueur \textit{cote} et qui a la pointe vers le bas.
\item Écrire la fonction \textbf{\textit{triangle3(cote,angle)}} qui dessine un triangle équilatéral dont les côtés sont de longueur \textit{cote} et d’une orientation bien déterminée.
\end{enumerate}



\exerciceun{}

\begin{enumerate}
	\item Écrire la fonction \textbf{\textit{carre(cote)}} qui trace un carre de côté cote . Il est préférable que la tortue termine son dessin là où elle a démarré et avec la même orientation.
\item En déduire la fonction \textbf{ligne\_de\_carres(n,cote)} qui trace n carrés sur une ligne chaque carré étant de coté \textbf{cote} (on utilisera la fonction carre ).
\begin{center}
\psset{xunit=1cm , yunit=1cm}
\begin{pspicture*}(-.5,-0.1)(10.5,1)
	\multido{\n=0+1}{7}{\put(\n,0){ \pspolygon(0,0)(0.8,0)(0.8,0.8)(0,0.8) }}
	\end{pspicture*}
\end{center}
\item Écrire la fonction \textbf{\textit{carres\string_croissants(n,cote)}} qui trace une ligne de n carrés, le premier carré étant de côté cote , le suivant de côté 1,25 fois la taille du côté du carré qui le précède ; les carrés seront espacés la première fois de cote/4 puis cette distance sera multipliée aussi par 1,25 à chaque fois.\\
	Vous utiliserez la fonction carre mais pas \textbf{\textit{ligne\string_de\string_carres}} .
\end{enumerate}


\end{document}

